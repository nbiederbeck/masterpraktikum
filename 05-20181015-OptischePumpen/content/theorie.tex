\section{Theorie}%
\label{sec:theorie}

Entartete Energieniveaus spalten sich in einem angelegten Magnetfeld auf.
Dies wird Zeeman-Effekt genannt.

In diesem Experiment wird die Energiedifferenz der Zeeman-Aufspaltung in
Rubidium durch Optisches Pumpen ausgemessen.


\subsection{Termschema und Zeeman-Aufspaltung}%
\label{sub:termschema_und_zeeman_aufspaltung}

Rubidium gehört zu ersten Hauptgruppe und ist somit ein Alkali-Element.
Alkali-Elemente besitzen auf der äußersten Schale ein freies Elektron.
Dies vereinfacht die folgenden Betrachtungen, da jede abgeschlossene Schale
keinen Netto-Drehimpuls besitzt.

Der Gesamtdrehimpuls $\vec{J}$ der Elektronenhülle eines Atoms entspricht also
dem eines einzelnen Elektrons.
Er ist mit einem magnetischen Moment $\vec{\mu}_{\text{J}}$ gekoppelt.
Der Zusammenhang lautet
\begin{equation}
  \vec{\mu}_{\text{J}} = -g_{\text{J}} \mu_{\text{B}} \vec{J}.
\end{equation}
Hier ist
\begin{equation}
  \mu_{\text{B}} = \frac{e \hbar}{2 m_\text{e}}
\end{equation}
das Bohrsche Magneton und $g_{\text{J}}$ der Land\'e-Faktor.

Der Gesamtdrehimpuls setzt sich zusammen aus dem Bahndrehimpuls $\vec{L}$ und
dem Spin $\vec{S}$.
Es gelten
\begin{align}
  \vec{\mu}_{\text{L}} &= -\mu_{\text{B}} \vec{L}, \\
  \vec{\mu}_{\text{S}} &= -g_{\text{S}} \mu_{\text{B}} \vec{S}, \\
  \intertext{sowie}
  \vec{\mu}_{\text{J}} &= \vec{\mu}_{\text{S}} + \vec{\mu}_{\text{L}},
\end{align}
wobei $g_{\text{S}} = \num{2}$ der Land\'e-Faktor des freien Elektrons ist.
Es werden außerdem die Quantenzahlen $j, s, l$ definiert, die zu den jeweiligen Spins
gehören.

Beim Anlegen eines äußeren Magnetfeldes $\vec{B}$ wechselwirkt das magnetische Moment mit diesem.
Das mag. Moment $\vec{\mu}_{\text{J}}$ präzidiert um die Feldrichtung von
$\vec{B}$ und somit mittelt sich die senkrechte Komponente raus.
Aufgrund der Richtungsquantelung kann diese nur ganzzahlige Werte annehmen:
\begin{equation}
  E_\text{mag} = m_\text{j} g_{\text{J}} \mu_\text{B} B,
\end{equation}
wobei $m_\text{j}$ die Werte $-j \ldots +j$ annehmen kann.

Es spalten sich also die Energieniveaus beim Anlegen eines Feldes in $2j + 1$ Unterniveaus auf.
Dies wird als Zeeman-Effekt bezeichnet.




\begin{figure}[ht]
  \centering
  \includegraphics[width=0.8\linewidth]{build/aufspaltungen.pdf}
  \caption{%
    Zeeman-Aufspaltungen eines Alkali-Atoms mit Kernspin $\mathbf{I}
    = \sfrac{3}{2}$.\cite{anleitung}%
  }%
  \label{fig:aufspaltungen}
\end{figure}

\textit{Spin, Kernspin, Bahndrehimpuls, Land\'e-Faktor, Bohrsches Magneton,
Richtungsquantelung, Hyperfeinstruktur}

\subsection{Optisches Pumpen}%
\label{sub:optisches_pumpen}

\textit{Boltzmann-Verteilung, Besetzungsinversion, Pumpvorgang, Auswahlregeln,
Spontane/Induzierte Emission, Polarisation, Transparenz}

\textit{Quadratischer Zeeman-Effekt}


\subsection{Transiente Effekte}%
\label{sub:transiente_effekte}

\textit{Larmor-Frequenz, Anregung, e-Funktion}
