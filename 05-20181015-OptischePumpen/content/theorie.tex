% todo:
% Kapitel 1.1 (Termschema und Zeeman-Aufspaltung):
%   Spin, Kernspin, Bahndrehimpuls, Lande-Faktor,
%   Bohrsches Magneton Richtungsquantelung, Hyperfeinstruktur
% Kapitel 1.2 (Optisches Pumpen):
%   Boltzmann-Verteilung, Besetzungsinversion, Pumpvorgang,
%   Auswahlregeln, Spontane/Induzierte Emission,
%   Polarisation, Transparenz
% Kapitel 1.4 (Transiente Effekte):
%   Larmor-Frequenz, Anregung, e-Funktion
\section{Theorie}%
\label{sec:theorie}

Entartete Energieniveaus spalten sich in einem angelegten Magnetfeld auf.
Dies wird Zeeman-Effekt genannt.

In diesem Experiment wird die Energiedifferenz der Zeeman-Aufspaltung in
Rubidium durch Optisches Pumpen ausgemessen.


\subsection{Termschema und Zeeman-Aufspaltung}%
\label{sub:termschema_und_zeeman_aufspaltung}

Rubidium gehört zu ersten Hauptgruppe und ist somit ein Alkali-Element.
Alkali-Elemente besitzen auf der äußersten Schale ein freies Elektron.
Dies vereinfacht die folgenden Betrachtungen, da jede abgeschlossene Schale
keinen Netto-Drehimpuls besitzt.

Der Gesamtdrehimpuls $\vec{J}$ der Elektronenhülle eines Atoms entspricht also
dem eines einzelnen Elektrons.
Er ist mit einem magnetischen Moment $\vec{\mu}_{\text{J}}$ gekoppelt.
Der Zusammenhang lautet
\begin{equation}
  \vec{\mu}_{\text{J}} = -g_{\text{J}} \mu_{\text{B}} \vec{J}.
\end{equation}
Hier ist
\begin{equation}
  \mu_{\text{B}} = \frac{e \hbar}{2 m_\text{e}}
\end{equation}
das Bohrsche Magneton und $g_{\text{J}}$ der Land\'e-Faktor.

Der Gesamtdrehimpuls setzt sich zusammen aus dem Bahndrehimpuls $\vec{L}$ und
dem Spin $\vec{S}$.
Es gelten
\begin{align}
  \vec{\mu}_{\text{L}} &= -\mu_{\text{B}} \vec{L}, \\
  \vec{\mu}_{\text{S}} &= -g_{\text{S}} \mu_{\text{B}} \vec{S}, \\
  \intertext{sowie}
  \vec{\mu}_{\text{J}} &= \vec{\mu}_{\text{S}} + \vec{\mu}_{\text{L}},
\end{align}
wobei $g_{\text{S}} = \num{2}$ der Land\'e-Faktor des freien Elektrons ist.
Es werden außerdem die Quantenzahlen $j, s, l$ definiert, die zu den jeweiligen Spins
gehören.

Beim Anlegen eines äußeren Magnetfeldes $\vec{B}$ wechselwirkt das magnetische Moment mit diesem.
Das magnetische Moment $\vec{\mu}_{\text{J}}$ präzidiert um die Feldrichtung von
$\vec{B}$ und somit mittelt sich die senkrechte Komponente raus.
Aufgrund der Richtungsquantelung kann die parallele Komponente nur ganzzahlige Werte annehmen:
\begin{equation}
  E_\text{mag} = m_\text{j} g_{\text{J}} \mu_\text{B} B,
\end{equation}
wobei $m_\text{j}$ die Werte $-j,~\ldots,~+j$ annehmen kann.

Es spalten sich also die Energieniveaus beim Anlegen eines Feldes in $2j + 1$ Unterniveaus auf.
Dies wird als Zeeman-Effekt bezeichnet.


\subsubsection{Einfluss des Kernspins}%
\label{sub:einfluss_des_kernspins}

Die eben angestellten Betrachtungen setzen einen Kernspin von $\mathbf{I} = 0$
voraus.
Dies ist im Allgemeinen nicht der Fall.
Auch in diesem Experiment haben die Rubidiumisotope einen von null verschiedenen Kernspin
$\mathbf{I} \neq 0$.

Es wird eine neue Quantenzahl $m_\text{f}$ für den Gesamtdrehimpuls
\begin{equation}
  \vec{F} = \vec{J} + \vec{\mathbf{I}}
\end{equation}
eingeführt, welche von $-F$ bis $F$ laufen kann.
Hierbei kann $F$ die Werte
$\left|\vec{\mathbf{I}} - \vec{J}\right|$ bis $\vec{\mathbf{I}} + \vec{J}$
annehmen.

Die Aufspaltung der Energieniveaus in Folge des Kernspins heißt
Hyperfeinstruktur, die weitere Aufspaltung in $2F + 1$ Unterniveaus ist dann
der Zeeman-Effekt bei Anlegen eines äußeren Magnetfeldes.

In Abbildung~\ref{fig:aufspaltungen} sind die Aufspaltungen beispielhaft
dargestellt.

\begin{figure}[ht]
  \centering
  \includegraphics[width=0.8\linewidth]{build/aufspaltungen.pdf}
  \caption{%
    Zeeman-Aufspaltungen eines Alkali-Atoms mit Kernspin $\mathbf{I}
    = \sfrac{3}{2}$.\cite{anleitung}%
  }%
  \label{fig:aufspaltungen}
\end{figure}


\subsection{Optisches Pumpen}%
\label{sub:optisches_pumpen}

Das Optische Pumpen wird verwendet, um eine Besetzungsinversion zu erhalten.

Im thermischen Gleichgewicht ist der Grundzustand immer höher besetzt, als die
anderen Zustände.
Es muss eine Besetzungsinversion hergestellt werden, um Relaxationsvorgänge im
Medium beobachten zu können.

Diese Relaxationsvorgänge sind induzierte und spontane Emission von
Lichtquanten.
Aufgrund der Energie-Zeit-Unschärfe können höherenergetische Elektronen die
niedrigeren Energiezustände sehen und sich auf diese setzen.
Das ist die spontane Emission.
Bei der induzierten Emission werden Elektronen von Photonen angeregt,
und das bei der Relaxation entstehende Photon ist mit dem induzierenden Photon
in Phase und hat die selbe Wellenlänge.

Damit induzierte Emission stattfinden kann, muss die Energie des eingestrahlten Photon
genau der Energiedifferenz zwischen zwei Zuständen entsprechen:
\begin{align}
  h \nu &= E_2 - E_1, \\
  \intertext{was bei benachbarten Zeeman-Niveaus}
  \label{eq:zeeman_energie}
  h \nu &= g_\text{F} \mu_\text{B} B
\end{align}
entspricht.

Ein Alkali-Atom hat den Grundzustand
${}^2\!S_{\sfrac{1}{2}}$
und die beiden angeregten Zustände
${}^2\!P_{\sfrac{1}{2}}$
und
${}^2\!P_{\sfrac{3}{2}}$
(vgl. Abbildung~\ref{fig:dublett}).
Im vorliegenden Experiment wird nur $D_1$-Licht betrachtet,
sodass nur der Grundzustand und der erste angeregte Zustand im Folgenden
betrachtet werden.
Aufgrund des Gesamtdrehimpulses, der in beiden Zuständen $J = \frac{1}{2}$ ist,
kann $m_\text{J}$ nur die Werte $\pm\sfrac{1}{2}$ annehmen.
Somit werden die Zustände bei angelegtem äußerem Magnetfeld in zwei
Zeeman-Unterniveaus gespalten.

\begin{figure}[ht]
  \centering
  \includegraphics[width=0.6\linewidth]{build/dublett.pdf}
  \caption{%
    Dublettstruktur in Alkali-Spektren.\cite{anleitung}
    Es ist das typische $D_1$-$D_2$-Dublett zu sehen.
    Die Quantenzahlen sind eingetragen.
  }%
  \label{fig:dublett}
\end{figure}

\begin{figure}[ht]
  \centering
  \includegraphics[width=0.6\linewidth]{build/auswahlregeln.pdf}
  \caption{%
    Zeeman-Aufspaltung des Grundzustandes und ersten angeregten
  Zustandes eines Alkali-Atoms ohne Kernspin.\cite{anleitung}
  }%
  \label{fig:auswahlregeln}
\end{figure}

Zwischen diesen Unterniveaus existieren vier Übergänge, siehe hierzu
Abbildung~\ref{fig:auswahlregeln}.
Es sind mit $\Delta m_\text{J} = 0$ die $\pi$-Übergänge linear polarisierten
Lichtes gekennzeichnet und bei den $\sigma$-Übergängen handelt es sich um
zirkular polarisiertes Licht.

Wird nun rechtszirkular polarisiertes $D_1$-Licht in eine Dampfzelle von Rubidium
gestrahlt, so werden die Elektronen im untersten Zustand
(${}^2\!S_{\sfrac{1}{2}}$, $m_\text{J} = - \sfrac{1}{2}$)
in den obersten
(${}^2\!P_{\sfrac{1}{2}}$, $m_\text{J} = + \sfrac{1}{2}$)
gehoben.
Elektronen in den angeregten Zuständen können durch spontane Emission wieder
in niedrigere Zustände fallen,
da aber der Grundzustand ständig gepumpt wird,
leert sich dieser zunehmend.
In Folge dessen kann das $D_1$-Licht nicht mehr absorbiert werden und das Gas
wird durchsichtig (vgl. Abbildung~\ref{fig:transparenz}).

\begin{figure}[ht]
  \centering
  \includegraphics[width=0.6\linewidth]{build/transparenz.pdf}
  \caption{%
    Zunehmende Transparenz einer Alkali-Dampfzelle.\cite{anleitung}
    Bei Einstrahlung von $D_1$-Licht wird der ($m_\text{J}
    = - \sfrac{1}{2}$)-Zustand leer gepumpt, sodass keine weiteren Photonen
    absorbiert werden können.
  }
  \label{fig:transparenz}
\end{figure}




\subsection{Präzisionsmessung der Zeeman-Aufspaltung}%
\label{sub:prazisionsmessung_der_zeeman_aufspaltung}

Bei einem nicht vorhandenen Magnetfeld ist die Dampfzelle undurchsichtig,
da die Energieniveaus nicht aufgespalten sind und somit der Grundzustand nicht
leergepumpt werden kann.
Somit lässt sich durch Einstellen von externen Magnetfeldern das Erdmagnetfeld
kompensieren und ausmessen.

Bei Anlegen eines frequenzvariablen Hochfrequenzfeldes (RF) spalten die
Energieniveaus auf
und der Grundzustand kann gepumpt werden.
Erreicht das Magnetfeld den Wert
\begin{align}
  B_\text{m} &= \frac{4 \pi m_e}{e g_\text{J}} \nu, \\
  \intertext{also gerade die Energiedifferenz}
  h \nu &= g_\text{J} \mu_\text{B} B_\text{m} \Delta m_\text{J},
\end{align}
so wird Elektronenemission induziert und der Grundzustand wird teilweise wieder
bevölkert.
Dieser wird jedoch gleichzeitig auch weiter gepumpt, sodass eine
Resonanzstelle wie in Abbildung~\ref{fig:resonanz} sichtbar wird.

\begin{figure}[ht]
  \centering
  \includegraphics[width=0.7\linewidth]{build/resonanz.pdf}
  \caption{%
    Transparenz der Dampfzelle in Abhängigkeit des angelegten
    Hochfrequenzfeldes.\cite{anleitung}
    Der Dip bei $B_\text{m}$ ist die Resonanzstelle, bei der die
    Energiedifferenz gerade der Energie der eingestrahlten Photonen entspricht.
  }%
  \label{fig:resonanz}
\end{figure}


Bei hoher Magnetfeldern müssen Terme höherer Ordnung berücksichtigt werden.
So lautet die Zeeman-Energie statt~\eqref{eq:zeeman_energie}
\begin{equation}
  E = g_\text{F} \mu_\text{B} B + {g_\text{F}}^2 {\mu_\text{B}}^2 {B}^2 \frac{\left(1
  - 2 m_\text{F} \right)}{\Delta E_\text{Hy}},
\end{equation}
wobei $\Delta E_\text{Hy}$ die Hyperfeinstrukturaufspaltung zwischen den
Niveaus $F$ und $F + 1$ ist.
Dieser sogenannte \textit{Quadratische Zeeman-Effekt} ist also abhängig von
der Quantenzahl $m_\text{F}$.


\subsection{Transiente Effekte}%
\label{sub:transiente_effekte}

Wird das RF-Feld schnell ein- und ausgeschaltet, während seine Frequenz auf eine
Resonanz eingestellt ist, und es senkrecht zum äußeren Magnetfeld steht,
so stellen sich sogenannte transiente Effekte ein.

Die resonante Frequenz ist
\begin{align}
  \omega_0 &= 2 \pi \nu_0 = g_\text{F} \frac{\mu_0}{h} B_0, \\
  \intertext{wobei unter der Verwendung des gyromagnetischen Verhältnis}
  \gamma &= g_\text{F} \frac{\mu_0}{h} \\
  \intertext{die Larmor-Frequenz definiert wird:}
  \omega_0 &= \gamma B_0.
\end{align}

Beim Anschalten des senkrechten RF-Feldes wird der Elektronenspin aus der
Ruhelage im Magnetfeld ausgelenkt.
Aufgrund des Eigenspins stellt sich eine Präzessionsbewegung ein.

Diese Präzessionsbewegung kann beschrieben werden, indem in das Eigensystem des
Elektrons gewechselt wird.
Hier lässt sich das linear oszillierende Magnetfeld durch zwei überlagerte
gegenläufig zirkular polarisierte Magnetfelder beschreiben.
Dabei ist
\begin{align}
  B_\text{eff} &= B + \frac{\omega}{\gamma}.
\end{align}

Dabei ist das Feld, das mit~$+\omega$ rotiert, in Phase mit der
Präzession und das Feld mit~$-\omega$ so schnell, dass der Effekt auf das
Elektron im Mittel Null ist.

Da im rotierten Koordinatensystem der Gesamtspin $F$ mit der Larmor-Frequenz um
$B_\text{RF} = B_\text{eff}$ rotiert, ist die Periode
\begin{align}
  \label{eq:larmor_periode}
  T &= \frac{1}{\gamma B_\text{RF}}.
\end{align}
Die Periode nimmt offenbar mit steigendem Magnetfeld ab.

Im Resonanzfall ist also die Relation der Perioden bei verschiedenen Rubidiumisotopen
\begin{align}
  \label{eq:resonanz_perioden}
  \frac{T_{87}}{T_{85}} &= \frac{\gamma_{85}}{\gamma_{87}}.
\end{align}


\subsection{Helmholtzspule}%
\label{sub:helmholtzspule}
Das Magnetfeld im Zentrum einer Helmholtzspule ist
\begin{align}
  B(0) &= \mu_{0} \cdot \frac{8 \cdot I \cdot N}{\sqrt{125} \cdot R}.
\end{align}
