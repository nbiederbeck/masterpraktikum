\section{Durchführung}%
\label{sec:durchfuehrung}

\subsection{Versuchsaufbau}%
\label{sub:versuchsaufbau}

Der Versuchsaufbau ist in Abbildung~\ref{fig:aufbau} dargestellt.

Eine Rubidiumgaslampe wird als Spektrallampe verwendet, da diese
unter anderem das benötigte $D_1$\=/Licht ausstrahlt.
Das Licht wird mittels Sammellinse kollimiert.
Anschließend wird mittels Interferenzfilter das $D_1$\=/Licht gefiltert.
Mittels Polarisationsfilter und $\lambda/4$-Platte wird zirkular polarisiertes
Licht erzeugt.

Das Licht fällt in eine Dampfzelle, die auf \SI{50}{\celsius} erhitzt ist.
Das Erhitzen führt dazu, dass die Energieniveaus angereichert werden.
Das aus der Dampfzelle austretende Licht wird wieder fokussiert und auf ein
Si-Photoelement gegeben.
Dieses ist über einen Linearverstärker an ein Oszilloskop angeschlossen.

\begin{figure}[ht]
  \centering
  \includegraphics[width=0.8\linewidth]{build/aufbau.pdf}
  \caption{Schematische Darstellung des Versuchsaufbaus.~\cite{anleitung}}%
  \label{fig:aufbau}
\end{figure}

Es werden drei Helmhotzspulenpaare in dem Versuchsaufbau verwendet.
Eine Vertikalfeldspule mit
$R = \SI{11.735}{\centi\meter}$ und $N = \num{20}$
wird verwendet, um die vertikale Komponente des Erdmagnetfelds zu annulieren
und so auszumessen.

Die horizontale Komponente des Erdmagnetfelds wird grob über die Ausrichtung
des Aufbaus im Raum kompensiert,
feiner über die Horizontalfeldspule mit
$R = \SI{15.97}{\centi\meter}$ und $N = \num{154}$.

Die dritte Spule steht auch horizontal und wird als Sweep-Spule verwendet.
Sie hat
$R = \SI{16.39}{\centi\meter}$ und $N = \num{11}$
und liegt direkt auf der Horizontalfeldspule.
Mit der Sweep-Spule kann ein Magnetfeldbereich abgefahren werden.


\subsection{Messprogramm}%
\label{sub:messprogramm}

Zuerst wird die Strahlintensität maximiert, indem die Linsen in die Fokuspunkte
gestellt werden.
Dann werden die weiteren optischen Elemente eingesetzt und der Aufbau mit einer
schwarzen Decke abgedeckt.

An dem Oszilloskop wird der Recorder-Ausgang der Sweep-Spule auf Kanal~1 und die
Photodiodenspannung auf Kanal~2 gegeben.
Das Display wird auf das $XY$\=/Format gestellt.
Es sollte ein Dip wie in Abbildung~\ref{fig:resonanz} zu sehen sein.
Davon wird ein Foto angefertigt.
Zur Kompensation des Erdmagnetfelds wird dieser Dip auf eine minimale Breite
eingestellt.


\subsubsection{Bestimmung der Kernspins}%
\label{sub:bestimmung_der_kernspins}

Die Frequenzen an der RF-Spule werden zwischen \SI{100}{\kilo\hertz} und
\SI{1}{\mega\hertz} in Schritten von \SI{100}{\kilo\hertz} variiert.

Das Horizontalfeld wird so eingestellt, dass die Resonanzstellen zu sehen sind
und
die Resonanzstellen werden mittels Sweep-Spule direkt angefahren.

Es werden die Spulenspannungen der Horizontalfeldspule und Sweep-Spule
an den entsprechenden Ausgängen mit einem Multimeter gemessen.

Aus diesen Messungen lassen sich die Land\'e-Faktoren der beiden Rubidium
Isotope bestimmen, aus denen wiederum die jeweiligen Kernspins bestimmt werden
können.

\subsubsection{Bestimmung des Isotopenverhältnisses}%
\label{sub:bestimmung_des_isotopenverhaltnisses}

Das Isotopenverhältnis lässt sich auf zwei Arten bestimmen.
Aus dem aufgenommenen Foto (s.o.) und Daten des Dips können die
Amplituden der Resonanzstellen verglichen werden.

Die zweite Möglichkeit nutzt die transienten Effekte.
Hierfür wird die Frequenz wieder auf \SI{100}{\kilo\hertz} gestellt,
und die Resonanzstelle ausgewählt.
Es wird ein zweiter Funktionsgenerator an \enquote{Input TF Modulation}
angelegt, der eine Rechteckspannung mit \SI{5}{\hertz} ausgibt.
Der Ausgang wird zusätzlich auf den ersten Kanal des Oszilloskops gelegt, das
sich im $YT$\=/Modus befindet, und als Trigger verwendet.
Die Mittelungsfunktion des Oszilloskops wird aktiviert.

Es ist eine ansteigende Kurve zu sehen.
Dies ist die Auslenkung des Elektronenspins aus der Ruhelage.

Der Trigger wird auf die fallende Flanke gestellt und es wird horizontal
gezoomt.
Jetzt ist die oszillierende Relaxation in die Ruhelage sichtbar.
Die Periode der Oszillation wird bestimmt und gegen die Amplitude des
RF\=/Generators aufgetragen.
Die Amplitude wird zwischen \SI{1}{\volt} und \SI{10}{\volt} variiert.
Für beide Resonanzstellen wird eine Hyperbel gefittet.
Das Verhältnis der Steigungen der Hyperbeln entspricht dem Isotopenverhältnis.
