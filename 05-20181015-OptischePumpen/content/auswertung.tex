\section{Auswertung}%
\label{sec:auswertung}

Ziel der Auswertung ist es einerseits das Verhältniss der Radiumisotope
$^{85}$Rd und $^{87}$Rd zu bestimmen, sowie die Land\'efaktoren $g_\text{F}$ und
den Kernspins $I$.
Desweiteren wird das Magnetfeld einerseits über eine statische und dynamische
Methode bestommen.
\subsection{Statische Bestimmung des Erdmagnetfeldes}%
\label{sub:statische_bestimmung_des_erdmagnetfeldes}
\begin{wrapfigure}[15]{l}{0.45\textwidth}
	\centering
	\includegraphics[width=\linewidth]{picture/Transmission_Spek_cut.JPG}
	\caption{Transmissionslinien der Rubidiumisotope $^{85}$Rb und $^{87}$Rb bei
	einer Anregungsfrequenz von \SI{100}{\kilo\hertz}.}
	\label{fig:transmission}
\end{wrapfigure}
Das Magnetfeld der Helmholtzspulen wird anhand von Formel~\ref{eq:??} 
berechnet. 
Dazu wird das Magnetfeld der Sweepspule und der horizontalen addiert. 
Durch das Wechseln der Sweepspule vom contiunus auf den statischen Betrieb
können die Resonanzstellen im Spektrum abgefahren werden.
In Abbildung~\ref{fig:transmission} ist der Fotostrom gegen die Frequenz der
RF-Spule aufgetragen. 

\subsubsection{Statische Methode}%
\label{ssub:subsubsection_name}
Der große Peak entspricht grade dem Magnetfeld welches die Horizontalkomponente
des Erdmagnetfeldes $B_\text{Erd}$ kompensiert.
Die Vertikalkomponente wird optimiert, sodass der Peak möglichst schmal ist. 
Es ergibt sich ein Magnetfeld von 
\begin{eqnarray}
	B_\text{Erd,horiz} =& \SI{36.7}{\micro\tesla}, \\
	B_\text{Erd,verti} =& \SI{19.3}{\micro\tesla}. 
\end{eqnarray}

\subsubsection{Dynamische Methode}%
\label{ssub:dynamische_methode}
Zur Bestimmung des Magnetfeldes über die Dynamische Methode, werden die
Resonanzstellen der beiden Isotope vermessen und linear gefittet.
Die angelegten magnetischen Felder $B$ sind in Abhängigkeit der Frequenz in
Tabelle~\ref{tab:b_nu} aufgeführt.
Die Magnetfelder $B$ aufgetragen gegen die Anregungsfrequenzen $\nu$ sind in
Abbildung~\ref{fig:static_B} zu sehen.

\begin{table}[h]
	\centering
	\caption{Magnetfeld zur Einstellung der Resonanz der Rubidium Isotope.}
	\label{tab:b_nu}
	\sisetup{%
		round-mode=places,
		table-format=3.1,
		round-precision=1,
	}
		\input{build/b_nu.tex}
\end{table}

Die Messwerte werden linear gefittet und der y-Achsenabschnitt $b$ spricht genau den
Offset den das Erdmangnetfeld erzeugt.
\begin{figure}[h]
	\centering
	\includegraphics[width=0.8\linewidth]{build/static_B.pdf}
	\caption{Linearer Zusammenhang der Resonanzfrequenz und dem angelegten
	magnetischen Feld für beide Isotope.}
	\label{fig:static_B}
\end{figure}
Das zur Erdmangnetfeld kompensation benötigte Feld entspricht dem Wert $b$ aus
Tabelle~\ref{tab:lin_params}.

\begin{table}[h]
	\centering
	\caption{Lineare Fitparameter zur Bestimmung des Land\'efaktors und des
	Erdmagnetfeldes.}
	\label{tab:lin_params}
	\sisetup{%
		round-mode=figures,
		round-precision=3,
		table-format=1.2e3
	}
	\input{build/lin_params.tex}
\end{table}



\subsection{Land\'efaktoren und Kernmoment der Rubidium Isotope}%
\label{sub:landefaktoren_der_rubidium_isotope}
Die Land\'efaktoren werden entsprechend Gleichhung~\ref{eq:??} aus der Steigung
des Fittes in Abbildung~\ref{fig:??} bestimmt. 
Durch Umformen ergeben sich die in der Tabelle~\ref{tab:lande} aufgeführten
Land\'efaktoren $g_\text{exp}$ und die theoretischen Werte $g_\text{theo}$.


\begin{table}[h]
	\centering
	\caption{Die Abgeleiteten größen Land\'efaktor und Kernspin aus der Steigung
	$\nu$ gegen B.}
	\label{tab:lande}
	\begin{subtable}[t]{0.4\textwidth}
	\centering
	\caption{Land\'efaktoren}
	\label{tab:label}
	\input{build/lande.tex}
	\end{subtable}
	\begin{subtable}[t]{0.4\textwidth}
	\centering
	\caption{Kernspins}
	\label{tab:label}
	\input{build/kernspin.tex}
	\end{subtable}
\end{table}


\begin{table}[h]
\end{table}
\subsection{quadratischer Zeemanneffekt}%
\label{sub:quadratischer_zeemanneffekt}
Der quadratische Zeemanneffekt führt zu einer Energieabsenkung. 
Er liefert erst bei größeren Feldstärken sifnifikanten Beiträge.
Zur Abschätzung wird eine Feldstärke von B=$\SI{1}{\milli\tesla}$ 
und die Land\'efaktoren von Rubidium angenommen und die Energien entsprechend
Formel~\ref{eq:??} ausgerechnet.
\begin{equation}
	\frac{E_\text{squ} = \SI{1.04 +- 0.01e-29}{\joule}}{E_\text{lin} = 
		\SI{4.57 +- 0.02e-27}{\joule}} \approx 0.2 \%
\end{equation}
Er kann bei kleinen Feldern von um die \SI{100}{\micro\tesla} vernachlässigt werden.

\subsection{Bestimmung des Isotopenverhaeltniss}%
\label{sub:bestimmung_des_isotopenverhaeltniss}
Zur Bestimmung des Isotopenverhaeltniss wird einerseits eine statische Methode
wwelche auf dem Amplitudenverhältniss der Transmissionslinien beruht und
andererseits eine dynamische über die Rabi-Oszilationen gemessen.
\subsubsection{statisch}%
\label{ssub:statisch}
Das Amplitudenverhältniss kann aus der Ambildung~\ref{fig:??} abgelesen werde.
Dabei entspricht die erste Transmissionslinien dem $^{85}$Rb und die zweite dem
$^{85}$Rb Isotope.
Es wird ein Amplitudenverhältniss von 
\begin{equation}
	\frac{A_{85}}{A_{87}} = \frac{\SI{2.6}{\volt}}{\SI{1.4}{\volt}} = 1.86 
\end{equation}
gemessen, was von dem natürlichen um \SI{100}{\percent} abweicht.
Augenscheinlich wurde das Rubidiumgas mit dem $^{??}$Rb Isotope angereichert um
bessere optische Eigenschaften zu erreichen.


\subsubsection{dynamisch}%
\label{ssub:dynamisch}

Zur Bestimmung des Isotopenverhältniss werden bei einer RF-Frequenz von
\SI{100}{\kilo\hertz} die Spulenspannung variert und die Oszilationen gemessen.
Die entsprechende Schwingungsdauern bei den verschiedenen Spannungen sind in
Tabelle~\ref{tab:schw} aufgetragen.
Die Ozsillationen entstehen nachdem ein Zustand entladen wurde und anschliesend
ein .......
Ein Oszilloskopbild ist für beide Resonanzstellen in Abbildung~\ref{fig:osz} zu
sehen.
\begin{figure}[h]
	\centering
	\begin{subfigure}[c]{0.45\textwidth}
	\begin{center}
	\includegraphics[width=\textwidth]{./picture/Peak_1.JPG}
	\end{center}
	\caption{$^{85}$Rb}
	\label{fig:}
	\end{subfigure}
	\begin{subfigure}[c]{0.45\textwidth}
	\begin{center}
	\includegraphics[width=\textwidth]{./picture/Peak_2.JPG}
	\end{center}
	\caption{$^{87}$Rb}
	\label{fig:}
	\end{subfigure}
	\caption{Ladung und Entladung der Angeregten und Anschließende Oszillationen}
	\label{fig:osz}
\end{figure}
Die Periodendauern werden bestimmt indem mit der \texttt{scipy.signal.find\_peaks}
Funktion die Extrema der periodischen Funktionen vestimmt werden. 
Bei kleinen Spulenspannungen kommt es dabei zu Untergrundrauschen und der
Beeinflussung durch die hohen Magnetfelder nebenstehende Experimente (Abbildung
\ref{fig:85a}), sodass die Periode nicht mehr eindeutig bestimmt werden kann. 
Desweiteren wurd die zeitliche Auflösung bei zur Bestimmung der Periodendauer
der zweiten Transmissionslinien etwas zu grob gewählt, sodass die zum Teil sehr
groß ist.
Die bestimmten Periodendauern sind in Tabelle~\ref{tab:schw} aufgelistet.
\begin{table}[h]
	\centering
	\caption{caption}
	\label{tab:schw}
	\begin{subtable}[t]{0.4\textwidth}
	\centering
	\caption{$^{85}$Rb}
		\input{build/T1.tex}
	\end{subtable}
	\begin{subtable}[t]{0.4\textwidth}
	\centering
	\caption{$^{87}$Rb}
		\input{build/T2.tex}
	\end{subtable}
\end{table}
In der Abbildung~\ref{fig:periode} wurden die Feldstärken gegen die Spulenspannungen
aufgetragen und die Messdaten mit der Funktion~\ref{eq:fit} gefittet.
\begin{figure}[h]
	\centering
	\begin{subfigure}[c]{0.45\textwidth}
	\begin{center}
		\includegraphics[width=\textwidth]{build/firstPeak_9.png}
	\end{center}
	\caption{$^{85}$Rb \SI{10}{\volt}}
	\label{fig:}
	\end{subfigure}
	\begin{subfigure}[c]{0.45\textwidth}
	\begin{center}
		\includegraphics[width=\textwidth]{build/secondPeak_9.png}
	\end{center}
	\caption{$^{87}$Rb \SI{10}{\volt}}
	\label{fig:}
	\end{subfigure}

	\begin{subfigure}[c]{0.45\textwidth}
	\begin{center}
		\includegraphics[width=\textwidth]{picture/firstPeak_3.png}
	\end{center}
	\caption{$^{85}$Rb \SI{1}{\volt}}
	\label{fig:85a}
	\end{subfigure}
	\begin{subfigure}[c]{0.45\textwidth}
	\begin{center}
		\includegraphics[width=\textwidth]{build/secondPeak_3.png}
	\end{center}
	\caption{$^{87}$Rb \SI{3}{\volt}}
	\label{fig:}
	\end{subfigure}
	\caption{}
	\label{fig:periode}
\end{figure}
\begin{equation}
	\label{eq:fit}
	T(U) = a + \frac{b}{U + c}
\end{equation}
Die Ergebniss sind in Abbildung~\ref{fig:fitexp} dargestellt.
\begin{figure}[h]
	\centering
	\begin{subfigure}[c]{0.45\textwidth}
	\begin{center}
		\includegraphics[width=\textwidth]{build/firstPeak.pdf}
	\end{center}
	\caption{$^{85}$Rb}
	\label{fig:}
	\end{subfigure}
	\begin{subfigure}[c]{0.45\textwidth}
	\begin{center}
	\includegraphics[width=\textwidth]{build/secondPeak.pdf}
	\end{center}
	\caption{$^{87}$Rb}
	\label{fig:}
	\end{subfigure}
	\caption{}
	\label{fig:fitexp}
\end{figure}
Aus den Verhältnissen der Steigungsparameter $b$ der Fits wird das
Isotopenverhältniss von dem Rubidiumgas abgeschätzt. 
Es beträgt
\begin{equation}
	\frac{b_{85}}{b_{87}} =
	\frac{\SI{10.7+-9.0}{\second\per\volt}}{\SI{6.6+-3.5}{\second\per\volt}} =
	\num{1.6 +- 1.6} \ .
\end{equation}
Das natürliche Rubidium Verhältniss ist x:x.
Dies ist ein hinweis darauf, dass das verwendete Gemisch möglicherweise für
bessere optische Eigenschaften mit dem $^{85}$Rb Isotop angereichert ist. 
