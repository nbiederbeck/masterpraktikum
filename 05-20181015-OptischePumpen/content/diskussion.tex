\section{Diskussion}%
\label{sec:diskussion}
Die statisch bestimmte horizontal Komponenete des Erdmagnetfelde
$B_\text{Erd}=$\SI{19.7}{\micro\tesla} weicht um
\SI{1.5}{\percent} und die mittels X-Achsenabschnitt $B_\text{Erd}=$\SI{1.90}{\micro\tesla} bestimmt um
\SI{5}{\percent} von dem Literaturwert ab.
Dabei stellte das Ausrichten des Versuchsaufbaus anhand des vorliegenden
Kompasses ein Problem dar, aufgrund der Trägheit der Kompassnadel.
Die ermittelten Land\'efaktoren $g_\text{exp}$ weichen um \SI{1.4}{\percent}
und \SI{2.2}{\percent} vom Theoriewert ab.
Die für beide Isotope ermittelten Kernspins betragen $I_{87}=\num{1.53}$ und
$I_{85}=\num{2.57}$ und weisen eine Abweichung von
\SI{2.0}{\percent} für $^{85}$Rb und \SI{2.8}{\percent} für 
$^{87}$Rb auf \cite{rubidium}.
Das mittels Amplituden bestimmt Isotopenverhältniss beträgt \num{1.86}. 
Mittels der Larmorschwingungen wird ein Isotopenverhältniss von \num{1.6 +- 1.6}
bestimmt.
Die große Unsicherheit kommt aufgrund der Störeinflüsse der Umgebung und der
benachbarten Magnetfelder zustande.
Die benachbarten Magnetfelder verstimmen die eingestellte Resonanzstelle,
sodass die Periodendauer nicht mehr der Larmorperiode entspricht.
