% \begin{frame}[t]{Wettervorhersage}
% 	\begin{block}{Recurrent Neural Network \tiny\alert{(work in progress)}}
% 		\begin{itemize}
% 			\item Perfekt f\"ur Zeitreihenanalysen
% 		\end{itemize}
% 		\begin{align*}
% 			\left(\begin{matrix}
% 					\texttt{Zeit} \\
% 					\texttt{Temperaturen} \\
% 					\texttt{Luftdruck} \\
% 					\texttt{Luftfeuchtigkeit} \\
% 					\texttt{Wolkenklasse} \\
% 			\end{matrix}\right)
% 			\Longrightarrow \texttt{Wetter in der Zukunft}
% 		\end{align*}
% 		\begin{columns}[onlytextwidth]
% 			\begin{column}{0.48\textwidth}
% 				\begin{itemize}
% 					\item Lernt bei jedem neuen Datenpunkt neu
% 				\end{itemize}
% 			\end{column}
% 			\begin{column}{0.48\textwidth}
% 				\centering
% 				\includegraphics[height=0.2\textheight]{picture/neuralnet_brain.png}
% 			\end{column}
% 		\end{columns}
% 	\end{block}
% \end{frame}

\begin{frame}[c]{Wettervorhersage}
	\begin{block}{Recurrent Neural Network \tiny\alert{(work in progress)}}
				\begin{itemize}
					\item Perfekt f\"ur Zeitreihenanalysen
					\item $\vec{y}_\text{n+1} = f(\vec{y}_\text{n}, \vec{y}_\text{n-1},
						\ldots, \vec{y}_\text{n-m})$
						\begin{itemize}
							\item[mit] $\vec{y}_\text{n}$ = (\texttt{Temperatur}, \texttt{Luftdruck},
						\texttt{Luftfeuchtigkeit}, \texttt{Wolkenklasse}, \ldots)$^\intercal$
						\end{itemize}
					\item Lernt bei jedem neuen Datenpunkt neu
					\item \alert{ein neues Anwendungsbeispiel in der Lehre f\"ur
						RNNs}
				\end{itemize}
				\vspace{-0.7cm}
				\flushright{%
				\includegraphics[width=0.2\textwidth]{picture/neuralnet_brain.png}
			}
	\end{block}
\end{frame}
