\newpage
\section{Anhang}\label{sec:Anhang}

\begin{table}[ht]
		\centering
		\caption{Reflexe der Proben und die Gitterkonstanten mit ihrem
		systematischen Fehler.}
		\label{tab:wink}
		\begin{subtable}{\textwidth}
				\centering
				\caption{Probe 2}
				\begin{tabular}{c c c c c c}
						\toprule
						i & $\theta$ / deg & m & m$/\sin^2(\theta)$ & $\cos^2(\theta)$
						  & a / \AA \\ \midrule
						\input{build/reflexe1.txt}
						\bottomrule
				\end{tabular}
		\end{subtable}
		\begin{subtable}{\textwidth}
				\centering
				\caption{Salz 2}
				\begin{tabular}{c c c c c c}
						\toprule
						i & $\theta$ / deg & m & m$/\sin^2(\theta)$ & $\cos^2(\theta)$
						  & a / \AA \\ 
						\midrule
						\input{build/reflexe2.txt}
						\bottomrule
				\end{tabular}
		\end{subtable}
\end{table}

\begin{table}[ht]
		\centering
		\caption{Strukturfaktoren fuer verschieden kubische Gittertypen.}
		\label{tab:struct}
		\begin{tabular}{c c c c c c c c c c}
				\toprule
				hkl & m & sc & bcc & fcc & Diam. & Zinkbl. & Steins. & Caesiumc. & Fluorid \\ 
				\midrule
				\input{build/structf.txt}	
				\bottomrule
		\end{tabular}
\end{table}

