\section{Diskussion}\label{sec:Diskussion}
Der Fehler der Winkelauflösung  dominiert die Unsicherheit der bestimmten
Gitterkonstanten. 
Das Ablesen der beiden Emissionslinien ist kaum möglich,
sodass in der Regel nur versucht wurde, den deutlicheren der beiden Reflexe zu bestimmen.
Die Emissionswahrscheinlichkeit der dickeren Linie ist desweiteren auch
ungefähr eine Größenordnung wahrscheinlicher. 
Es fällt auf, dass die Reflexe bei kleinen Winkeln nicht besonders gut zu den
Gittern passen, da kleine Winkel aufgrund der Probenbreite überschätzt werden.
Bei der linearen Ausgleichsgeraden ist die Unsicherheit des Fits bei Probe 2 
wesentlich geringer als bei Salz 2. 
Mögliche Ursachen dafür sind nicht bekannt. 
Anhand der Formfaktoren lassen sich relative Änderungen der Intensitätsverteilungen ausrechnen,
anhand derer auf die Probe geschlossen wird.
Beide Proben werden dem \textit{fcc}-Gittertyp zugeordnet. 
Probe 2 weist einen Netzebenenabstand von \SI{3.64 +- 0.01}{\angstrom}
und Salz 2 von \SI{4.70 +- 0.12}{\angstrom} auf.
