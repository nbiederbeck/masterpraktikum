\section{Diskussion}\label{sec:Diskussion}
Der Fehler der Winkelauflösung  dominiert die Unsicherheit der bestimmten
Gitterkonstante. 
Das Ablesen der beiden Emissionslinien ist kaum möglich. 
Sodass in der Regel nur versucht wurde die deutlicheren der beiden Reflexe zu 
bestimmen.
Die Emissionswahrscheinlichkeit der dickeren Linie ist desweiteren auch
ungefähr eine Größenordnung wahrscheinlicher. 
Es fällt auf das die Reflexe bei kleinen Winkeln nicht besonders gut zu den
Gittern passen, da kleine Winkel aufgrund der Probenbreite überschätzt werden.
Bei der lineraren Ausgleichsgraden ist die Unsicherheit des Fits bei Probe 2 
wesentlich geringer als bei Salz 2. 
Mögliche Ursachen dafür sind nicht bekannt. 
Anhand der Formfaktoren lassen sich relative Änderungen der Intensitätsverteilungen ausrechnen
anhand derer auf die Probe geschlossen wird.
Beide Proben werden dem fcc Gittertyp zugeordnet. 
Die zweite Probe weist einen Netzebenenabstand von \SI{3.64 +- 0.01}{\angstrom}
und das zweite Salz von \SI{4.70 +- 0.12}{\angstrom} auf.
