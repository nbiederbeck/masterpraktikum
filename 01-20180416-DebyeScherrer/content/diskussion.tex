\section{Diskussion}\label{sec:Diskussion}
Der Fehler der Winkelauflösung  dominiert die Unsicherheit der bestimmten
Gitterkonstanten. 
Das Ablesen der beiden Emissionslinien ist kaum möglich,
sodass in der Regel nur versucht wurde, den deutlicheren der beiden Reflexe zu bestimmen.
Die Emissionswahrscheinlichkeit der dickeren Linie ist desweiteren auch
doppelt so wahrscheinlich. 
Es fällt auf, dass die Gitter bei kleinen Winkeln nicht besonders gut zu den
Reflexen passen, da kleine Winkel aufgrund der Probenbreite überschätzt werden.
Alle Messwerte zur Bestimmung der vom systematischen Fehler befreiten
Gitterkonstante liegen im Rahmen der Unsicherheit auf dem Fit.
Mittels der Proportionalität vom Strukturfaktor und der Intensität wird
versucht die relativen Intensitätsänderungen bei der Bestimmung der
Gitterstruktur zu berücksichtigen.
Probe 2 wird dem \textit{fcc}-Gittertyp und Salz 2 der Fluoridstruktur zugeordnet. 
Probe 2 weist eine Gitterkonstante von \SI{3.64 +- 0.01}{\angstrom}
und Salz 2 von \SI{5.48 +- 0.01}{\angstrom} auf.
Probe 2 scheint Kupfer zu sein, was einer Abweichung von \SI{0.8}{\percent} vom Literaturwert
\cite{kupfer} entspricht.
Am wahrscheinlichsten ist Salz 2 Calciumfluorid. Dies würde einer Abweichung
der Gitterkonstanten vom Literaturwert \cite{CaF2} von \SI{0.4}{\percent} entsprechen.
