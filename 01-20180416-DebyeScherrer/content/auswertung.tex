\section{Auswertung}\label{sec:Auswertung}

Zur Bestimmung der Gitterstrukturen und der Netzebenenabstände werden die
Radien der Fotostreifen in Streuwinkel umgerechnet.
Der Umfang der Box der Kamera beträgt genau \SI{360}{\milli\meter}.
\SI{1}{\milli\meter} entspricht genau dem Beugungswinkel $\theta$ von
\SI{1}{\degree}.
Da das Einscannen der aufgenommenen Negative aufgrund schwieriger
Belichtungsverhältnisse nicht möglich ist, werden die Winkel der aufgenommenen
Reflexe mittels einer Schieblehre ausgemessen.
Die Winkelverteilung der Reflexe sind in Tabelle~\ref{tab:wink} aufgetragen.
Dabei wird für jeden Messwert ein Messfehler von \SI{1}{\milli\meter} und
zusätzlich der Fehler der Winkelaufspaltung der zwei Emissionslinien
entsprechend Formel~\eqref{eq:lamtheta} angenommen.
Dieser Fehler wird maximal für große Winkel, beträgt $< \SI{1}{\degree}$ und ist mit dem
Ablesefehler summiert zusammen als Unsicherheit in Tabelle~\ref{tab:wink}
aufgeführt.
In der Auswertung wird im Weiteren mit der gemittelten Wellenlänge
$\overline{\lambda}_{K\alpha}$ gerechnet, da die Aufspaltung der beiden Emissionslinien 
mit den vorhandenen Instrumenten nicht weiter vermessen werden kann.

\begin{table}[ht]
		\centering
		\caption{Reflexe der Proben und die Gitterkonstanten mit ihrem
		systematischen Fehler.}
		\label{tab:wink}
		\begin{subtable}{\textwidth}
				\centering
				\caption{Probe 2}
				\begin{tabular}{c c c c c c}
						\toprule
						i & $\theta$ / deg & m & m$/\sin^2(\theta)$ & $\cos^2(\theta)$
						  & a / \AA \\ \midrule
						\input{build/reflexe1.txt}
						\bottomrule
				\end{tabular}
		\end{subtable}
		\begin{subtable}{\textwidth}
				\centering
				\caption{Salz 2}
				\begin{tabular}{c c c c c c}
						\toprule
						i & $\theta$ / deg & m & m$/\sin^2(\theta)$ & $\cos^2(\theta)$
						  & a / \AA \\ 
						\midrule
						\input{build/reflexe2.txt}
						\bottomrule
				\end{tabular}
		\end{subtable}
\end{table}

Entsprechend Formel~\eqref{eq:const} lassen sich Strukturfaktoren simulieren.
Diese sind in Tabelle~\ref{tab:struct}, bis zu Reflexen der 4. Ordnung, aufgetragen.
Dazu werden systematisch die Atomformfaktoren der in der Anleitung
\cite{anleitung} aufgeführten Ionenkombinationen berechnet und im folgenden mit
den Messwerten verglichen. 
In Tabelle~\ref{tab:struct} sind die zusammengesetzten Strukturfaktoren unter der Annahme der
Atomfaktoren für Calcium und Fluor aufgetragen und fuer die nicht
zusammgesetzten mit einem Atomformfaktor von eins.

Für die Zuordnung von den Strukturfaktoren zu den Reflexen wird die Annahme
getroffen,
dass nur die Reflexe wo der Strukturfaktor groß ist eine auswertbare Schwärzung
des Negatives verursachen. 
Die Reflexe werden dem Gitter zugeordnet bei dem der Quotienten aus der Ordnung 
$m$ und $\sin^2(\theta)$ (siehe Tabelle~\ref{tab:wink}) möglichst konstant ist.
Des Weiteren wird versucht, die relative Stärke der Reflexe bei dem Vergleich mit den
Strukturfaktoren zu berücksichtigen.

\begin{figure}[ht]
		\centering
		\begin{subfigure}{0.49\textwidth}
				\centering
				\includegraphics[width=\textwidth]{build/lin_fit1.pdf}
				\caption{Probe 2.}
				\label{fig:prb1}
		\end{subfigure}
		\begin{subfigure}{0.49\textwidth}
				\centering
				\includegraphics[width=\textwidth]{build/lin_fit2.pdf}
				\caption{Salz 2.}
				\label{fig:prb1}
		\end{subfigure}
		\caption{Bestimmung der exakten Gitterkonstanten aus der Abhängigkeit
		des	Messfehlers und den Gittervektoren.}
\end{figure}

Probe 2 scheint am besten zur Struktur des fcc-Gitters zu passen.
Dabei wurde allerdings der erste Reflex von Probe 2 verworfen, bei dem es sich
nach Vergleich mit den verschiedenen Gitterstrukturen wohl um eine Verunreinigung
handelt.
Bei Salz 2 scheinen die Intensitätsmaxima verglichen mit Probe 2 stärker zu
varieren. 
Es lässt sich sowohl der Struktur der Zinkblende als auch der Fluoridstruktur
zuordenen. 
Jeweils charakteristisch für beide Gittertypen ist, dass der zweite Reflex
stärker als der erste ausgeprägt ist.
Nach Formel~\eqref{eq:gittervektor} lassen sich die
Gitterkonstanten $a$ bestimmen, die einen Versatz aufgrund des
Probendurchmessers, als auch die Verschiebung der Dreh- zur Kamera-Achse
aufweisen, welcher $\propto \cos^2(\theta)$ ist.

Aus einem linearen Fit der Form $f(x) = a \cdot x + b$ können die vom
systematischen Fehler befreiten  
Gitterkonstanten bestimmt werden, da der Messfehler $\propto \cos^2(\theta)$ ist
(vergleiche Kapitel~\ref{sec:Messmethode}).
Dabei entspricht der $y$-Achsenabschnitt der Gitterkonstante $a$.
Die vom systematischen Fehler befreiten Gitterkonstanten sind in Tabelle~\ref{tab:gitt} aufgeführt.

\begin{table}[ht]
		\centering
		\caption{Linerarer Fit mit $y$-Achsenabschnitt $a$ und Steigung $b$ zur
				Bestimmung der exakten Gitterkonstante.}
		\label{tab:gitt}
		\begin{tabular}{l c c}
				\toprule
        & $a$ / \AA & $b$ / \AA \\
				\midrule
				Probe 2: 	& \input{build/params1.txt}
				Salz 2: 	& \input{build/params2.txt}
				\bottomrule
		\end{tabular}
\end{table}

Mittels Gitterstruktur und Gitterkonstante wird versucht die Bestandteile der
Probe zu erschließen. 
Probe 2 scheint mit dem Abgleich mit Quelle \cite{kupfer} möglicherweise Kupfer
zu sein. 
Wie durch die Atomformfaktoren bereits ermittelt scheint Salz 2 Calciumfluorid
zu sein. 
