\section{Auswertung}\label{sec:Auswertung}

Zur Bestimmung der Gitterstrukturen und der Netzebenenabstaende werden die
Radien der Fotostreifen in Streuwinkel umgerechnet. 
Der Umfang der Box der Kamera betraegt genau \SI{360}{\milli\meter}.
\SI{1}{\milli\meter} entspricht genau dem Beugungswinkel $\theta$ von 
\SI{1}{\degree}. 
Da das einscannen der aufgenommenen Negative aufgrund schwieriger
Belichtungsverhaeltnisse nicht moeglich ist, werden die Winkel der aufgenommenen
Reflexe mittels einer Schieblehre ausgemessen. 
Die Winkelverteilung der Reflexe sind in Tabelle \ref{tab:wink} aufgetragen. 
Entsprechend Formel \ref{eq:??}, lassen sich Strukturfaktoren simmulieren. 
Diese sind in Tabelle \ref{tab:struct} aufgetragen. 
Da nicht bekannt ist, nach welcher Struktur gesucht wird, wird bei 
zusammengesetzten Strukturen mit unterschiedlichen Atomen die Atomformfaktoren 
1 und $\sfrac{5}{6}$ gewaehlt und sonst 1. 

Die Reflexe werden dem Gitter zugeordnet, bei welchem die auftretenden Reflexe fuer
die nicht verschwindenden Strukturfaktoren der verschiedenen Gittertypen der 
Quotiente aus der Ordnung $m$ und $\sin^2(\theta)$ (siehe Tabelle \ref{tab:wink})
moeglichst Konstant ist.
Desweiteren wird versucht die relative Staerke der Reflexe bei dem Vergleich mit den
Strukturfaktoren zu beruecksichtigen. 

Sowohl Probe 2 als auch Salz 2 scheinen am besten zur Struktur des fcc-Gitter zu
passen. 
Dabei wurde allerdings der erste Reflex von Probe 2 verworfen, bei dem es sich 
nach Vergleich mit den verschiedenen Gitterstrukturen wohl um eine Verunreinigung
handelt. 
Unter der Annahme lassen sich nach Formel \ref{eq:gittervektor} die 
Gitterkonstanten $a$ bestimmen die einen Versatz aufgrund des 
Probendurchmessers, als auch die Verschiebung der Dreh- zur Kamera-Achse 
aufweisen, welcher $\propto \cos^2(\theta)$ ist.
\begin{figure}[ht]
		\centering
		\caption{Bestimmung der exakten Gitterkonstanten aus der Abhaengigkeit 
		des	Messfehlers und den naiven Gittervektoren.}
		\begin{subfigure}{0.49\textwidth}
				\centering
				\includegraphics[width=\textwidth]{build/lin_fit1.pdf}
				\caption{Probe 1}
				\label{fig:prb1}
		\end{subfigure}
		\begin{subfigure}{0.49\textwidth}
				\centering
				\includegraphics[width=\textwidth]{build/lin_fit2.pdf}
				\caption{Salz 1}
				\label{fig:prb1}
		\end{subfigure}
\end{figure}

Aus einem linearen Fit der Form $f(x) = a \cdot x + b$ werden die ungebiasten
Gitterkonstanten bestimmt. 
Dabei entspricht der y-Achsen Abschnitt die Gitterkonstante $a$. 
Die ungebiasten Gitterkonstanten sind in Tabelle \ref{tab:gitt} aufgefuehrt.

\begin{table}[ht]
		\centering
		\caption{linerarer Fit mit y-Achsenabschnitt a und Steigung b zur
				Bestimmung der exakten Gitterkonstante.}
		\label{tab:gitt}
		\begin{tabular}{l c c}
				\toprule
				& a / 10 nm & b / 10nm \\
				\midrule
				Probe 2: 	& \input{build/params1.txt}
				Salz 2: 	& \input{build/params2.txt}
				\bottomrule
		\end{tabular}
\end{table}
