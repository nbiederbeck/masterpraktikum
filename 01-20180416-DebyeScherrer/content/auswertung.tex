\section{Auswertung}\label{sec:Auswertung}

Die Kamera besitzt einen Radius von \SI{57.3}{\milli\meter}.
Somit ergibt sich fuer ein Umfang von \SI{}{\milli\meter}

\begin{table}[h]
		\centering
		\caption{Strukturfaktoren fuer verschieden kubische Gittertypen}
		\label{tab:label}
		\begin{tabular}{c c c c c c c c c c}
				\toprule
				hkl & m & sc & bcc & fcc & Diam. & Zinkbl. & Steins. & Caesiumc. & Fluorid \\ 
				\midrule
				\input{build/structf.txt}	
				\bottomrule
		\end{tabular}
\end{table}

\begin{table}[h]
		\centering
		\caption{Reflexe der Proben}
		\label{tab:label}
		\begin{subtable}{0.9\textwidth}
				\centering
				\caption{Probe 2}
				\begin{tabular}{c c c c c c}
						\toprule
						i & $\theta$ & m & m$/\sin^2(\theta)$ & $\cos^2(\theta)$ & a / 10nm \\ 
						\midrule
						\input{build/reflexe1.txt}
						\bottomrule
				\end{tabular}
		\end{subtable}
		\begin{subtable}{0.9\textwidth}
				\centering
				\caption{Salz 2}
				\begin{tabular}{c c c c c c}
						\toprule
						i & $\theta$ & m & m$/\sin^2(\theta)$ & $\cos^2(\theta)$ & a / 10nm \\ 
						\midrule
						\input{build/reflexe2.txt}
						\bottomrule
				\end{tabular}
		\end{subtable}
\end{table}

\begin{figure}[h]
		\centering
		\begin{subfigure}{0.49\textwidth}
				\centering
				\includegraphics[width=\textwidth]{build/test1.pdf}
				\caption{probe 1}
				\label{fig:prb1}
		\end{subfigure}
		\begin{subfigure}{0.49\textwidth}
				\centering
				\includegraphics[width=\textwidth]{build/test2.pdf}
				\caption{probe 1}
				\label{fig:prb1}
		\end{subfigure}
\end{figure}
