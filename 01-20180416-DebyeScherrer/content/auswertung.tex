\section{Auswertung}\label{sec:Auswertung}

Zur Bestimmung der Kitterstrukturen und der Netzebenenabstaende werden die
Radien der Fotostreifen in Streuwinkel umgerechnet. Der Umfang der Box der 
Kamera betraegt genau \SI{360}{\milli\meter}. \SI{1}{\milli\meter}
entspricht genau dem Beugungswinkel $\theta$ von \SI{1}{\degree}. Da das
einsacannen der aufgenommenen Negative aufgrund schwieriger
Belichtungsverhaeltnisse nicht moeglich ist, werden die Winkel der aufgenommenen
Reflexe mittels einer Schieblehre ausgemessen. Die Winkelverteilung der Reflexe
sind in Tabelle \ref{tab:??} aufgetragen. Die Strukturfaktoren der Gitter sind 
in Tabelle \ref{tab:??} aufgetragen. Bei zusammengestzten Strukturen mit 
unterschiedlichen Atomen wurden die Atomformfaktoren 1 und $\sfrac{5}{6}$ 
gewaehlt und sonst 1. 

Die Reflexe werden dem Gitter zugeordnet bei dem die auftretenden Reflexe fuer
die nicht verschwindenden Strukturfaktoren der verschiedenen Gittertypen der 
Quotiente aus der Ordnung $m$ und $\sin^2(\theta)$ (siehe Tabelle \ref{tab:??}) moeglichst Konstant ist.
Desweiteren wird versucht die Staerke der Reflexe bei dem Vergleich mit den
Strukturfaktoren zu beruecksichtigen. 

Sowohl Probe 2 als auch Salz 2 scheinen am besten zur Struktur des fcc-Gitter zu
passen. Unter der Annahme lassen sich nach Formel \ref{eq:gittervektor}
die Gitterkonstanten $a$ bestimmen die einen Versatz aufgrund des
Probendurchmessers, als auch die Verschiebung der Dreh- zur Kamera-Achse 
aufweisen, welcher $\propto \cos^2(\theta)$ ist.
\begin{figure}[ht]
		\centering
		\begin{subfigure}{0.49\textwidth}
				\centering
				\includegraphics[width=\textwidth]{build/lin_fit1.pdf}
				\caption{probe 1}
				\label{fig:prb1}
		\end{subfigure}
		\begin{subfigure}{0.49\textwidth}
				\centering
				\includegraphics[width=\textwidth]{build/lin_fit2.pdf}
				\caption{probe 1}
				\label{fig:prb1}
		\end{subfigure}
\end{figure}

Aus einem linearen fit der Form $f(x) = a \cdot x + b$ werden die ungebiasten
Gitterkonstanten bestimmt. Dabei entspricht der y-Achsen Abschnitt die
Gitterkonstante $a$. Die ungebiasten Gitterkonstanten sind in Tabelle
\ref{tab:gitt} aufgefuehrt.

\begin{table}[ht]
		\centering
		\caption{linerarer Fit mit y-Achsenabschnitt b und Steigung a}
		\label{tab:gitt}
		\begin{tabular}{c c}
				\toprule
				a / 10 nm & b / 10nm \\
				\midrule
				\input{build/params1.txt}
				\input{build/params2.txt}
				\bottomrule
		\end{tabular}
\end{table}

\section{Anhang}\label{sec:Anhang}

\begin{table}[ht]
		\centering
		\caption{Reflexe der Proben}
		\label{tab:label}
		\begin{subtable}{0.9\textwidth}
				\centering
				\caption{Probe 2}
				\begin{tabular}{c c c c c c}
						\toprule
						i & $\theta$ & m & m$/\sin^2(\theta)$ & $\cos^2(\theta)$ & a / 10nm \\ \midrule
						\input{build/reflexe1.txt}
						\bottomrule
				\end{tabular}
		\end{subtable}
		\begin{subtable}{0.9\textwidth}
				\centering
				\caption{Salz 2}
				\begin{tabular}{c c c c c c}
						\toprule
						i & $\theta$ & m & m$/\sin^2(\theta)$ & $\cos^2(\theta)$ & a / 10nm \\ 
						\midrule
						\input{build/reflexe2.txt}
						\bottomrule
				\end{tabular}
		\end{subtable}
\end{table}

\begin{table}[ht]
		\centering
		\caption{Strukturfaktoren fuer verschieden kubische Gittertypen}
		\label{tab:label}
		\begin{tabular}{c c c c c c c c c c}
				\toprule
				hkl & m & sc & bcc & fcc & Diam. & Zinkbl. & Steins. & Caesiumc. & Fluorid \\ 
				\midrule
				\input{build/structf.txt}	
				\bottomrule
		\end{tabular}
\end{table}

