\section{Auswertung}\label{sec:Auswertung}

Zur Bestimmung der Gitterstrukturen und der Netzebenenabstände werden die
Radien der Fotostreifen in Streuwinkel umgerechnet.
Der Umfang der Box der Kamera beträgt genau \SI{360}{\milli\meter}.
\SI{1}{\milli\meter} entspricht genau dem Beugungswinkel $\theta$ von
\SI{1}{\degree}.
Da das einscannen der aufgenommenen Negative aufgrund schwieriger
Belichtungsverhältnisse nicht möglich ist, werden die Winkel der aufgenommenen
Reflexe mittels einer Schieblehre ausgemessen.
Die Winkelverteilung der Reflexe sind in Tabelle~\ref{tab:wink} aufgetragen.
Dabei wird für jeden Messwert ein Messfehler von \SI{1}{\milli\meter} und
zusätzlich der Fehler der Winkelaufspaltung der zwei Emissionslinien
entsprechend Formel~\ref{eq:lamtheta} angenommen.
Dieser Fehler wird maximal für große Winkel, ist \SI{<1}{\degree} und mit dem
Ablesefehler summiert zusammen als Unsicherheit in Tabelle \ref{tab:wink}
aufgeführt.
In der Auswertung wird im weiteren mit der gemittelten Wellenlänge
$\lambda_{K\alpha}$ gerechnet, da die Aufspaltung der beiden Emissionslinien 
mit den vorhandenen Instrumenten nicht weiter vermessen werden kann.

Entsprechend Formel~\eqref{eq:const} lassen sich Strukturfaktoren simmulieren.
Diese sind in Tabelle~\ref{tab:struct}, bis zu Reflexen der 4 Ordnung, aufgetragen.
Da nicht bekannt ist, nach welcher Struktur gesucht wird, wird bei
zusammengesetzten Strukturen mit unterschiedlichen Atomen die Atomformfaktoren
1 und $\sfrac{5}{6}$ und sonst 1 gewählt.

Die Reflexe werden dem Gitter zugeordnet, bei welchem die auftretenden Reflexe für
die nicht verschwindenden Strukturfaktoren der verschiedenen Gittertypen der
Quotiente aus der Ordnung $m$ und $\sin^2(\theta)$ (siehe Tabelle~\ref{tab:wink})
möglichst Konstant ist.
Desweiteren wird versucht die relative Stärke der Reflexe bei dem Vergleich mit den
Strukturfaktoren zu berücksichtigen.

\begin{figure}[ht]
		\centering
		\begin{subfigure}{0.49\textwidth}
				\centering
				\includegraphics[width=\textwidth]{build/lin_fit1.pdf}
				\caption{Probe 2}
				\label{fig:prb1}
		\end{subfigure}
		\begin{subfigure}{0.49\textwidth}
				\centering
				\includegraphics[width=\textwidth]{build/lin_fit2.pdf}
				\caption{Salz 2}
				\label{fig:prb1}
		\end{subfigure}
		\caption{Bestimmung der exakten Gitterkonstanten aus der Abhängigkeit
		des	Messfehlers und den naiven Gittervektoren.}
\end{figure}

Sowohl Probe 2 als auch Salz 2 scheinen am besten zur Struktur des fcc-Gitter zu
passen.
Dabei wurde allerdings der erste Reflex von Probe 2 verworfen, bei dem es sich
nach Vergleich mit den verschiedenen Gitterstrukturen wohl um eine Verunreinigung
handelt.
Unter der Annahme lassen sich nach Formel~\eqref{eq:gittervektor} die
Gitterkonstanten $a$ bestimmen die einen Versatz aufgrund des
Probendurchmessers, als auch die Verschiebung der Dreh- zur Kamera-Achse
aufweisen, welcher $\propto \cos^2(\theta)$ ist.

Aus einem linearen Fit der Form $f(x) = a \cdot x + b$ können die ungebiasten
Gitterkonstanten bestimmt, werden da der Messfehler $\propto \cos^2(\theta)$ ist
(Vergleiche Kapitel~\ref{sec:Messmethode}).
Dabei entspricht der y-Achsen Abschnitt die Gitterkonstante $a$.
Die ungebiasten Gitterkonstanten sind in Tabelle~\ref{tab:gitt} aufgeführt.

\begin{table}[ht]
		\centering
		\caption{Linerarer Fit mit $y$-Achsenabschnitt $a$ und Steigung $b$ zur
				Bestimmung der exakten Gitterkonstante.}
		\label{tab:gitt}
		\begin{tabular}{l c c}
				\toprule
        & a / \AA & b / \AA \\
				\midrule
				Probe 2: 	& \input{build/params1.txt}
				Salz 2: 	& \input{build/params2.txt}
				\bottomrule
		\end{tabular}
\end{table}
