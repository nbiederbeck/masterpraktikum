\section{Theorie}\label{sec:Theorie}
\nocite{anleitung}

Der Großteil der Materie, die sich im festen Aggregatzustand befindet,
ist kristallin aufgebaut.
Dies bedeutet, dass seine Atome räumlich periodisch angeordnet sind,
aufgrund von zwischenatomaren Kräfte.
Da viele Festkörper, sogenannte Polykristallite, aus vielen Kristalliten bestehen,
muss über die in ihnen enthaltenen Kristalle mitteln, und damit beschränkt sich die
Untersuchung von Kristalleigenschaften auf Einkristalle oder einzelne Kristalle
von Polykristallen.

Um die periodische Struktur von Kristallen aufzulösen, wird eine Sonde in der
Größenordnung der Atomabstände benötigt.
Hierfür eignen sich Elektronen, langsame Neutronen und insbesondere Röntgenstrahlen.

In diesem Versuch wird die Debye-Scherre-Methode zur Kristallstrukturuntersuchung beschrieben,
die die Beugung von Röntgenstrahlung am Kristallgitter nutzt.

\subsection{Kristallstrukturen}%
\label{sub:kristallstrukturen}
% Basis
% Basisvektor
% Punktgitter
% Translation
% Elementarzelle
% Symmetrieeigenschaften
% Bravais-Gitter
Die Beschreibung der räumlichen Periodizität basiert auf einem Punktgitter.
Jeder Gitterpunkt kann aus einem oder mehreren Atomen bestehen,
dies wird als Basis bezeichnet.
In Abbildung~\ref{fig:punktgitter} sind ein Punktgitter, eine Basis und ein Basisgitter
dargestellt.
\begin{figure}
  \centering
  \includegraphics[width=0.8\linewidth]{build/punktgitter.pdf}
  \caption{Von links: Darstellung von Punktgitter, Basis und Basisgitter.\cite{anleitung}}%
  \label{fig:punktgitter}
\end{figure}

Die Vektoren $\vec{a}, \vec{b}, \vec{c}$, die bei einer Translation
\begin{align}
  \vec{T} &= x\vec{a} + y\vec{b} + z\vec{c} \\
  \intertext{mit}
  \left\{x, y, z\right\} &\in \mathbb{N}_0
\end{align}
das Gitter in sich selbst abbilden, heißen Basisvektoren.
Ihre Wahl ist nahezu beliebig, jedoch werden aus Symmetriegründen jene gewählt,
deren Länge dem Abstand der nächsten Nachbarn der Atome gleicht.

So existieren 14 verschiedene Gitter, die Bravais-Gitter genannt werden
und in 7 Gittertypen eingeteilt werden.
Einige davon werden in Kapitel~\ref{sub:kubische_kristallstrukturen} beschrieben.

Das von den Basisvektoren aufgespannte Parallelepided mit Volumen
\begin{equation}
  V = \vec{a} \cdot \left(\vec{b} \times \vec{c}\right)
\end{equation}
heißt Einheitszelle.


\subsection{Kubische Kristallstrukturen}%
\label{sub:kubische_kristallstrukturen}
% kubisch-raumzentriert
% kubisch-flächenzentriert
% Zinkblende-Struktur
% Steinsalz-Struktur
% Cäsiumchlorid-Struktur
% Fluorit-Struktur
% hexagonale Struktur
Die einfachsten Kristallstrukturen haben kubische Anordnung.
Dies bedeutet, dass ihre Basisvektoren jeweils senkrecht aufeinander stehen.
Hierbei wird zwischen ein- und mehratomigen Basen unterschieden.
Das einfachste (\textit{primitive}) kubische Gitter hat auf je ein Atom auf den Ecken eines Würfels.
Da sich das Atom auf der Ecke genau acht Würfelgitter teilt,
besteht die Elementarzelle also aus einem Atom.

Ein Gitter heißt \textit{kubisch-raumzentriert}, wenn zusätzlich zum primitiven Gitter
ein Atom in der Mitte des Würfels zu finden ist.
Die Elementarzelle besteht hier aus zwei Atomen.
Die Atome liegen, geschrieben in Koordinaten der Basisvektoren, bei
\begin{align*}
  % \label{eq:basisvektoren_kubisch-raumzentriert}
  a_1 = {\left(0, 0, 0\right)}^\top, \hspace{2em} a_2 = {\left(\sfrac{1}{2}, \sfrac{1}{2}, \sfrac{1}{2}\right)}^\top.
\end{align*}

Ein weiteres einatomiges Gitter ist das \textit{kubisch-flächenzentrierte} Gitter.
Hier sind die Atome sowohl auf den Ecken, als auch in der Mitte jeder Seitenfläche.
Die Elementarzelle besteht somit aus $8 \cdot \sfrac{1}{8} + 6 \cdot \sfrac{1}{2} = 4$ Atomen.
Die Atome liegen an den Stellen
\begin{align*}
  % \label{eq:basisvektoren_kubisch-flächenzentriert}
  a_1 = {\left(0, 0, 0\right)}^\top, \hspace{2em}
  a_2 = {\left(0, \sfrac{1}{2}, \sfrac{1}{2}\right)}^\top, \hspace{2em}
  a_3 = {\left(\sfrac{1}{2}, 0, \sfrac{1}{2}\right)}^\top, \hspace{2em}
  a_4 = {\left(\sfrac{1}{2}, \sfrac{1}{2}, 0\right)}^\top.
\end{align*}

Die \textit{Diamantstruktur} besteht aus zwei kubisch-flächenzentrierten Gittern,
die um eine Viertelraumdiagonale gegeneinander verschoben sind.
Die Elementarzelle besteht aus acht Atomen an der Orten
\begin{align*}
  a_1 = {\left(0, 0, 0\right)}^\top, \hspace{1em}
  a_2 = {\left(0, \sfrac{1}{2}, \sfrac{1}{2}\right)}^\top, \hspace{1em}
  a_3 = {\left(\sfrac{1}{2}, 0, \sfrac{1}{2}\right)}^\top, \hspace{1em}
  a_4 = {\left(\sfrac{1}{2}, \sfrac{1}{2}, 0\right)}^\top, \\
  a_5 = {\left(\sfrac{1}{4}, \sfrac{1}{4}, \sfrac{1}{4}\right)}^\top, \hspace{1em}
  a_6 = {\left(\sfrac{1}{4}, \sfrac{3}{4}, \sfrac{3}{4}\right)}^\top, \hspace{1em}
  a_7 = {\left(\sfrac{3}{4}, \sfrac{1}{4}, \sfrac{3}{4}\right)}^\top, \hspace{1em}
  a_8 = {\left(\sfrac{3}{4}, \sfrac{3}{4}, \sfrac{1}{4}\right)}^\top.
\end{align*}

Die \textit{Zinkblende-Struktur} besteht aus einer zweiatomigen Basis,
deren Atome jeweils ein kubisch-flächenzentriertes Gitter
der Diamantstruktur belegen.

Die \textit{Steinsalz-Struktur} ist ganz ähnlich wie die Zinkblende-Struktur,
jedoch sind die einzelnen Gitter um eine halbe Raumdiagonale gegeneiander verschoben.

Die häufige \textit{Cäsiumchlorid-Struktur} entspricht der Steinsalz-Struktur,
jedoch mit zwei kubisch-primitiven Gittern.

Die \textit{Fluorit-Struktur} besteht aus drei kubisch-flächenzentrierten Gittern,
die gegeneinander um $\sfrac{1}{4}$- und $\sfrac{3}{4}$-Raumdiagonale verschoben sind.
Die Elementarzelle besteht aus einer zweiatomigen Basis des Typs $AB_2$.

Eine weitere häufige, jedoch nicht kubische, ist die \textit{hexagonale Struktur}.
Hier beschreibt die Form der Elementarzelle ein rechtwinkliges Prisma,
und die Atome sind in der dichtesten Kugelpackung angeordnet.

\subsection{Netzebenen}
% Miller Indizes
% Netzebenenschar
% Netzebenenabstand
In sogenannten Netzebenen liegen die Schwerpunkte der Atome.
Parallele Netzebenen bilden eine Ebenenschar.
Die Gesamtheit einer Ebenenschar wird durch die drei
\textit{Millerschen Indizes} $\left(hkl\right)$ beschrieben.
Diese werden bestimmt, indem die reziproken Achsenabschnitte
der dem Koordinatenursprung nächsten Netzebene in Bruchteilen der
Basisvektorenlänge angegeben wird.
Weiterhin sollten die Indizes aus Rechengründen auf ganze Zahlen gebracht werden.

So ist zum Beispiel eine Netzebene, die die Koordinatenachsen in den
Punkten $2a, \sfrac{b}{2}, \sfrac{c}{3}$ schneidet,
durch die Miller-Indizes $(146)$ gegeben.
Die Rechenvorschrift lautet hier $2~\cdot~\text{reziprok}$.

Ein negativer Achsenabschnitt wird durch einen Balken über dem Index
kenntlich gemacht: $\left(\overline{h}kl\right)$.
Existiert in einer Richtung kein Achsenabschnitt, so ist der Miller-Index 0.

Der Netzebenenvektor steht senkrecht auf einer Netzebene und beschreibt
die zugehörige Netzebenenschar, sein Betrag ist gerade der Abstand zwischen zwei Netzebenen.
Dieser ist allgemein
\begin{align}
  d &= \frac{1}{\sqrt{%
    \frac{h^2}{a^2} + \frac{k^2}{b^2} + \frac{l^2}{c^2}
  }}, \\
  \intertext{was sich in einem kubischen Gitter auf}
  \label{eq:netzebenensbstand}
  d &= \frac{a}{\sqrt{%
    h^2 + k^2 + l^2
  }}
\end{align}
vereinfacht.
Hierbei sind $a, b, c$ die Längen der Gittervektoren,
im kubischen Gitter bezeichnet als Gitterkonstante.

\subsection{Beugung von Röntgenstrahlung an Kristallen}
Aufgrund der räumlichen Periodizität der Kristalle spielen
Interferenzeffekte eine große Rolle bei der Röntgenbeugung.

Ein Teilchen mit Masse $m$ und Ladung $q$ emittiert in einem
unpolarisierten elektrischen Wechselfeld eine Strahlung der Intensität
\begin{equation}
  \label{eq:strahlung_elektron}
  I_\text{e} (r, \theta) = I_0 {\left(\frac{\mu_0 q^2}{4\pi m}\right)}^2 \frac{1}{r^2} \frac{1+ \cos^2\! (2\theta)}{2},
\end{equation}
wobei $2\theta$ der Winkel zwischen einfallendem und gestreuten Strahl ist.
Es ist zu erkennen, dass die Streuung an Atomen aufgrund ihrer hohen Masse
vernachlässigt werden kann.

Die Streuung an Hüllenelektronen muss im Gegensatz zur Streuung an freien
Elektronen korrigiert werden:
\begin{equation}
  \label{eq:strahlung_huellenelektron}
  I (r, \theta, z) = I_0 {\left(\frac{\mu_0 {\left(z e_0 \right)}^2}{4\pi z m_0}\right)}^2 \frac{1}{r^2} \frac{1+\cos^2\! (2\theta)}{2}
  = z^2 I_\text{e} (r, \theta).
\end{equation}

Der Quotient $f$ der Intensitäten, mit
\begin{equation}
  f^2 = \frac{I_\text{a}}{I_\text{e}} \le 1
\end{equation}
heißt Atomformfaktor.

Der Atomformfaktor $f$ ist die Fouriertransformierte der Ladungsverteilung $\rho$ des Atoms.

Trifft nun eine Welle auf ein Atom, so existiert eine Phasendifferenz zwischen
den Streuungen an verschiedenen Elektronen in der Hülle.
Ist $\vec{k_0}$ der Wellenzahlvektor der einlaufenden und $\vec{k}$ der der gestreuten Welle,
so ist nach Abbildung~\ref{fig:phasenunterschied} der Gangunterschied der Wellen
\begin{figure}
  \centering
  \includegraphics[width=0.4\linewidth]{build/phasenunterschied.pdf}
  \caption{Skizze zur Berechnung des Phasenunterschieds bei der Streuung an zwei Punkten.\cite{anleitung}}%
  \label{fig:phasenunterschied}
\end{figure}
\begin{equation}
  \Delta s = s_1 + s_2 = \vec{r} \cdot \left(\frac{\vec{k}}{k} - \frac{\vec{k}_0}{k_0}\right).
\end{equation}
Dies bedeutet, dass der Phasenunterschied
\begin{equation}
  \Delta \varphi = 2 \pi \frac{\Delta s}{\lambda} = 2 \pi \vec{r} \cdot \left(\vec{k} - \vec{k}_0\right)
\end{equation}
ist.

% Gemäß Abbildung~\ref{fig:wellenvektor} ist gerade
% \begin{equation}
%   \left| \vec{k} - \vec{k}_0 \right| = \frac{2 \sin{\theta}}{\lambda}
% \end{equation}
% \begin{wrapfigure}{l}{0.4\textwidth}
%   \centering
%   \includegraphics[width=0.7\linewidth]{build/wellenvektor.pdf}
%   \caption{%
%     Zusammenhang zwischen Streuinkel und Wellenzahlvektoren der einfallenden und gestreuten Welle.\cite{anleitung}
%   }%
%   \label{fig:wellenvektor}
% \end{wrapfigure}

Die Phasendifferenz bei der Streuung an der Elementarzelle kann analog berechnet werden.
Um die Streuamplitude $A$ zu berechnen, muss phasenrichtig über alle Atom
in der Elementarzelle summiert werden.
Hier ist zu berücksichtigen, dass alle Atome einen eigenen Formfaktor $f_j$ haben:
\begin{equation}
  A = \sum_j f_j \exp\left(-2\pi i \vec{r_j} \cdot\left(\vec{k} - \vec{k}_0 \right)\right) I_e
\end{equation}

Werden die Ortsvektoren der Atome $\vec{r_j}$ durch die Basisvektoren dargestellt, so ist
\begin{equation}
  \label{eq:struktufaktor_atome}
  S := \sum_j f_j \exp\left(-2\pi i \left(x_j \vec{a} + y_j \vec{b} + z_j \vec{c}\right) \cdot \left(\vec{k} - \vec{k}_0 \right)\right) I_e
\end{equation}
der Strukturfaktor der Elementarzelle.

Die Forderung nach konstruktiver Interferenz erfordert die Beziehung
\begin{equation}
  \label{eq:bragg}
  n \lambda = 2 d \sin{\theta}, \hspace{2em} n \in \mathbb{N}.
\end{equation}
Dies ist die Braggsche Bedingung.
In allen anderen Fällen sind die Interferenzen destruktiv und Beugungsreflexe entfallen.

Um den Strukturfaktor noch übersichtlicher zu gestalten, wird der reziproke Gittervektor eingeführt.
Dieser ist, ausgedrückt mit den Wellenzahlvektoren
\begin{equation}
  \label{eq:gittervektor}
  \vec{g} = \vec{k} - \vec{k}_0.
\end{equation}
Der reziproke Gittervektor ist der charakteristische Netzebenenvektor der Netzebenenscharen im reziproken Gitter.
Dieses wird durch reziporke Basisvektoren aufgespannt.
Sie lassen sich folgendermaßen aus den normalen Basisvektoren berechnen:
\begin{align}
  \label{eq:reziproke_basisvektoren}
  \vec{A} &= \frac{1}{V} \left(\vec{b} \times \vec{c}\right), \\
  \vec{B} &= \frac{1}{V} \left(\vec{c} \times \vec{a}\right), \\
  \vec{C} &= \frac{1}{V} \left(\vec{a} \times \vec{b}\right), \\
  \intertext{mit dem Volumen}
  V &= \vec{a} \cdot \left(\vec{b} \times \vec{c}\right).
\end{align}

Der reziproke Gittervektor lautet dann
\begin{equation}
  \label{eq:gittervektor_basis}
  \vec{g}(hkl) = h\vec{A} + k\vec{B} + l\vec{C}.
\end{equation}

Somit lautet der Strukturfaktor mit Gleichungen~\eqref{eq:struktufaktor_atome},~\eqref{eq:gittervektor},~\eqref{eq:gittervektor_basis}
\begin{equation}
  S\left(hkl\right) = \sum_j f_j \exp\left(x_j h + y_j k + z_j l\right).
\end{equation}

Ein ausfallender Strukturfaktor, für den ein Braggscher Winkel existiert,
gibt Hinweise auf die Struktur von Kristallen,
ebenso wie der existierende Strukturfaktor.

Um die Gitterkonstanten zu bestimmen,
wird der Netzebenensabstand~\ref{eq:netzebenensbstand}
in die Braggsche Bedingung~\ref{eq:bragg} eingesetzt:
\begin{equation}
  \label{eq:const}
  {\left(\frac{2a}{\lambda}\right)}^2
  = \frac{n^2 \cdot \left(h^2 + k^2 + l^2\right)}{\sin^2\!\theta_i}
  = \frac{m_i}{\sin^2\!\theta_i} = \text{const.}
\end{equation}
Nun kann $a$ bestimmt werden, da die Wellenlänge $\lambda$ der Röntgenröhre bekannt ist.

% Streuzentren
% Streuung an
% - Elektronen
% - Atomen
% - Einheitszellen
% Atomformfaktor
% Phasenunterschiede
% Fouriertrafo
% Interferenzen
% Strukturamplitude $S$
% Braggsche Bedingung
% reziproke Gittervektoren
% reziprokes Giter

\subsection{Messmethode}
Da ein Einkristall sehr wenig konstruktive Interferenzen zulässt,
müsste Röntgenstrahlung zufällig aus der richtigen Richtung auf den Kristall
strahlen, um überhaupt einen Reflex aufzubringen.
Um die Reflexwahrscheinlichkeit zu erhöhen, wird die Probe pulverisiert.
Die Anordnung der Kristalle ist dann nämlich statistisch verteilt
und es mehr Röntgenstrahlen treffen im Braggschen Winkel auf die Kristalle.

Die gebeugte Röntgenstrahlung der Netzebenenschar $(hkl)$ trifft unter
dem Winkel $\theta$ auf einen Filmstreifen.
Aufgrund der Verteilung der Orientierungen wird die reflektierte Strahlung
als Kreis sichtbar.

Die Probe wird während der Messung rotiert, um die Orientierungen der Kristalle
weiterhin zu variieren.

Das Messverfahren ist in Kapitel~\ref{sub:messung} weiter beschrieben.

Es können bei der Messung zwei systematische Fehler auftreten.
\begin{enumerate}
  \item\label{item:absorption} Absorption der Röntgenstrahlung \\
    Die Röntgenstrahlung wird nur am Rand der zylindrischen Probe gestreut,
    im mittleren Teil wird aufgrund der Dicke die Röntgenstrahlung absorbiert.
    Vergleiche hierzu Abbildung~\ref{fig:fehler_absorption}.
    Nach Bradley und Jay\cite{bradleyjay} lautet die Korrektur
    \begin{equation}
      \label{eq:korrektur_lokalisation}
      \frac{\Delta a_\text{A}}{a} = \frac{\rho}{2R}\left(1 - \frac{R}{F}\right) \frac{\cos^2\!\theta}{\theta}
    \end{equation}
  \item\label{item:lokalisierung} Lokalisierung der Probenachse \\
    Die Verschiebung der Probenachse gegen die Achse des Films bringt
    einen Winkelfehler von
    \begin{equation}
      4 \Delta \theta = -\frac{2 \Delta s}{R}
    \end{equation}
    Vergleiche hierzu Abbildung~\ref{fig:fehler_lokalisation}.
    Geometrische Überlegungen zeigen, dass die Korrektur
    \begin{equation}
      \frac{a_\text{V}}{a} = \frac{v}{R} \cos^2\!{\theta}
    \end{equation}
    ist.
\end{enumerate}

\begin{figure}
  \centering
  \begin{subfigure}[b]{0.48\textwidth}
    \centering
    \includegraphics[width=0.9\linewidth]{build/ausdehnung_probe.pdf}
    \caption{%
      Absorption der Röntgenstrahlung.
    }%
    \label{fig:fehler_absorption}
  \end{subfigure}
  %
  \begin{subfigure}[b]{0.48\textwidth}
    \centering
    \includegraphics[width=0.9\linewidth]{build/verzug_probe.pdf}
    \caption{%
      Lokalisierung der Probenachse.
    }%
    \label{fig:fehler_lokalisation}
  \end{subfigure}
  \caption{%
    Systematischer Fehler in der Messung.\cite{anleitung}
  }
\end{figure}

Der systematische Fehler $\Delta a_\text{V}$ ist groß gegenüber $\Delta a_\text{A}$.
Dies bedeutet, dass
\begin{equation}
  \Delta a_\text{ges} = \Delta a_\text{V} + \Delta a_\text{A} \propto \cos^2\!{\theta}.
\end{equation}
Bei einer linearen Regression dieser Parameter ist
der Achsenabschnitt der $a$-Achse der beste Schätzer für die Gitterkonstante.

% Debye-Scherrer
% kristalline Probe
% Filmmethode
% Kollimator
% Kegelmantel
% Probe drehen
% systematischer Fehler (2 Stück)

\subsection{Röntgenstrahlung}%
\label{sub:roentgenstrahlung}
Die für den Versuch benötigte Röntgenstrahlung wird durch eine Röntgenröhre erzeugt.
Die Beschleunigungsspannung $U$ ist groß gegen die $K_\alpha$-Linie der Anode.
Mittels eines $\beta$-Filters wird die zusätzlich entstehende $K_\beta$-Linie eliminiert.
Übrig bleiben zwei $K_\alpha$-Linien, deren Wellenlängen aufgrund der unterschiedlichen
Emissionswahrscheinlichkeiten zu $\overline{\lambda}_{K_\alpha} = \SI{1.5417}{\angstrom}$
gemittelt wird.
Bei Winkeln im Bereich von \SI{90}{\degree} muss die Aufspaltung beiden $K_\alpha$-Linien
berücksichtigt werden:
\begin{equation}
  \Delta \theta_{1, 2} = \frac{\lambda_1 - \lambda_2}{\overline{\lambda}} \tan{\theta}
\end{equation}

% Röntgenstrahlung
% Wellenlängen
% Aufgabe
