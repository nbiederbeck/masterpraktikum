\section{Theorie}\label{sec:Theorie}
\nocite{anleitung}

Der Großteil der Materie, die sich im festen Aggregatzustand befindet,
ist kristallin aufgebaut.
Dies bedeutet, dass seine Atome räumlich periodisch angeordnet sind,
aufgrund von zwischenatomaren Kräfte.
Da viele Festkörper, sogenannte Polykristallite, aus vielen Kristalliten bestehen,
muss über die in ihnen enthaltenen Kristalle mitteln, und damit beschränkt sich die
Untersuchung von Kristalleigenschaften auf Einkristalle oder einzelne Kristalle
von Polykristallen.

Um die periodische Struktur von Kristallen aufzulösen, wird eine Sonde in der
Größenordnung der Atomabstände benötigt.
Hierfür eignen sich Elektronen, langsame Neutronen und insbesondere Röntgenstrahlen.

In diesem Versuch wird die Debye-Scherre-Methode zur Kristallstrukturuntersuchung beschrieben,
die die Beugung von Röntgenstrahlung am Kristallgitter nutzt.

\subsection{Kristallstrukturen}%
\label{sub:kristallstrukturen}
% Basis
% Basisvektor
% Punktgitter
% Translation
% Elementarzelle
% Symmetrieeigenschaften
% Bravais-Gitter
Die Beschreibung der räumlichen Periodizität basiert auf einem Punktgitter.
Jeder Gitterpunkt kann aus einem oder mehreren Atomen bestehen,
dies wird als Basis bezeichnet.
In Abbildung~\ref{fig:punktgitter} sind ein Punktgitter, eine Basis und ein Basisgitter
dargestellt.
\begin{figure}
  \centering
  \includegraphics[width=0.8\linewidth]{build/punktgitter.pdf}
  \caption{Darstellung von Punktgitter, Basis, Basisgitter (v.l.).\cite{anleitung}}%
  \label{fig:punktgitter}
\end{figure}

Die Vektoren $\vec{a}, \vec{b}, \vec{c}$, die bei einer Translation
\begin{equation}
  \vec{T} = x\vec{a} + y\vec{b} + z\vec{c}
\end{equation}
mit $\left\{x, y, z\right\} = 1$ das Gitter in sich selbst abbilden, heißen Basisvektoren.
Ihre Wahl ist nahezu beliebig, jedoch werden aus Symmetriegründen jene gewählt,
deren Länge dem Abstand der nächsten Nachbarn der Atome gleicht.

So existieren 14 verschiedene Gitter, die Bravais-Gitter genannt werden
und in 7 Gittertypen eingeteilt werden.
Einige davon werden in Kapitel~\ref{sub:kubische_kristallstrukturen} beschrieben.

Das von den Basisvektoren aufgespannte Parallelepided mit Volumen
\begin{equation}
  V = \vec{a} \cdot \left(\vec{b} \times \vec{c}\right)
\end{equation}
heißt Einheitszelle.


\subsection{Kubische Kristallstrukturen}%
\label{sub:kubische_kristallstrukturen}
% kubisch-raumzentriert
% kubisch-flächenzentriert
% Zinkblende-Struktur
% Steinsalz-Struktur 
% Cäsiumchlorid-Struktur 
% Fluorit-Struktur 
% hexagonale Struktur 
Die einfachsten Kristallstrukturen haben kubische Anordnung.
Dies bedeutet, dass ihre Basisvektoren jeweils senkrecht aufeinander stehen.
Hierbei wird zwischen ein- und mehratomigen Basen unterschieden.
Das einfachste (\textit{primitive}) kubische Gitter hat auf je ein Atom auf den Ecken eines Würfels.
Da sich das Atom auf der Ecke genau acht Würfelgitter teilt,
besteht die Elementarzelle also aus einem Atom.

Ein Gitter heißt \textit{kubisch-raumzentriert}, wenn zusätzlich zum primitiven Gitter
ein Atom in der Mitte des Würfels zu finden ist.
Die Elementarzelle besteht hier aus zwei Atomen.
Die Atome liegen, geschrieben in Koordinaten der Basisvektoren, bei
\begin{align*}
  % \label{eq:basisvektoren_kubisch-raumzentriert}
  a_1 = {\left(0, 0, 0\right)}^T, \hspace{2em} a_2 = {\left(\sfrac{1}{2}, \sfrac{1}{2}, \sfrac{1}{2}\right)}^T.
\end{align*}

Ein weiteres einatomiges Gitter ist das \textit{kubisch-flächenzentrierte} Gitter.
Hier sind die Atome sowohl auf den Ecken, als auch in der Mitte jeder Seitenfläche.
Die Elementarzelle besteht somit aus $8 \cdot \sfrac{1}{8} + 6 \cdot \sfrac{1}{2} = 4$ Atomen.
Die Atome liegen an den Stellen
\begin{align*}
  % \label{eq:basisvektoren_kubisch-flächenzentriert}
  a_1 = {\left(0, 0, 0\right)}^T, \hspace{2em}
  a_2 = {\left(0, \sfrac{1}{2}, \sfrac{1}{2}\right)}, \hspace{2em}
  a_3 = {\left(\sfrac{1}{2}, 0, \sfrac{1}{2}\right)}, \hspace{2em}
  a_4 = {\left(\sfrac{1}{2}, \sfrac{1}{2}, 0\right)}.
\end{align*}

Die \textit{Zinkblende-Struktur} besteht aus einer zweiatomigen Basis,
deren Atome jeweils auf primitiven Gittern sitzen.
Die Elementarzelle besteht aus zwei Atomen, sie sitzen also bei
\begin{align*}
  % \label{eq:basisvektoren_zinkblende}
  a_1 = {\left(0, 0, 0\right)}^T, \hspace{2em} b_1 = {\left(\sfrac{1}{4}, \sfrac{1}{4}, \sfrac{1}{4}\right)}^T.
\end{align*}
lauten.

\subsection{Netzebenen}
% Miller Indizes
% Netzebenenschar
% Netzebenenabstand
In sogenannten Netzebenen liegen die Schwerpunkte der Atome.
Parallele Netzebenen bilden eine Ebenenschar.
Die Gesamtheit einer Ebenenschar wird durch die drei
\textit{Millerschen Indizes} $\left(hkl\right)$ beschrieben.
Diese werden bestimmt, indem die reziproken Achsenabschnitte
der dem Koordinatenursprung nächsten Netzebene in Bruchteilen der
Basisvektorenlänge angegeben wird.
Weiterhin sollten die Indizes aus Rechengründen auf ganze Zahlen gebracht werden.

So ist zum Beispiel eine Netzebene, die die Koordinatenachsen in den
Punkten $2a, \sfrac{b}{2}, \sfrac{c}{3}$ schneidet,
durch die Miller-Indizes $(146)$ gegeben.
Die Rechenvorschrift lautet hier $2~\cdot~\text{reziprok}$.

Ein negativer Achsenabschnitt wird durch einen Balken über dem Index
kenntlich gemacht: $\left(\overline{h}kl\right)$.
Existiert in einer Richtung kein Achsenabschnitt, so ist der Miller-Index 0.

Der Netzebenenabstand steht senkrecht auf einer Ebenenschar und beschreibt
gerade den abstandt zwischen zwei Netzebenen.
Er ist gerade
\begin{equation}
  \label{eq:netzebenensbstand}
  d = \frac{1}{\sqrt{%
    \frac{h^2}{a^2} + \frac{k^2}{b^2} + \frac{l^2}{c^2}
  }}.
\end{equation}

\subsection{Beugung von Röntgenstrahlung an Kristallen}
% Streuzentren 
% Streuung an 
% - Elektronen 
% - Atomen 
% - Einheitszellen 
% Atomformfaktor 
% Phasenunterschiede 
% Fouriertrafo 
% Interferenzen 
% Strukturamplitude $S$ 
% Braggsche Bedingung 
% reziproke Gittervektoren 
% reziprokes Giter 

\subsection{Messmethode}
% Debye-Scherrer 
% kristalline Probe 
% Filmmethode 
% Kollimator 
% Kegelmantel 
% Probe drehen 
% systematischer Fehler (2 Stück) 
% Röntgenstrahlung 
% Wellenlängen 
% Aufgabe 
