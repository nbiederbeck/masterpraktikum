\section{Theorie}\label{sec:theorie}
Mit Hilfe eines Operationsverstaerker lassen sich Ausgangsspannungen $U_A$
erzeugen welche in einem Bereich proportional zur Differenz der beiden
angelegten Spannungen ist. 
Ziel des Versuches ist es verschiedene Schaltungen mittels eines
Operationsverstaerker zu realisieren und die Unterschiede zwischen eines idealen
und realen Operationsverstaerker zu veranschaulichen.

\subsection{idealer Operationsverstaerker}%
\label{sub:idealer_operationsverstaerker}

\begin{wrapfigure}{l}{0.4\textwidth}
		\centering
		\includegraphics[width=\linewidth]{./build/operationsverstaerker.pdf}
		\caption{Schaltbild eines idealen Operationsverstaerkers.
		\cite{anleitung}}
		\label{fig:opv}
\end{wrapfigure}
In einem Schaltbild wird ein Operationsverstaerker durch ein gleichseitiges
Dreieck gekennzeichnet. 
An diesem werden zwei Eingangsspannungen angeschlossen, wobei die
Eingangsspannungen $U_\text{p}$ in Phase
und $U_\text{n}$ in Gegenphase zur Ausgangsspannung
$U_\text{a}$ ist.
Der Eingang $U_\text{p}$ wird aufgrund seiner Phasenbeziehung zur Ausgangsspannung
als nicht-invertierender und $U_\text{n}$ als invertierender Eingang bezeichnet. 
Im linearen Aussteuerungsberreich verstaerkert der Operationsverstaerker die
Differenz zwischen den beiden Eingaengen mit dem Verstaerkungsfaktor $V$ linear
\begin{equation}
		U_\text{a} = V \left( U_\text{p} - U_\text{n} \right)
\end{equation}
wohinauf die Ausgangsspannung bei groeseren eingangsspannung gegen einen
Saettigungswert strebt.
Der Verstaerkungsfaktor ist durch die Betriebspannung festgelegt
und bildet die Verstaerkungsgrenzen $\pm U_\text{B}$  des Linearverstaerkers.
In Abbildung \ref{fig:kennlinie} ist eine Beispielhafte Kennline eines
Operationsverstaerkers zu sehen. 
\begin{figure}[h]
		\centering
		\includegraphics[width=0.5\linewidth]{build/aussteuerungsbereich.pdf}
		\caption{Kennlinie eines Operationsverstaerkers. \cite{anleitung}}
		\label{fig:kennlinie}
\end{figure}
Fuer den idealen Operationsverstaerker werden die annahmen getroffen das der
Verstaerkungsfaktor $V = \infty$, der Eingangswiderstand $r_\text{e} =
\infty$ und der Ausgangswiderstand $r_\text{a} = 0$ ist.
Die Annahmen des idealen Operationsverstaerkers beschreiben aber nur begrenzt
die realitaet, weswegen korrekturterme eingefuehrt werden. 

\subsection{Realer Operationsverstaerker}%
\label{sub:realer_operationsverstaerker}

Zut beschreibung eines realen Operationsverstaerkers muessen noch korrektur
Therme eingefuehrt werden, welche beim idealen verschwinden. 
Wird auf beiden eingaengen dieselbe Spannung $U_\text{Gl}$ angelegt, so
verschwindet nach Formel \ref{eq:??} die Ausgangsspannung beim idealen Opv. 
Beim realen werden aufgrund von Unsymmetrien dennoch Spannugen gemessen so dass
sich die Gleichtaktverstaerkung 
\begin{equation}
		V_\text{Gl} = \frac{\Delta U_\text{A}}{\Delta U_\text{Gl}}
\end{equation}
definiert werden kann.
Ebenso sind die Eingangswiderstaende endlich, sodass es zu Eingangsstroemen auf
den beiden Eingaengen $I_\text{p}$ und $I_\text{n}$ kommt.
Der Offsetstrom wird als die Differenz beider Stroeme definiert 
\begin{equation}
		I_0 = I_\text{p} - I_\text{n}
\end{equation}
wenn beide Eingangsspannungen verschwinden. 
Der Eingangsruhestrom $I_\text{B}$ ergibt sich aus dem Mittel der beiden Eingangsstroeme.
Fuer den Fall das eine Eingangsspannung $U_{i}$ verschwindet ergibt sich der
Differenzeingangswiderstand $r_\text{D}$ zu
\begin{equation}
		r_\text{D} = \frac{\Delta U_\text{i}}{\Delta I_\text{i}} \hspace{2cm} U_\text{j} = 0 
\end{equation}
wobei $i,j \in p,n$ und $i \neq j$.
Der Gleichtakteingangswiderstand ist definiert ueber die Gleichtaktsspannung
$U_\text{GL}$ und dem Gleichtaktsstrom $I_\text{Gl} = I_\text{p} + I_\text{n}$.
Die offsetspannung $U_0$ ist die Spannungsdifferenz um die die Eingaenge
verschoben werden muessen damit die Auspangsspannung $U_\text{a}$ verschwindet.
\begin{equation}
		U_0 = U_\text{p} - U_\text{n}
\end{equation}

\subsection{Linearverstaerker}%
\label{sub:linearverstaerker}
Aufgrund des schmalen Aussteuerungsberreich sind Linearverstaerker passe nur
begrenzt einsetzbar.
Zur Verbreiterung des Aussterungsbereich der Operationverstaerker mit einem 
ein Gegenkopplungszweig beschaltet.
\begin{figure}[h]
		\centering
		\includegraphics[width=0.8\linewidth]{build/linearverstaerker.pdf}
		\caption{Gegengekoppelter Linearverstaerker}
		\label{fig:lin}
\end{figure}
Dabei wird ein Teil der Ausgangsspannung ueber den Widerstand $R_\text{n}$ ueber
den Gegenkopplungszweig zum invertierten Eingang zurueck gegeben. 
Dadurch laesst sich eine Zunahme des Aussteuerungsberreich erreichen, wobei die
Gesamtverstaerkung $V'$ abnimmt.
Bei grossen Leerlaufverstaerkungen ist die Spannung $U_\text{n}$ gering, sodass
sich nach dem ersten Kirchhoffschen Gesetz fuer den Knoten A
\begin{equation}
		\frac{U_1}{R_1} + \frac{U_\text{A}}{R_\text{n}} = 0
\end{equation}
der Strom Verschwinden muss. 
Aus dem Verhaeltniss der Eingangs $U_1$ zur Ausgangsspannung $U_\text{A}$ ergibt
sich der Verstaerkungsfaktor.
\begin{equation}
		V' = - \frac{R_\text{N}}{R_1}
\end{equation}
Die Verstaerkung ist im Idealfall nur von dem Verhaeltniss der beiden
Widerstaende abhaenigig. 
Im realfall besehen noch Abhaenigikeit zwischen der Leerlaufverstaerkung, den
Eingangs- und Ausgangswiderstaenden usw. 
Fuer einen unbelasteten Spannungsteiler ($I_\text{n} = 0$) gilt dass bei einem
endlichen Leerlaufverstaerkungsfaktor
\begin{equation}
		U_\text{N} = \frac{U_\text{A}}{V}	
\end{equation}
Anhand der 1 Kirchhoffschen Regel gilt fuer den Punkt $A$
\begin{equation}
		\frac{U_\text{N}-U_1}{U_\text{a}-U_1} = \frac{R_1}{R_1 + R_n}
\end{equation}
Daraus resultiert fuer den Verstaerkungsfaktor 
\begin{equation}
		\frac{1}{V'}= - \frac{U_1}{U_\text{A}} = \frac{1}{V} +
		\frac{R_1}{R_\text{N}} \left( a + \frac{1}{V} \right) \approx
		\frac{1}{V} + \frac{R_1}{R_\text{N}}
\end{equation}
sodass bei der Wahl von $R_\text{n}/ R_\text{1}$ viel kleiner als die
Leerlaufverstaerkung $V$ ein idealer Operationverstaerker erhaelt mit einer
geringerem Verstaerkungsgrad und grossem Aussteuerungsberreich.

\begin{equation}
		g := \frac{V}{V'}
\end{equation}

\begin{equation}
		\frac{\Delta V'}{V'} = \frac{\Delta V}{g V}
\end{equation}

\begin{figure}[h]
		\centering
		\includegraphics[width=0.8\linewidth]{build/frequenz_lin.pdf}
		\caption{}
		\label{fig:freq}
\end{figure}

\begin{figure}[h]
		\centering
		\includegraphics[width=0.8\linewidth]{build/n_inv_elektrometer.pdf}
		\caption{test}
		\label{fig:test}
\end{figure}

\begin{equation}
		V' = \frac{U_\text{A}}{U_1} = \frac{U_\text{A}}{U_\text{N}} =
		\frac{R_\text{N} + R_1}{R_1}
\end{equation}

\begin{figure}[h]
		\centering
		\includegraphics[width=0.8\linewidth]{build/amperemeter.pdf}
		\caption{}
		\label{fig:}
\end{figure}
