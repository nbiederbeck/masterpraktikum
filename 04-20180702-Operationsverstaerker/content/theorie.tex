\section{Theorie}\label{sec:theorie}
Mit Hilfe eines Operationsverstaerker lassen sich Ausgangsspannungen $U_A$
erzeugen welche in einem Bereich proportional zur Differenz der beiden
angelegten Spannungen ist. 
Ziel des Versuches ist es verschiedene Schaltungen mittels eines
Operationsverstaerker zu realisieren und die Unterschiede zwischen eines idealen
und realen Operationsverstaerker zu veranschaulichen.

\subsection{idealer Operationsverstaerker}%
\label{sub:idealer_operationsverstaerker}

In einem Schaltbild wird ein Operationsverstaerker durch ein gleichseitiges
Dreieck gekennzeichnet. 
An diesem werden zwei Eingangsspannungen angeschlossen, wobei die
Eingangsspannungen $U_\text{p}$ in Phase
und $U_\text{n}$ in Gegenphase zur Ausgangsspannung
$U_\text{a}$ ist.
\begin{figure}[h]
		\centering
		\includegraphics[width=0.5\linewidth]{./build/operationsverstaerker.pdf}
		\caption{Schaltbild eines idealen Operationsverstaerkers.
		\cite{anleitung}}
		\label{fig:opv}
\end{figure}
Der Eingang $U_\text{p}$ wird aufgrund seiner Phasenbeziehung zur Ausgangsspannung
als nicht-invertierender und $U_\text{n}$ als invertierender Eingang bezeichnet. 
Im linearen Aussteuerungsberreich verstaerkert der Operationsverstaerker die
Differenz zwischen den beiden Eingaengen mit dem Verstaerkungsfaktor $V$ linear
\begin{equation}
		U_\text{a} = V \left( U_\text{p} - U_\text{n} \right)
\end{equation}
wohinauf die Ausgangsspannung bei groeseren eingangsspannung gegen einen
Saettigungswert strebt.
Der Verstaerkungsfaktor ist durch die Betriebspannung festgelegt
und bildet die Verstaerkungsgrenzen $\pm U_\text{B}$  des Linearverstaerkers.
In Abbildung \ref{fig:kennlinie} ist eine Beispielhafte Kennline eines
Operationsverstaerkers zu sehen. 
\begin{figure}[h]
		\centering
		\includegraphics[width=0.5\linewidth]{build/aussteuerungsbereich.pdf}
		\caption{Kennlinie eines Operationsverstaerkers. \cite{anleitung}}
		\label{fig:kennlinie}
\end{figure}
