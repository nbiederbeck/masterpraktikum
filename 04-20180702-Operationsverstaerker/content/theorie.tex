\section{Theorie}\label{sec:theorie}
Mit Hilfe eines Operationsverstaerker lassen sich Ausgangsspannungen $U_A$
erzeugen welche in einem Bereich proportional zur Differenz der beiden
angelegten Spannungen ist. 
Ziel des Versuches ist es verschiedene Schaltungen mittels eines
Operationsverstaerker zu realisieren und die Unterschiede zwischen eines idealen
und realen Operationsverstaerker zu veranschaulichen.

\subsection{idealer Operationsverstaerker}%
\label{sub:idealer_operationsverstaerker}

\begin{wrapfigure}{l}{0.4\textwidth}
		\centering
		\includegraphics[width=\linewidth]{./build/operationsverstaerker.pdf}
		\caption{Schaltbild eines idealen Operationsverstaerkers.
		\cite{anleitung}}
		\label{fig:opv}
\end{wrapfigure}
In einem Schaltbild wird ein Operationsverstaerker durch ein gleichseitiges
Dreieck gekennzeichnet. 
An diesem werden zwei Eingangsspannungen angeschlossen, wobei die
Eingangsspannungen $U_\text{p}$ in Phase
und $U_\text{n}$ in Gegenphase zur Ausgangsspannung
$U_\text{a}$ ist.
Der Eingang $U_\text{p}$ wird aufgrund seiner Phasenbeziehung zur Ausgangsspannung
als nicht-invertierender und $U_\text{n}$ als invertierender Eingang bezeichnet. 
Im linearen Aussteuerungsberreich verstaerkert der Operationsverstaerker die
Differenz zwischen den beiden Eingaengen mit dem Verstaerkungsfaktor $V$ linear
\begin{equation}
		U_\text{a} = V \left( U_\text{p} - U_\text{n} \right)
\end{equation}
wohinauf die Ausgangsspannung bei groeseren eingangsspannung gegen einen
Saettigungswert strebt.
Der Verstaerkungsfaktor ist durch die Betriebspannung festgelegt
und bildet die Verstaerkungsgrenzen $\pm U_\text{B}$  des Linearverstaerkers.
In Abbildung \ref{fig:kennlinie} ist eine Beispielhafte Kennline eines
Operationsverstaerkers zu sehen. 
\begin{figure}[h]
		\centering
		\includegraphics[width=0.5\linewidth]{build/aussteuerungsbereich.pdf}
		\caption{Kennlinie eines Operationsverstaerkers. \cite{anleitung}}
		\label{fig:kennlinie}
\end{figure}
Fuer den idealen Operationsverstaerker werden die annahmen getroffen das der
Verstaerkungsfaktor $V = \infty$, der Eingangswiderstand $r_\text{e} =
\infty$ und der Ausgangswiderstand $r_\text{a} = 0$ ist.
Die Annahmen des idealen Operationsverstaerkers beschreiben aber nur begrenzt
die realitaet, weswegen korrekturterme eingefuehrt werden. 

\subsection{Realer Operationsverstaerker}%
\label{sub:realer_operationsverstaerker}

Zut beschreibung eines realen Operationsverstaerkers muessen noch korrektur
Therme eingefuehrt werden, welche beim idealen verschwinden. 
Wird auf beiden eingaengen dieselbe Spannung $U_\text{Gl}$ angelegt, so
verschwindet nach Formel \ref{eq:??} die Ausgangsspannung beim idealen Opv. 
Beim realen werden aufgrund von Unsymmetrien dennoch Spannugen gemessen so dass
sich die Gleichtaktverstaerkung 
\begin{equation}
		V_\text{Gl} = \frac{\Delta U_\text{A}}{\Delta U_\text{Gl}}
\end{equation}
definiert werden kann.
Ebenso sind die Eingangswiderstaende endlich, sodass es zu Eingangsstroemen auf
den beiden Eingaengen $I_\text{p}$ und $I_\text{n}$ kommt.
Der Offsetstrom wird als die Differenz beider Stroeme definiert 
\begin{equation}
		I_0 = I_\text{p} - I_\text{n}
\end{equation}
wenn beide Eingangsspannungen verschwinden. 
Der Eingangsruhestrom $I_\text{B}$ ergibt sich aus dem Mittel der beiden Eingangsstroeme.
Fuer den Fall das eine Eingangsspannung $U_{i}$ verschwindet ergibt sich der
Differenzeingangswiderstand $r_\text{D}$ zu
\begin{equation}
		r_\text{D} = \frac{\Delta U_\text{i}}{\Delta I_\text{i}} \hspace{2cm} U_\text{j} = 0 
\end{equation}
wobei $i,j \in p,n$ und $i \neq j$.
Der Gleichtakteingangswiderstand ist definiert ueber die Gleichtaktsspannung
$U_\text{GL}$ und dem Gleichtaktsstrom $I_\text{Gl} = I_\text{p} + I_\text{n}$.
Die offsetspannung $U_0$ ist die Spannungsdifferenz um die die Eingaenge
verschoben werden muessen damit die Auspangsspannung $U_\text{a}$ verschwindet.
\begin{equation}
		U_0 = U_\text{p} - U_\text{n}
\end{equation}

\subsection{Linearverstaerker}%
\label{sub:linearverstaerker}
Aufgrund des schmalen Aussteuerungsberreich sind Linearverstaerker passe nur
begrenzt einsetzbar.
Zur Verbreiterung des Aussterungsbereich der Operationverstaerker mit einem 
ein Gegenkopplungszweig beschaltet.
\begin{figure}[h]
		\centering
		\includegraphics[width=0.5\linewidth]{build/linearverstaerker.pdf}
		\caption{Gegengekoppelter Linearverstaerker}
		\label{fig:lin}
\end{figure}
Dabei wird ein Teil der Ausgangsspannung ueber den Widerstand $R_\text{n}$ ueber
den Gegenkopplungszweig zum invertierten Eingang zurueck gegeben. 
Dadurch laesst sich eine Zunahme des Aussteuerungsberreich erreichen, wobei die
Gesamtverstaerkung $V'$ abnimmt.
Bei grossen Leerlaufverstaerkungen ist die Spannung $U_\text{n}$ gering, sodass
sich nach dem ersten Kirchhoffschen Gesetz fuer den Knoten A
\begin{equation}
		\frac{U_1}{R_1} + \frac{U_\text{A}}{R_\text{n}} = 0
\end{equation}
der Strom Verschwinden muss. 
Aus dem Verhaeltniss der Eingangs $U_1$ zur Ausgangsspannung $U_\text{A}$ ergibt
sich der Verstaerkungsfaktor.
\begin{equation}
		V' = - \frac{R_\text{N}}{R_1}
\end{equation}
Die Verstaerkung ist im Idealfall nur von dem Verhaeltniss der beiden
Widerstaende abhaenigig. 
Im realfall besehen noch Abhaenigikeit zwischen der Leerlaufverstaerkung, den
Eingangs- und Ausgangswiderstaenden usw. 
Fuer einen unbelasteten Spannungsteiler ($I_\text{n} = 0$) gilt dass bei einem
endlichen Leerlaufverstaerkungsfaktor
\begin{equation}
		U_\text{N} = \frac{U_\text{A}}{V}	
\end{equation}
Anhand der 1 Kirchhoffschen Regel gilt fuer den Punkt $A$
\begin{equation}
		\frac{U_\text{N}-U_1}{U_\text{a}-U_1} = \frac{R_1}{R_1 + R_n}
\end{equation}
Daraus resultiert fuer den Verstaerkungsfaktor 
\begin{equation}
		\frac{1}{V'}= - \frac{U_1}{U_\text{A}} = \frac{1}{V} +
		\frac{R_1}{R_\text{N}} \left( a + \frac{1}{V} \right) \approx
		\frac{1}{V} + \frac{R_1}{R_\text{N}}
\end{equation}
sodass bei der Wahl von $R_\text{n}/ R_\text{1}$ viel kleiner als die
Leerlaufverstaerkung $V$ ein idealer Operationverstaerker erhaelt mit einer
geringerem Verstaerkungsgrad, wobei der Verstaerkungsfaktor nur vom Verhaeltniss
nur von dem Widerstandsverhaeltniss abhaengt.
Desweiteren wird durch die Gegenkopplung die Stabilitaet der
Verstaerkerschaltung erhoeht, sowie der Ausgangswiderstand um ein g-faches
verkleinert.
\begin{equation}
		g := \frac{V}{V'}
\end{equation}
Aequivalent wird die Leerlaufverstaerkung vermindert.
\begin{equation}
		\frac{\Delta V'}{V'} = \frac{\Delta V}{g V}
\end{equation}
Ebenso nimmt die Bandbreite fuer grosse g zu, sodass aufgrund eines grosseren
Grequenzbandes mehr Frequenzen unverzerrt verstaerkt werden koennen.
Bildet das Produkt aus Verstaerkung und Bandbreite eine Konstantke wird dies
durch die Transitfrequenz charakterisiert.
\begin{figure}[h]
		\centering
		\includegraphics[width=0.5\linewidth]{build/frequenz_lin.pdf}
		\caption{}
		\label{fig:freq}
\end{figure}
Dies ist die Frequenz bei dem die Verstaerkung den Wert eins annimmt.
Eine charakteristische Bandbreite eines Operationsverstaerkers ist in Abbildung
\ref{fig:freq} zu sehen.

\subsection{Tiefpass}%
\label{sub:tiefpass}

Der Verstaerkungfaktor des Operationsverstaerkers aufgetragen gegen die Frequenz bildet ein typischen
Tiefpass.
Bei diesem werden hohe Frequenzen stark gedaempft. 
Beispielhaft ist in Abbildung \ref{fig:tiefpass} ein Tiefpass erster Ordnung
dargestellt.
\begin{figure}[h]
		\centering
		\includegraphics[width=0.5\linewidth]{build/tiefpass.pdf}
		\caption{Denkbar einfachster Tiefpass erster Ordnung.}
		\label{fig:tiefpass}
\end{figure}
Unter beruecksichtigung der Maschenregel laesst sich der Verstaerkungsfaktor bzw
Daempfungsfaktor des Tiefpass ausdruecken. 
Dazu wird das Verhaeltniss aus $U_a$ zu $U_e$ gebildet woraus sich der
Blindwiderstand $H$ ergibt.
\begin{equation}
		\label{eq:blindwiederstand}
		H = \frac{Z_\text{C}}{Z_\text{C} + R} 
\end{equation}
Der Blindwiderstand $H$ ist eine Funktion der Wechselspannungfrequenz $\omega$
und dessen Realteil entspricht 
\begin{align}
		\label{eq:h_betrag}
		\left| H(\omega) \right| &= \frac{\sfrac{1}{j \omega C}}{\sfrac{1}{j\omega
		C} + R}  \\
								 &= \frac{1}{\sqrt{1 + (\omega CR)^2}}
\end{align}
dem 'Ersatzwiderstand'.
Fuer Frequenzen $w > \sfrac{1}{RC}$ wird die Eingangsspannung stark gedaempft und bildet
die abfallende Flanke in Abbildung \ref{fig:freq}.

\subsection{Umkehr-Integrator}%
\label{sub:umkehr_integrator}

Durch Rueckkopplung des Ausgangsstrom ueber ein Kondensator laesst sich die
Eingangsspannung $U_1$ integrieren (vergleiche Abbildung \ref{fig:integrator}).
\begin{figure}[h]
		\centering
		\includegraphics[width=0.5\linewidth]{build/umkehr_integrator.pdf}
		\caption{Umkehrgenerator}
		\label{fig:integrator}
\end{figure}
Die Rueckkopplung funktioniert in aequivalenz zum Linearverstaerker.
Die Ladung auf dem Kondensator entspricht dem Produkt aus der Ausgangsspannung
$U_\text{A}$ und der Kapazitaet $C$.
\begin{equation}
		\int I_\text{C} \, dt = C U_\text{A}
\end{equation}
Unter der Vernachlaessigung des Operationsverstaerkerstroms $I_\text{OV}$ gilt
fuer den Knotenpunkt A:
\begin{align}
		0 = \sum_i I_i =& I_1 + I_\text{OV} + I_\text{C} \\
		=& \frac{U_1(t)}{R} + 0 + C \cdot \frac{d U_\text{A}(t)}{dt}
\end{align}
Unter Beruecksichtigung das die Eingangsspannung von der Zeit abhaengt (z.B.
Sinusfoermig $U_1(t)= U_0 \cos(\omega t)$) ergibt sich die Integration der Eingangspannung $U_1$ nach
der Zeit.
\begin{align}
		U_\text{A}(t) &= - \frac{1}{RC} \int U_\text{1}(t) \, dt \\
				   &= \frac{U_0}{\omega R C} \cos (\omega t)
\end{align}
Fuer eine sinusfoermige Eingangspannung faellt die Verstaerkung daher linear mit 
der Frequenz.
Im allgemeinen koennen beliebig Funktionen integriert werden, wobei es bei
Unstaetigkeiten zu ueberschwingungen kommen kann (Gibbs'sches Phaenomen).

\subsection{Umkehr-Differentiator}%
\label{sub:umkehr_differentiator}
Aequivalent zum Integrator laesst sich durch austauschen der Kapazitaet durch
den Widerstand vice versa die Eingangspannung Differenzieren.
Ein Schaltbild des Umkehr-Differentiator ist in Abblidung \ref{fig:diff} zu
sehen.
\begin{figure}[h]
		\centering
		\includegraphics[width=0.5\linewidth]{build/umkehr_diff.pdf}
		\caption{um}
		\label{fig:diff}
\end{figure}
Fuer den Knotenpunkt vor dem nicht-invertierenden gilt nach der ersten
Kirchhoffschen Regel:
\begin{equation}
		C \frac{d U_1(t)}{dt} + \frac{U_A(t)}{R} = 0
\end{equation}
Durch umstellen und einmaliges differenzieren ergibt sich beispielsweise fuer
eine sinusfoermige Eingangsspannung $U_1(t)$.
\begin{align}
		U_\text{A}(t) &= - RC \frac{dU_1(t)}{dt} \\
				   &= - \omega R C U_0 \cos(\omega t)
\end{align}
Beim Differenzieren waechst die Verstaerkung bei einer sinusforemigen
Eingangsspannung linear mit der Frequenz.

\subsection{Schmitt-Trigger}%
\label{sub:schmitt_trigger}

Beim Schmitt-Trigger wird ein Teil der Ausgangsspannung auf den nicht
invertierten Eingang zurueck gegeben.
Es laesst sich fuer den Knoten vor dem rueckgekoppelten Eingang die Gleichung 
\begin{equation}
		\frac{U_1}{R_1} + \frac{U_\text{A}}{R_\text{p}} = 0
\end{equation}
aufstellen.
Sobald die Ausgangsspannung $U_\text{A}$ den Saettigungswert $U_\text{B}$ erreicht kommt es zur
Verstaerkung wenn 
\begin{equation}
		\label{eq:schmitt}
		U_1 > \frac{R_1}{R_\text{P}} U_\text{B} \ .
\end{equation}
Die Rueckgekoppelte Spannung ist nun groesser als die Angelegte und der Schalter
kippt auf den Wert $U_\text{B}$.
\begin{figure}[h]
		\centering
		\includegraphics[width=0.5\linewidth]{build/schmitt_trigger.pdf}
		\caption{Schmitt}
		\label{fig:}
\end{figure}
Schmitt-Trigger werden zur Erzeugung von Binaeren Signalen und der
Signalverstaerkung genutzt.

\subsection{Signalgenerator}%
\label{sub:signalgenerator}
Durch das betreiben des Schmitt triggers mit eier Sinusspannung wird bei
entsprechender Wahl der Wiederstaende am Ausgang ein Rechtecksignal erzeugt.
Der Ausgang des Schmitttriggers wird auf den Eingang eines daran anschliessenden
Integrator gelegt. 
\begin{figure}[h]
		\centering
		\includegraphics[width=0.8\linewidth]{build/dreiecksgen.pdf}
		\caption{}
		\label{fig:}
\end{figure}
Durch die Integration eines Rechtecksignals wird aus diesem ein Dreiecksignal
welches am Ausgang des Integrators abgenommen werden kann.
Die Periode der Dreiecks und Rechtecksspannung ist durch die Periode der
Eingangspannung $U_1$ festgelegt.

\subsection{Gedaempfte Sinusschwingung}%
\label{sub:gedaempfte_sinusschwingung}

\begin{figure}[h]
		\centering
		\includegraphics[width=0.8\linewidth]{build/schwingungsdiffgl.pdf}
		\caption{}
		\label{fig:}
\end{figure}

\begin{equation}
		\label{eq:ov2}
		U_2 = - \frac{1}{RC} \int \left( U_1 + \frac{\eta}{10} U_\text{A} \right) \, dt
\end{equation}

\begin{equation}
		\label{eq:schwingung_diff}
		\frac{d^2 U_\text{A}}{dt^2} - \frac{\eta}{10 R C} \frac{d
		U_\text{A}}{dt} + \frac{1}{(RC)^2} U_\text{A} = 0
\end{equation}


\begin{equation}
		U_\text{A}(t) = U_0 \exp \left( \frac{\eta t}{20 RC} \right) 
		\sin \left(\frac{t}{RC}\right)
\end{equation}
