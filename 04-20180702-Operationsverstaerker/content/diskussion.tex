\section{Diskussion}%
\label{sec:diskussion}


Die Abweichungen der gemessenen Verstärkungsfaktoren $V_i$
von dem theoretischen Verhältnis $\sfrac{R_N}{R_1}$ liegen
in der Größenordnung von $\SI{0.1}{\percent}$ bis $\SI{8}{\percent}$.
Dies bedeutet, dass die Schaltungen korrekt aufgebaut wurden und die Operationsverstärker nicht gesättigt waren.
Auffällig ist, dass die Fits bei Verstärkungen $> 1$ besser passen, als bei Verstärkungen $< 1$.
An Abbildung~\ref{fig:phasendiff} wird deutlich,
dass für Frequenzen $\nu > \nu_\text{G}$
die Eingangsspannung stark gedämpft wird und die Phasenverschiebung abnimmt.
Dies bedeutet, dass ein lineareverstärkender Operationsverstärker wie ein Tiefpass funktioniert, vergleiche hierzu Gleichung~\eqref{eq:tiefpass}.


Die Funktionsweisen des Umkehrintegrators und -differentiators konnten verifiziert werden.
Die drei verschiedenen Eingangssignale werden korrekt integriert bzw.\ differentiert.
Die entsprechenden Abhängigkeiten von der Frequenz, wie in Kapiteln~\ref{sub:umkehr_integrator} und~\ref{sub:umkehr_differentiator} hergeleitet,
konnte mit geringen Fehlern in den Fits nachgewiesen werden.

Es ist in Abbildung~\ref{fig:schmitt} zu erkennen,
dass der Schmitt-Trigger realisiert wurde.
Die relative Abweichung der in Gleichung~\eqref{eq:schmitt}
hergeleiteten Triggerspannung vom gemessenen Wert ist
mit etwa $\SI{6}{\percent}$ sehr gering.

Die Erzeugung einer Dreieckspannung mittels Integrator, Schmitt-Trigger und Sinuseingangspannung funktioniert wie erwartet,
an Abbildung~\ref{fig:dreieck_generator} wird deutlich,
dass die Ausgangsfrequenz gleich der Eingangsfrequenz ist.

Die gedämpfte Schwingung ist in Abbildung~\ref{fig:gedaempft_oszilloskop}
gut zu erkennen.
Die Rechteckspannung regt die Sinusspannung an,
die dann exponentiell abfällt.
Die Zeitkonstante $\tau$ wird im Fit mit sehr geringen Fehlern geschätzt.
