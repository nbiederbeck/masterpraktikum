\section{Durchführung}%
\label{sec:durchführung}
Mit Hilfe des Operationsverstärkers sollen verschiedene Schaltungen gesteckt
und vermessen werden.
Dafür wird der Widerstand und die Kapazität der verwendeten Bauteile
vermessen.

\subsection{Linearverstärker}%
\label{sub:linearverstaerker}
Zuerst wird der Linearverstärker auf dem Steckbrett aufgebaut.
Es wird der Verstärkungsfaktor in Abhängigkeit der Frequenz eines
Linearverstärkers gemessen.
Desweiteren wird die Phasenverschiebung des Eingangssignals zum Ausgangssignals
in Abhängigkeit der Frequenz aufgenommen.
Dies wird für 4 verschiedene Widerstandsverhältnisse wiederholt.

\subsection{Umkehrintegrator/ -differenziator}%
\label{sub:umkehrintegrator_differenziator}
Die entsprechende Schaltung wird sowohl für den Umkehrintegrator als auch den
Differenziator aufgebaut. 
An ihr wird die Ausgangsspannung in Abhängigkeit der Frequenz gemessen und ein
Oszilloskop-Bild von der Integration/ Differenziation einer Rechteck, Dreieck
und Sinusspannung aufgenommen.

\subsection{Schmitt-Trigger}%
\label{sub:schmitt_trigger}
Es wird die Schaltung des Schmitt-Triggers aufgebaut und an ihr bei bekannten
Widerständen die Spannung gemessen bei welcher der Schmitt-Trigger kippt.
Dies geschieht indem ein Oszilloskop-Bild aufgenommen wird.

\subsection{Dreieckgenerator}%
\label{sub:dreieckgenerator}
Nachdem der Dreieckgenerator aufgebaut ist wird die Amplitude der
Spannung sowie dessen Frequenz gemessen.

\subsection{Gedämpfte Schwingung}%
\label{sub:gedaempfte_schwingung}
Fuer die vorgegebenen Bauteile wird der gedämpfte Schwingkreis aufgebaut und
bei kleinen Anregungsfrequenzen ein Trajektorie des Schwingkreis aufgenommen.
Die Aufnahme wird einerseits für den exponentiell gedämpfte wie verstärkten
Schwingkreis gemacht.
