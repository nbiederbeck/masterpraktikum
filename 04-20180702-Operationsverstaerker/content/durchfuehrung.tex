\section{Durchführung}%
\label{sec:durchführung}
Mit Hilfe des Operationsverstaerkers sollen verschiedene Schaltungen gesteckt
und vermessen werden.
Dafuer wird der Widerstand und die Kapazitaet der verwendeten bauteile
vermessen.

\subsection{Linearverstaerker}%
\label{sub:linearverstaerker}
Zuerst wird der Linearverstaerker auf dem Steckbrett aufgebaut.
Es wird der Verstaerkungsfaktor in abhaengigkeit der Frequenz eines
Linearversteckers gemessen.
Desweiteren wird die Phasenverschiebung des Eingangssignals zum Ausgangssignals
in Abhaengigkeit der Frequenz aufgenommen.
Dies wird fuer 4 verschiedene Widerstandsverhaeltnisse wiederholt.

\subsection{Umkehrintegrator/ -differenziator}%
\label{sub:umkehrintegrator_differenziator}
Die entsprechende Schaltung wird sowohl fuer den Umkehrintegrator als auch den
Differenziator aufgebaut. 
An ihr wird die Ausgangsspannung in Abhaengigkeit der Frequenz gemessen und ein
Oszilloskop-Bild von der Integration/ Differenziation einer Rechteck, Dreieck
und Sinusspannung aufgenommen.

\subsection{Schmitt-Trigger}%
\label{sub:schmitt_trigger}
Es wird die Schaltung des Schmitt-Triggers aufgebaut und an ihr bei bekannten
wiederstaenden die Spannung gemessen bei welcher der Schmitt-Trigger kippt.
Dies geschieht indem ein Oszilloskopbild aufgenommen wird.

\subsection{Dreieckgenerator}%
\label{sub:dreieckgenerator}
Nachdem der Dreieckgenerator aufgebaut ist wird die Amplitude der
Spannung sowie dessen Frequenz gemessen.

\subsection{Gedaempfte Schwingung}%
\label{sub:gedaempfte_schwingung}
Fuer die vorgegebenen Bauteile wird der gedaempfte Schwingkreis aufgebaut und
bei kleinen Anregungsfrequenzen ein Trajektorie des Schwingkreis aufgenommen.
Die Aufnahme wird einerseits fuer den Exponentiel gedaempfte wie verstaerkten
Schwingkreis gemacht.
