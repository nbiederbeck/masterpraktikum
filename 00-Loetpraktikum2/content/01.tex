\section{Aufgabe (Farbcodierungen)}%
\label{sec:aufgabe_1}

\textbf{Task:}
Sie finden auf einem Widerstand den Farbcode 'gelb, violett, rot, silber'. 
Um was für einen Widerstand hadelt es sich und welchen Wert hat der Widerstand?

\textbf{Solution:}
Widerstände in runder Bauform für elektronische Schaltungen werden 
oder können oft nicht mit Ziffern bedruckt werden. 
Um ihre Werte zu kennzeichnen, werden Farbcodierungen verwendet. 
Es geben umlaufende farbige Ringe den Widerstandswert und die Toleranzklasse 
an. 

Es gibt Farbcodes mit drei, vier, fünf oder sechs Ringen. Bei drei oder vier
Ringen geben die ersten beiden Ringe einen zweistelligen Wert von 10 $\Omega$ 
bis 99 $\Omega$ an und der dritte Ring gibt einen Multiplikator an.
Der vierte Ring, falls vorhanden, gibt die Toleranzklasse an. 
Fehlt er, ist die Toleranz ±20 \%.
Bei fünf oder sechs Ringen geben die ersten drei Ringe den Wert an (100 bis 999
$\Omega$), der vierte Ring ist der Multiplikator und der fünfte Ring die
Toleranzklasse. Ist ein sechster Ring vorhanden, gibt er den
Temperaturkoeffizienten (Stabilität) an.

Die Ableserichtung wird auf zwei verschiedene Weisen gekennzeichnet: 
\begin{itemize}
    \item der erste Ring vom Rand des Widerstandskörpers einen kleineren 
        Abstand als der letzte Ring
    \item der letzte Ring ist räumlich abgesetzt
\end{itemize}
Prüfung: 
Die andere Leserichtung ergibt keinen Wert der zugehörigen E-Reihe 
oder lässt sich gar nicht entschlüsseln (z. B. letzter Ring ist silber 
oder gold, was für den ersten Ring nicht zulässig ist). 

\begin{table}
    \centering
    \caption{Farbkodierung von Widerständen}
    \label{tab:label}
    \begin{tabular}{l c c c}
        \toprule
        Farbe & Wert & Multiplikator & Toleranz \\
        \midrule
        keine   & ---   & ---       & $\pm$ 20 \% \\
        silber  & ---   & $10^{-2}$ & $\pm$ 10 \% \\
        gold    & ---   & $10^{-1}$ & $\pm$ 5 \% \\
        schwarz & 0     & $10^{0}$  &  --- \\
        braun   & 1     & $10^{1}$  &  $\pm$ 1 \% \\
        rot     & 2     & $10^{2}$  &  $\pm$ 2 \% \\
        orange  & 3     & $10^{3}$  &  --- \\
        gelb    & 4     & $10^{4}$  &  --- \\
        grün    & 5     & $10^{5}$  &  $\pm$ 0.5 \% \\
        blau    & 6     & $10^{6}$  &  $\pm$ 0.25 \% \\
        violett & 7     & $10^{7}$  &  $\pm$ 0.1 \% \\
        grau    & 8     & $10^{8}$  &  $\pm$ 0.05 \% \\
        weiß    & 9     & $10^{9}$  &  --- \\
        \bottomrule
    \end{tabular}
\end{table}
