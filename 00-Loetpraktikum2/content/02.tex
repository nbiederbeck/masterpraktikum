\section{Aufgabe (Wahrheit/Unwahrheit)}%
\label{sec:aufgabe_1}

2. Welche Aussage ist richtig?

\begin{itemize}
    \item bei hoch dotierten Halbleitern ist die Sperrschicht sehr klein
        \begin{itemize}
            \item 
        \end{itemize}
    \item das längere Bein einer Diode ist die Kathode 
        \begin{itemize}
            \item \textbf{falsch} Die Anode der Leuchtdiode,
                die durch einen längeren Anschlussdraht gekennzeichnet ist, 
                muss mit dem Pluspol 
                und die Kathode mit dem Minuspol der Stromquelle 
                verbunden sein. 
        \end{itemize}
    \item die technische Stromrichtung zeigt von plus nach minus
        \begin{itemize}
            \item \textbf{richtig} „technischen Stromrichtung“ ist in erster 
                Linie historisch bedingt. 
                er geht von einem Strom von Ladungen aus, 
                die sich der Feldlinienrichtung des elektrischen Feldes folgend
                vom positiven zum negativen Spannungspol bewegen. 
                Dass es dagegen in metallischen Leitern die Elektronen sind, 
                die als Ladungsträger den Stromfluss bewirken 
                und dabei genau umgekehrt vom negativen zum positiven Pol 
                fließen, war zur Zeit dieser Begriffsbildung noch unbekannt.
        \end{itemize}
    \item Zenerdioden werden im Durchlaßbereich in Sperrichtung betrieben
        \begin{itemize}
            \item 
        \end{itemize}
    \item das Dreieckige Symbol mit zwei einfallenden Pfeilen ist das Symbol für
        eine Leuchtdiode
        \begin{itemize}
            \item \textbf{falsch} dies soll eine Photodiode darstellen, welche
                aus Licht einen Photostrom produziert. 
                Die Pfeile in die entgegengesetzte Richtung symbolisieren den
                Prozess bei dem aus Elektrischen Strom licht produziert
                wird (Leuchtdiode).
        \end{itemize}
    \item alle Dioden haben eine Sperrwirkung
        \begin{itemize}
            \item \textbf{falsch} beispielsweise die Tunneldiode bei der ein
                hochdotiertes n-leitendes Germanium-Plättchen in das eine 
                ebenfalls hochdotierte Indium-Pille einlegiert ist Sperrt 
                nicht sondern regt Beispielsweise LC-Schwingkreise zum 
                Schwingen an.
        \end{itemize}
    \item Dioden ändern ihren Innenwiderstand bei Temperaturänderung
    \begin{itemize}
        \item \textbf{richtig} bei einer höheren Temperatur stoßen die 
            Ladungsträger öfter zusammen und werden somit unbeweglicher. 
            Doch gerade durch die höhere Temperatur werden weitere 
            Ladungsträger aus dem Halbleitermaterial frei, 
            was zur Erhöhung der Leitfähigkeit führt. 
            Die Eigenleitung des Halbleiters steigt. 
            Das führt zu einem größeren Sperrstrom.
    \end{itemize}
\end{itemize}
