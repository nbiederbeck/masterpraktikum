% vim: spelllang=en
\section{Theory}\label{theory}

A diode laser is a cheap and tiny alternative to dye-lasers, therefore it is often used at student
labs.
In this report the absorption spectrum of rubidium (Rb) is studied, to gain knowledge of using
a diode laser.

\subsection{Introduction}\label{introduction}

A laser consists of three components:

\begin{enumerate}
  \item pump source
  \item optical medium
  \item cavity / resonator
\end{enumerate}

When an excited electron relaxes, it is able to emit a photon.
To excite electrons of the optical medium into states of higher energy, a pump source is used.
This could be another laser, biological effects, or, as it is in this case, a current.

Now with the electrons of the optical medium in the excited states, there are four(five?) possible
effects, grouped in two(three?) categories.
The first group is the non-radiative relaxation, where the energy of the electrons is transferred
to the medium lattice, the so-called phonons.
The second group is radiative relaxation, with spontaneous and induced emission.
Spontaneous emission is the effect, where an electron spontaneously relaxes into a lower energy
state, emitting a photon.

The most important effect for lasers is induced emission.
The energy of an incoming photon induces the emission of another photon.
The two photons are coherent, which means they have the same phase.

The cavity is used as a resonator in which the photons travel many times through the optical medium
to induce emission of coherent photons.

\subsection{Diodes}\label{diodes}

A diode is a p-n-dotted semiconductor.
In the n-plane electrons are added, that the plane has negative charge.
On the other hand ions are added to the p-plane so that it has positive charge.
These so-called electron-hole pairs build an electric field between the planes.
If current is applied either the current flows through or it is blocked due to the depleted region
in between the planes (see figure~\ref{fig:depletion_region}).
% \includegraphics{}

A light emitting diode (LED) % when does it emit light?

\subsection{Three state laser}\label{three-state-laser}

According to the Boltzmann distribution the ground state is always most populated in thermal
equilibrium.
At least three states are needed for a laser to lase: The pump source excites the electrons from
the ground state up to the highest exited state.
The lifetime \(\tau_1\) of electrons in this energy state is very short, the electrons relax
spontaneously two the second but lower excited state, where they have a much longer lifetime
\(\tau_2 \gg \tau_1\).
In the second excited state electrons can be kicked out by incoming photons, emitting coherent
radiation (induced emission).
For an example the energy states of a four state laser are shown in
figure~\ref{fig:four_state_laser}.
% \includegraphics{}

\subsection{Tuning of the diode laser}\label{tuning-of-the-diode-laser}

There are many options to alter the lasing function of the diode used in this experiment.
Each option alters different attributes of the laser, which will be elaborated in the following.

\subsubsection{Medium gain}\label{medium-gain}

\subsubsection{internal cavity}\label{internal-cavity}

\subsubsection{grating feedback}\label{grating-feedback}

\subsubsection{external cavity}\label{external-cavity}
