\subsection{Superstar}%
\label{sub:superstar}

Da MAGIC ein Stereoteleskop ist,
müssen die Daten beider Teleskope gemeinsam analysiert werden.
Superstar wird verwendet,
um die Image Parameter der gereinigten Daten beider
Teleskope zu einer Stereo-Parameter Datei zusammenzuführen.
Dabei werden weitere Parameter bestimmt,
die nur aus den Informationen beider Teleskope gewonnen werden können.

\paragraph{Theorie}%

\begin{wrapfigure}[12]{O}{0.4\textwidth}
  \centering
  \includegraphics[width=0.9\linewidth]{tikz/build/theta2.pdf}
  \caption{Stereoparameter eines Schauers.}%
  \label{fig:reco}
\end{wrapfigure}

Da die Quellposition in verschiedene Bereichen der beiden Kamerasensoren
liegt,
weise die Ellipsen unterschiedliche Orientierung auf.
Aus dem Schnittpunkt zweier Geraden,
die durch die Hauptachsen der Ellipsen gehen,
wird die rekonstruierte Quellposition bestimmt.

Auf die ähnliche Art wird über eine Koordinatentrafo vom Kamerasystem ins
Erdsystem der Aufprallpunkt des Schauers auf dem Erdboden
bestimmt.

% \textit{Height of shower max},
% \textit{Cherenkov radius},
% \textit{Cherenkov density},

Außerdem werden die Einzelergebnisse der beiden Teleskope verglichen,
um weitere Parameter zu bestimmen.
Für die \textit{Energy discrepancy} wird
ein $\chi^2$-Test durchgeführt,
wobei angenommen wird,
dass die wahre Energie in beiden Teleskopen identisch ist.
Die Richtungsrekonstruktionen beider Teleskope werden verglichen,
und die Differenz in \textit{DISP difference} gespeichert.
