\subsection{Superstar}%
\label{sub:superstar}

Da MAGIC ein Stereoteleskopsystem ist,
können die Daten beider Teleskope gemeinsam analysiert werden,
um eine deutlich bessere Rekonstruktion zu erhalten.
% müssen die Daten beider Teleskope gemeinsam analysiert werden.
Superstar wird verwendet,
um die Image Parameter der gereinigten Daten beider
Teleskope zu einer Stereo-Parameter Datei zusammenzuführen.
Dabei werden weitere Parameter bestimmt,
die nur aus den Informationen beider Teleskope gewonnen werden können.

\paragraph{Theorie}%

\begin{wrapfigure}[14]{O}{0.4\textwidth}
  \centering
  \includegraphics[width=0.9\linewidth]{tikz/build/theta2.pdf}
  \caption{Stereoparameter eines Schauers.}%
  \label{fig:reco}
\end{wrapfigure}

Da die Teleskope mit einem Abstand von einigen \SI{10}{\meter} zueinander
stehen,
weisen die Ellipsen in beiden Kamerabildern eine unterschiedliche Position und
Orientierung auf.
% Da die Quellposition in verschiedenen Bereichen der beiden Kamerasensoren
% liegt,
% weisen die Ellipsen unterschiedliche Orientierungen auf.
Aus dem Schnittpunkt der Geraden,
die durch die Hauptachsen der Ellipsen gehen,
wird eine vorläufige Quellposition bestimmt.

Auf eine ähnliche Art wird über eine Koordinatentrafo vom Kamerasystem ins
Erdsystem der Aufprallpunkt des Schauers auf dem Erdboden
bestimmt.

% \textit{Height of shower max},
% \textit{Cherenkov radius},
% \textit{Cherenkov density},

Außerdem werden die Einzelergebnisse der beiden Teleskope verglichen,
um weitere Parameter zu bestimmen.
Für die \textit{Energy discrepancy} wird
ein $\chi^2$-Test durchgeführt,
wobei angenommen wird,
dass die wahre Energie in beiden Teleskopen identisch ist.
Die Richtungsrekonstruktionen beider Teleskope werden verglichen,
und die Differenz in \textit{DISP difference} gespeichert.
