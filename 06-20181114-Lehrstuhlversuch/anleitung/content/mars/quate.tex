\subsection{Quate}%
\label{sub:quate}

Quate ist ein Programm,
das Mittelwerte von einem Parametersatz ausrechnet,
und Daten selektiert und klassifiziert.
Durch das Aussortieren kann die Analyse vereinfacht werden da so die Monte Carlo
Simmulation von unwahrscheinlichen ereignissen erspart bleibt. 

(Wir sollten noch erwAhnen das wir Crab observieren damit der Richtige zenit
winkel rausgesucht werden kann.)

\paragraph{Durchführung}%
Das Programm \texttt{quate} wird über die 
\texttt{quate.rc}-File bedient.
In diesem werden Einstellungen getroffen, 
sodass die Daten sinnvoll prozessiert werden.
Dazu ist der Azimut und Zenit entsprechend der zu 
observierenden Quelle einzustellen. 
Dies ist über die Parameter \texttt{AzMin},
\texttt{AzMax}, \texttt{ZdMin} und \texttt{ZdMin}
möglich.
ZusAtzlich ist der Parameter 
\texttt{MinAerosolTrans9km = 0.9} anzupassen, 
damit ... berücktsichtigt wird.
Überlegen Sie sich bei welche Daten das prozessieren sinnvoll ist, 
\textit{(On/ Off/ MC)} und führen Sie dies durch.
Geben Sie dafür die entsprechenden Pfade zu den Daten an.
\begin{lstlisting}
	quate -b -s \
		--config=quate.rc \
		--out=OUT_PATH \ 
		--ind=IND_PATH 
\end{lstlisting}
Das Programm \texttt{coach} erstellt jeweils ein Ordner 
\texttt{bad} und \texttt{good},
wo die einzelnen Runs einsortiert werden. 
Dies ist für die folgenden Analyseschritte wichtig.
