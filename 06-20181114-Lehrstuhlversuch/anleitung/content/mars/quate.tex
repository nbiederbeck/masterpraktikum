\subsection{Quate}%
\label{sub:quate}

Quate ist ein Programm,
das Mittelwerte von einem Parametersatz ausrechnet,
und Daten selektiert und klassifiziert.
Durch das Aussortieren kann die Analyse vereinfacht werden da so die Monte Carlo
Simulation von unwahrscheinlichen Ereignissen erspart bleibt.

{\color{red}(Wir sollten noch erwähnen das wir Crab observieren damit der richtige Zenitwinkel rausgesucht werden kann.)}

\paragraph{Durchführung}%
Das Programm \texttt{quate} wird über die
Datei \texttt{quate.rc} konfiguriert.
In dieser werden Einstellungen getroffen,
sodass die Daten sinnvoll prozessiert werden.
Dazu ist der Azimut und Zenit entsprechend der zu
observierenden Quelle einzustellen.
Dies ist über die Parameter \texttt{AzMin},
\texttt{AzMax}, \texttt{ZdMin} und \texttt{ZdMin}
möglich.
Zusätzlich ist der Parameter
\texttt{MinAerosolTrans9km = 0.8} anzupassen,
damit {\color{red}\ldots} berücksichtigt wird.
Überlegen Sie sich,
bei welche Daten das Selektieren sinnvoll ist
(On/Off/MC)
und führen Sie dieses durch.
Geben Sie dafür die entsprechenden Pfade zu den Daten an.

\begin{lstlisting}
  quate -b -s -q \
    --stereo \
    --config=quate.rc \
    --out=<OUTPUT-DIRECTORY> \
    --ind=<INPUT-DIRECTORY>
\end{lstlisting}

Quate erstellt im \texttt{<INPUT-DIRECTORY>}
jeweils ein Ordner \texttt{bad} und \texttt{good},
in denen symbolische Links zu den jeweils ausgewählten Dateien liegen.
Dies ist für die folgenden Analyseschritte wichtig.
