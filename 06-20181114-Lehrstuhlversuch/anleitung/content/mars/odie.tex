\subsection{Odie}%
\label{sub:odie}
Es wird ein Maß benötigt, um entscheiden zu können,
ob die Ursache der gemessen Events die Quelle selber ist.
Das Wahrscheinlichkeitsmaß wird gewöhnlicherweise in Form von Gaußschen
Standardabweichungen $\sigma$ angegeben.
Odie berechnet die Siginifikanz der Daten an einer entsprechenden Quellposition.
Ab einer Signifikanz \SI{3}{\sigma} wird von einem \textit{Hinweis}
und ab \SI{5}{\sigma} von einer \textit{Detektion} gesprochen.

\paragraph{Theorie}%

Die Wahrscheinlichkeit wird angegeben, indem geprüft wird,
ob die über\-schüssigen Ereignisse auf eine zufällige
Fluktaktion des Untergrunds zurück\-zu\-führen sind.
Dies geschieht mit einem Likelihood\-/Quotienten\-/Test.

Die Herleitung ist im Einzelnen für die Durchführung des
Versuches nicht relevant,
dient aber zum vollen Verständnis des Versuchs.

Zur Bestimmung des Photonenflusses der Quelle wird für die Zeit $t_\text{on}$ in die Region der
erwarteten Quelle geschaut und die Anzahl an Ereignissen $N_\text{on}$ gezählt.
Zur Abschätzung des Untergrundes werden für die Zeit
\begin{equation*}
    t_\text{off} = \frac{t_\text{on}}{\alpha}
\end{equation*}
$N_\text{off}$
Events in einer Region,
in welcher keine Gammaquelle erwartet wird,
% wo keine Quelle erwartet wird,
gemessen.
Da Gamma Ray Ereignisse (und Untergrundereignisse)
unabhängig voneinander sind,
sind diese poissonverteilt.
Da das Verhältnis von Signal $N_\text{S}$ zu Untergrund
$N_\text{B}$ sehr klein ist
und der Untergrund statistischen Schwankungen unterliegt,
ist mit
\begin{equation}
	\hat{N}_\text{S} = N_\text{on} - \hat{N}_\text{B} = N_\text{on} - \alpha N_\text{off}
\end{equation}
die Signifikanz
\begin{equation}
	S = \frac{\hat{N}_\text{S}}{\sqrt{\hat{N}_\text{S}}},
\end{equation}
nur grob abgeschätzt.
Statistische Fluktuationen des Untergrundes werden bei dieser Abschätzung der
Signifikanz stark gewichtet.
% da es sich um statistisches Rauschen des
% Hintergrunds handeln kann.

Eine statistisch stabile Methode, die Signifikanz zu bestimmen, stellt die
Likelihood Ratio Methode nach Li und Ma da.
Es wird angenommen, dass die erwartete Anzahl an Signal
Photonen $\langle N_\text{on} \rangle$ und die Anzahl an Untergrundereignissen
$\langle N_\text{off} \rangle$ nicht bekannt sind.
Die zu prüfende Nullhypothese ist:
\begin{quote}
	Keine Quelle existiert und
    alle gemessenen Events $N_\text{on}$ gehören zum Untergrund.
\end{quote}
Dies ist äquivalent zu $\langle N_\text{S} \rangle=0$.

Zur Bestimmung der Liklihoodfunktion der Nullhypothese,
verschwinden die
Ereignisse aus der Quelle und die Untergrundereignisse entsprechen der
gewichteten Summe aus der On und Off Region:
\begin{equation}
	L(X|E_0, \hat{T}_\text{c})= P_\text{r} \left[
		N_\text{on}, N_\text{off} |
		\langle N_\text{S} \rangle = 0,
		\langle N_\text{B} \rangle = \frac{\alpha}{1 + \alpha} (N_\text{on} +
			N_\text{off})
	\right]
\end{equation}
Die maximale Likelihood ergibt sich,
indem die Gegenhypothese,
dass die On-Region um die Untergrundereignisse bereinigt ist,
getestet wird:
\begin{equation}
	L(X|\hat{E}, \hat{T})= P_\text{r} \left[
		N_\text{on}, N_\text{off} |
		\langle N_\text{S} \rangle = N_\text{on} - \alpha N_\text{off},
		\langle N_\text{B} \rangle = \alpha N_\text{off}
	\right]
\end{equation}
Durch Einsetzen in den Likelihood Quotienten Test
\begin{equation}
	\lambda = \frac{L(X|E_0, \hat{T}_\text{c})}{L(X|\hat{E}, \hat{T})}
\end{equation}
und die Annahme, dass genügend Ereignisse detektiert wurden, sodass diese einer
$\chi^2$-Verteilung mit einem Freiheitsgrad folgen,
\begin{equation}
	\sqrt{- 2 \ln \lambda} = \chi(1),
\end{equation}
ergibt sich die Li und Ma Signifikanz
\begin{equation}
  S = \sqrt{2} {\left(
      N_\text{on} \log \left[
        \left( \frac{1 + \alpha}{\alpha} \right) \left(
          \frac{N_\text{on}}{N_\text{on} + N_\text{off}}
        \right)
      \right]
      + N_\text{off} \log \left[
        (\alpha + 1) \left(
          \frac{N_\text{off}}{N_\text{on} + N_\text{off}}
        \right)
      \right]
  \right)} ^ {1/2}.
\end{equation}

Die Signifikanz kann sowohl auf dem gesamten Datensatz,
als auch auf dem vom Untergrund bereinigten Datensatz bestimmt werden. %nicht-separierten als auch Gamma/Hadron-separierten Daten
% bestimmt werden.
Sie hängt neben dem Verhältnis von $N_\text{on}$ zu $N_\text{off}$ auch
maßgeblich von der Größe von $N_\text{on}$ ab.
Eine gute Gamma/Hadron Separation hält den Untergrund gering,
das Signal jedoch hoch
und maximiert somit die Signifikanz.

Der Parameter $\theta$ gibt den Abstand zwischen
angenommener und rekonstruierter Quellposition
für jedes Event an.
Durch einen Schnitt in $\theta$ werden Events verworfen,
welche einen zu großen Abstand von der angenommenen Quelle aufweisen.
Hingegen können diffuse Untergrundereignisse jeden möglichen Abstand zur
angenommenen Signalquelle haben,
da sie den Himmelsausschnitt homogen ausfüllen.
Ziel ist es, einen Schnitt in $\theta^2$ (\texttt{theta2}) so zu setzen,
dass die Signifikanz maximal wird.
Ein beispielhafter Schnitt in Theta ist in
Abbildung~\ref{fig:thetacut} zu sehen.
% Er kann zum Beispiel durch Intervallschachtelung bestimmt werden.

\begin{wrapfigure}[12]{L}{0.45\textwidth}
		\centering
		\includegraphics[width=\linewidth]{build/theta2.pdf}
		\caption{Theta2 Schnitt auf Daten zur Maximierung der Signifikanz.}%
		\label{fig:thetacut}
\end{wrapfigure}

\paragraph{Durchführung}%

% Die Konfigurationsdatei \texttt{odie.rc} wird verwendet.
Es wird die Konfigurationsdatei \texttt{odie.rc} verwendet.
Es muss das Feld \texttt{Odie.dataName} angepasst werden.
Es gilt wieder, mit Wildcards die von Melibea prozessierten Daten zu nutzen
(Bsp.: \texttt{*\_Q\_*.root}).

\begin{lstlisting}
	odie -q -b --config=odie.rc
\end{lstlisting}

\textit{Nach dem ausführen von Odie werden $\theta^2$-Cuts und die PSF vorgeschlagen.
Es gilt die Accuracy zu minimieren,
und eine moeglichst realistische Beschreibung der Quellausdehnung (PSF) zu finden.
Dazu muss der Parameter \texttt{cuts} iterativ angepasst und das Programm Odie
mehrmals durchgeführt werden.
}
