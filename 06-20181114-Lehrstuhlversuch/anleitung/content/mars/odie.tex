\subsection{Odie}%
\label{sub:odie}
In der Gamma Ray Astronomy werden die Teleskope in der Regel auf einen Punkt
ausgerichtet, an dem eine Quelle angenommen wird.
Die gemessenen Events, die das Teleskop triggert, können vom
kosmischen Hintergrund oder von einer echten Quelle, falls diese existiert, sein.
Es wird ein Maß benötigt, um entscheiden zu können,
ob die Ursache der gemessen Events die Quelle selber ist.
Das Wahrscheinlichkeitsmaß wird gewöhnlicherweise in Form von Gaußschen
Standardabweichungen $\sigma$ angegeben.
Kosmische Quellen können also
durch das Suchen von Regionen mit hoher Signifikanz (siehe
Abbildung~\ref{fig:skymap})
identifiziert werden.
Desweiteren können die Sensitivitäten und Analysen von verschiedenen
Teleskopen untereinander,
durch die auf die Zeit normierte Signifikanz,
verglichen werden.

\paragraph{Theorie}%

Die Wahrscheinlichkeit wird angegeben, indem geprüft wird,
ob die (zum kosmischen Hintergrund) über\-schüssigen
Ereignisse   auf eine zufällige Fluktaktion zur
Hintergrundstrahlung zurückzuführen sind.
Dies geschieht mit einem Likelihood\-/Quotienten\-/Test.

Die Herleitung ist im Einzelnen für die Durchführung des
Versuches nicht relevant,
dient aber zum vollen Verständnis des Versuchs.

Zur Bestimmung des Photonenflusses der Quelle wird für die Zeit $t_\text{on}$ in die Region der
erwarteten Quelle geschaut und die Anzahl an Photonen $N_\text{on}$ gezählt.
Zur Abschätzung des Untergrundes werden für die Zeit $t_\text{off}$
(äquivalent zu $t_\text{on} / \alpha$)
die Events des kosmischen Hintergrunds $N_\text{off}$ gemessen.
Da das Verhältnis von Signal $N_\text{S}$ zu Untergrund
$N_\text{B}$ sehr klein ist
und der Hintergrund statistischen Schwankungen unterliegt,
ist
\begin{equation}
	N_\text{S} = N_\text{on} - \hat{N}_\text{B} = N_\text{on} - \alpha N_\text{off}
\end{equation}
nur eine schlechte Abschätzung der Signifikanz
\begin{equation}
	S = \frac{N_\text{S}}{\sqrt{N_\text{S}}},
\end{equation}
da es sich um statistisches Rauschen des
Hintergrunds handeln kann.

Eine statistisch stabile Methode, die Signifikanz zu bestimmen, stellt die
Likelihood Ratio Methode nach Li und Ma da.
Es wird angenommen, dass die erwartete Anzahl an Signal
Photonen $\langle N_\text{on} \rangle$ und die Anzahl Hintergrund Photonen
$\langle N_\text{off} \rangle$ nicht bekannt sind.
Die zu prüfende Nullhypothese ist:
\begin{quote}
	Keine Quelle existiert und
	alle gemessenen Events gehören zum kosmischen Hintergrund.
\end{quote}
Dies ist äquivalent zu $\langle N_\text{on} \rangle=0$.
Da Gamma Ray Ereignisse alle unabhängig voneinander sind,
sind diese poissonverteilt.

Zur Bestimmung der Liklihoodfunktion der Nullhypothese,
verschwinden die
Ereignisse aus der Quelle und die Anzahl an Hintergrund Photonen addiert sich
aus der On und Off Region:
\begin{equation}
	L(X|E_0, \hat{T}_\text{c})= P_\text{r} \left[
		N_\text{on}, N_\text{off} |
		\langle N_\text{S} \rangle = 0,
		\langle N_\text{B} \rangle = \frac{\alpha}{1 + \alpha} (N_\text{on} +
			N_\text{off})
	\right]
\end{equation}
Die maximale Likelihood ergibt sich, indem die On-Region
um die Events der Hintergrundstrahlung bereinigt wird:
\begin{equation}
	L(X|\hat{E}, \hat{T})= P_\text{r} \left[
		N_\text{on}, N_\text{off} |
		\langle N_\text{S} \rangle = N_\text{on} - \alpha N_\text{off},
		\langle N_\text{B} \rangle = \alpha N_\text{off}
	\right]
\end{equation}
Durch Einsetzen in den Likelihood Quotienten Test
\begin{equation}
	\lambda = \frac{L(X|E_0, \hat{T}_\text{c})}{L(X|\hat{E}, \hat{T})}
\end{equation}
und die Annahme, dass genügend Ereignisse detektiert wurden, sodass diese einer
$\chi^2$-Verteilung mit einem Freiheitsgrad folgen,
\begin{equation}
	\sqrt{- 2 \ln \lambda} = \chi(1),
\end{equation}
ergibt sich die Li und Ma Signifikanz
\begin{equation}
  S = \sqrt{2} {\left(
      N_\text{on} \log \left[
        (\tau + 1) \left(
          \frac{N_\text{on}}{N_\text{on} + N_\text{off}}
        \right)
      \right]
      + N_\text{off} \log \left[
        \left( \frac{1 + \tau}{\tau} \right) \left(
          \frac{N_\text{off}}{N_\text{on} + N_\text{off}}
        \right)
      \right]
  \right)} ^ {1/2}.
\end{equation}

\begin{wrapfigure}[13]{L}{0.45\textwidth}
		\centering
		\includegraphics[width=\linewidth]{build/theta2.pdf}
		\caption{Theta2 Schnitt auf Daten zur Maximierung der Signifikanz.}%
		\label{fig:thetacut}
\end{wrapfigure}

Die Signifikanz kann sowohl auf Rohdaten als auch Gamma/Hadron-separierten Daten
bestimmt werden.
Sie hängt neben dem Verhältnis von $N_\text{on}$ zu $N_\text{off}$ auch
maßgeblich von dem Umfang an $N_\text{on}$ ab.
Eine gute Gamma/Hadron Separation hält den Untergrund gering und maximiert die
Signifikanz.

Der Parameter $\theta$ gibt den Abstand zwischen
angenommener und rekonstruierter Quellposition
für jedes Event an.
Durch einen Schnitt in $\theta$ werden Events verworfen,
welche einen zu großen Abstand von der angenommenen Quelle aufweisen.
Die Richtungsrekonstruktion eines Signalereignisses weist
einen geringen Abstand zur angenommenen Quelle auf.
Hingegen können diffuse Untergrundereignisse jeden möglichen Abstand zur
angenommenen Signalquelle haben,
da sie den Detektor homogen ausfüllen.
Ziel ist es, einen Schnitt in $\theta^2$ (\texttt{theta2}) so zu setzen,
dass die Signifikanz maximal wird.
Ein beispielhafter Schnitt in Theta ist in
Abbildung~\ref{fig:thetacut} zu sehen.
% Er kann zum Beispiel durch Intervallschachtelung bestimmt werden.


\paragraph{Durchführung}%

Die Konfigurations\-datei \texttt{odie.rc} wird verwendet.
Es muss das Feld \texttt{Odie.dataName} angepasst werden.
Es gilt wieder, mit Wildcards die von Melibea prozessierten Daten zu nutzen
(Bsp.: \texttt{*\_Q\_*.root}).

\begin{lstlisting}
	odie -q -b --config=odie.rc
\end{lstlisting}
