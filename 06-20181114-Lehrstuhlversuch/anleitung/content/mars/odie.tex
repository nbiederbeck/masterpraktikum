\subsection{Odie}%
\label{sub:odie}
In der gra werden die Teleskope in der Regel auf einen Punkt 
ausgerichtet, an dem eine Quelle angenommen wird.
Die gemessenen Events die das Teleskop triggert, können vom 
kosmischen Hintergrund oder von einer echten Quelle, falls diese existiert seien.
Es wird ein Maß benötigt um entscheiden zu können, 
ob die Ursache der gemessen Evens die Quelle selber ist.
Das wahrscheinlichkeitsmaß wird gewöhnlicherweise in Form von Gaußschen
Standardabweichungen $\sigma$ angegeben. 
Kosmische Quellen können also, 
durch das suchen von Regionen mit hoher Signifikanz (siehe 
Abbildung \ref{fig:skymap}),
identifiziert werden.
Desweiteren kann die Sensivität und Analyse von verschiedenen 
Teleskopen untereinander,
durch die auf die Zeit normierte Signifikanz,
verglichen werden.

\paragraph{Theorie}%
Die Wahrscheinlichkeit wird angegeben in dem geprüft wird, ob die überschüssigen
Ereignisse (zum kosmischen Hintergrund)  auf eine zufällige Fluktaktion zur
Hintergrundstrahlung zurückzuführen sind.
Dies geschieht in einem Likelihood-Quotienten-Test.

Die Herleitung ist im einzelnen für die Durchführung des 
Versuches nicht relevant, 
dient aber zum vollen verständniss des Versuches.

Zur bestimmung des Photonflusses der Quelle wird für die Zeit $t_\text{on}$ in die Region der
erwateten Quelle geschaut und die Anzahl an Photonen $N_\text{on}$ gezählt.
Zur abschätzung des Untergrundes werden für die Zeit $t_\text{off}$ 
(äquivalent zu $t_\text{on} / \alpha$)
die Events der kosmischen Hintergrunds $N_\text{off}$ gemessen.
Da das Verhältniss von Signal $N_\text{S}$ zu Untergrund 
$N_\text{B}$ sehr klein ist 
und der Hintergrund statistichen Schwankungen unterliegt,
ist 
\begin{equation}
	N_\text{S} = N_\text{on} - \hat{N}_\text{B} = N_\text{on} - \alpha N_\text{off}
\end{equation}
nur eine schlechte Abgeschätzung, da es sich um statistisches Rauschen des
Hintergrund handeln kann.
Dazu wird die Signifikanz der Quelle bestimmt
\begin{equation}
	S = \frac{N_\text{S}}{\sqrt{N_\text{S}}} .
\end{equation}
Eine statistische Stabile Methode die Signifikanz zu bestimmen stellt die
Likelihood Ratio Methode nach Li and Ma da.
Es wird angenommen, dass die beiden Parameter die erwatete Anzahl an Signal
Photonen $\langle N_\text{on} \rangle$ und die Anzahl Hintergrund Photonen
$\langle N_\text{off} \rangle$ nicht bekannt sind. 
Die zu prüfende Nullhypothese ist: 
\begin{quote}
	Keine Quelle existiert und 
	alle gemessenen Events gehören zum kosmischen Hintergrund.
\end{quote}
Dies ist äquivalent zu $\langle N_\text{on} \rangle=0$.
Da gamma ray ereignisse alle unabhängig voneinander sind, sind diese Poisson
verteilt.

Zur Bestimmung der Liklihood funktion der Nullhypothese, verschwinden die
Ereignisse aus der Quelle und die Anzahl an hintergrund Strahlung addiert sich
aus der On und off Region. 
\begin{equation}
	L(X|E_0, \hat{T}_\text{c})= P_\text{r} \left[ 
		N_\text{on}, N_\text{off} |
		\langle N_\text{S} \rangle = 0,
		\langle N_\text{B} \rangle = \frac{\alpha}{1 + \alpha} (N_\text{on} +
			N_\text{off})
	\right]
\end{equation}
Die maximum Likelihood ergibt sich indem das On Signal,
um die Events der Hintergrundstrahlung bereinigt werden.
\begin{equation}
	L(X|\hat{E}, \hat{T})= P_\text{r} \left[ 
		N_\text{on}, N_\text{off} |
		\langle N_\text{S} \rangle = N_\text{on} - \alpha N_\text{off},
		\langle N_\text{B} \rangle = \alpha N_\text{off}
	\right]
\end{equation}
Durch Einsetzen in den Likelihood Quotienten Test
\begin{equation}
	\lambda = \frac{L(X|E_0, \hat{T}_\text{c})}{L(X|\hat{E}, \hat{T})}
\end{equation}
und der Annahme das genügend Ereignisse detektiert wurden, sodass diese einer
$\chi^2$ mit einem Freiheitsgrad folgen, 
\begin{equation}
	\sqrt{- 2 \ln \lambda} = \chi(1)
\end{equation}
ergibt sich die Li and Ma Signifikanz.
\begin{equation}
	S = \sqrt{2} \left(
		N_\text{on} \log \left[
			(\tau + 1) \left( 
				\frac{N_\text{on}}{N_\text{on} + N_\text{off}} 
			\right)
		\right]  
		+ N_\text{off} \log \left[
			\left( \frac{1 + \tau}{\tau} \right) \left( 
				\frac{N_\text{off}}{N_\text{on} + N_\text{off}} 
			\right)
		\right]  
	\right) ^ {1/2}
\end{equation}

\begin{wrapfigure}[10]{L}{0.45\textwidth}
		\centering
		\includegraphics[width=\linewidth]{build/theta2.pdf}
		\caption{Theta2 cut auf Daten zur Maximierung der Signifikanz.}
		\label{fig:thetacut}
\end{wrapfigure}
Die Signifikanz kann sowohl auf Gamma/Hadron seperierten als auch auf nicht
seperierten Daten bestimmt werden. 
Sie ist neben dem Verhältniss von $N_\text{on}$ zu $N_\text{off}$ auch
Maßgeblich von dem Umfang an $N_\text{on}$ ab.
Eine gute Gamma Hadron seperation hält den Untergrund gering und maximiert die
Signifikanz. 
Ziel ist es den theta cut so zu setzen das die Signifikanz maximal wird. 
Ein bespielhafter Thetacut ist in Abbildung \ref{fig:thetacut} zu sehen.
Er kann zum Beispiel durch Intervallschachtelung bestimmt werden.


\paragraph{Durchführung}%

Die Konfigurationsdatei \texttt{odie.rc} wird verwendet.
Es muss das Feld \texttt{Odie.dataName} angepasst werden.
Es gilt wieder, mit Wildcards die von Melibea prozessierten Daten zu nutzen
(Bsp.: \texttt{*\_Q\_*.root}).

\begin{lstlisting}
	odie -q -b --config=odie.rc
\end{lstlisting}
