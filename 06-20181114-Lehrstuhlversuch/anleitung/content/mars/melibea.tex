\subsection{Melibea}%
\label{sub:melibea}

\paragraph{Theorie}

Melibea wendet auf die von Superstar berechneten
Stereo-Parameter Dateien
die von Coach trainierten Modelle an.
Das Programm weist damit jedem Event
eine geschätzte Energie und Hadronness zu
und schätzt eine Quellposition (DISP).

% Auf diesen Werten können im weiteren Verlauf der Analyse
% Schnitte angewendet werden.
% Ein Schnitt auf der Hadronness $H_{\text{cut}}$
% weist Events mit einer Hadronnes $H_{\text{Evt}}~<~H_{\text{cut}}$
% die Klasse Signal zu.
% Events mit $H_{\text{Evt}} > H_{\text{cut}}$ werden der Klasse Untergrund
% zugeordnet.
% Hierbei ist das Ziel, einen möglichst reinen
% Datensatz (frei von Untergrund) zu erzeugen,
% unter der Einschränkung nicht zu viele Gammas zu verwerfen.
% Für eine Veranschaulichung des Schnittes siehe Abbildung~\ref{fig:uebersicht}.
% Für die weiteren Analyseschritte ist ein möglichst
% großer Datensatz von Vorteil.
% Die Hadronness muss so eingestellt werden,
% dass ein Kompromiss zwischen Reinheit und Größe des Datensatzes erreicht wird.


\paragraph{Durchführung}%

Für Stereo-Analysen muss die Konfigurationsdatei \texttt{melibea\_stereo.rc}
verwendet werden.
Es gibt für diese Analyse keine notwendigen Anpassungen in dieser Datei.
Melibea muss auf Daten und dem Testdatensatz der Monte Carlos einzeln angewendet werden.

\begin{lstlisting}
  # <PATH-TO> entspricht dem RF.outpath aus coach!
  melibea -q -b -f \
    --stereo \
    --config=melibea_stereo.rc \
    --rf \
    -erec \
    --calcstereodisp \
    --calc-disp-rf \
    --calc-disp2-rf \
    --disp-rf-sstrained \
    --rftree=<PATH-TO>/RF.root \
    --etab=<PATH-TO>/Energy_Table.root \
    --rfdisptree=<PATH-TO>/disp1/DispRF.root \
    --rfdisp2tree=<PATH-TO>/disp2/DispRF.root \
    --ind=<INPUT-DIRECTORY>/*.root \
    --out=<OUTPUT-DIRECTORY> \
    --log=<OUTPUT-DIRECTORY>/melibea.log
  # Ausserdem bei Monte Carlo
    -mc
    --ind=<INPUT-DIRECTORY>/*_wr_2.root \
\end{lstlisting}

Melibea erstellt für jede Eingangsdatei eine Outputdatei der
Form \texttt{*\_Q\_*} und die Datei \texttt{melibea.root}.
