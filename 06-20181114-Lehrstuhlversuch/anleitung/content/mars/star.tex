\subsection{Star}%
\label{sub:star}
Ziel von der Software Star ist die Bereinigung von Kamerabildern und die Berechnung
der Image Parameter auf den Daten.

\paragraph{Theorie}
Sobald die Teleskope ein Event triggern,
speichern sie die Lichtpulse in einem gewissen Zeitfenster von jedem PMT.\@
Diese Daten enthalten neben dem eigentlichen Tscherenkowlicht auch
Rauschen von elektronischen Bauteilen,
und Nachthimmelhintergrund z.B.\ durch den Mond.
% und andere Störungen.

\subparagraph{Image Cleaning}
Um sinn\-voll die Im\-age Pa\-ra\-met\-er bestimmen zu können,
müssen die Daten bereinigt werden.
Dazu werden alle Pixel verworfen,
die nicht eindeutig dem Teilchenschauer zugeordnet werden können.
% Dazu werden alle Pixel verworfen, in denen keine
% Tscherenkowphotonen gemessen werden.
% Des Weiteren ist gutes Image Cleaning in der Lage,
% % um einen konstanten Photostream zu bereinigen,
% niederenergetische Ereignisse vom Untergrund zu trennen und damit
% für einen geringen Energie-Schwellwert zu sorgen.
Die Tscherenkowschauer sind größer und energieärmer,
je niedriger die Energie des Primärphotons ist.
Somit ist es schwieriger, niederenergetische Schauer
vom Rauschen zu unterscheiden.
Ein gutes Image Cleaning ist essentiell für einen niedrigen
Energie-Schwellwert in der Analyse.

Das aktuell angewandte Image Cleaning heißt \textit{Sum Image Cleaning}.
Es besteht aus drei Schritten.
Zuerst werde die Pixel zu Clustern aus 2--4 nächsten Nachbarn zusammengefasst.
Es wird überprüft, ob die Summe an Photonen in einem Cluster
einen gewissen clustergrößenabhängigen Grenzwert übersteigt.
Ist dies der Fall, überstehen die Pixel das Cleaning.
Als zweites werden Schnitte auf der Ankunftszeit der so entstandenen
Pixel-Inseln gemacht.
Es werden die mittleren Ankunftszeiten jeder Insel bestimmt.
Weichen die Ankunftszeiten der Inseln zu stark von der Ankunftszeit der größten Insel ab,
werden die Pixel der Insel verworfen.
Pixel der übrigbleibenden Inseln werden im dritten Schritt gereinigt,
wobei
wieder überprüft wird, ob die Pixel einen gewissen Grenzwert überschreiten.
% zwei Grenzwerte $Q_{1} > Q_{2}$ existieren.
% Im ersten Schritt wird geprüft,
% ob die Pixel über dem
% Grenzwert $Q_{1}$ liegen,
% und ob sie einen Cluster mit mindestens zwei Pixeln bilden.
% Ist dies der Fall,
% werden zusätzlich angrenzende Pixel,
% die über dem Grenzwert $Q_{2}$ liegen,
% hinzugefügt.

Jeweils ein Beispiel für ein Kamerabild vor und nach dem Image Cleaning ist in
Abbildung~\ref{fig:cleaning} dargestellt.

\begin{figure}[ht]
  \centering
  \begin{subfigure}[c]{0.35\linewidth}
    \includegraphics[width=\linewidth]{pictures/uncleaned.png}
    \caption{Kamerabild.}%
    \label{fig:uncleaned}
  \end{subfigure}
	\hspace{1cm}
  \begin{subfigure}[c]{0.35\linewidth}
    \includegraphics[width=\linewidth]{pictures/cleaned.png}
    \caption{Bereinigtes Kamerabild.}%
    \label{fig:cleaned}
  \end{subfigure}
  \caption{Bereinigung der Kamerabilder zur Berechnung der Image Parameter.}%
  \label{fig:cleaning}
\end{figure}

\subparagraph{Image Parameter}%
\label{spar:image_parameter}

Tscherenkowschauer erzeugen in der Kamera ellipsenförmige Projektionen.
% Unter anderem anhand der Länge der Hauptachsen der Ellipse und deren Ausrichtung
% lässt sich die Energie des
% Primärteilchens bestimmen und die Richtung rekonstruieren.
Unter anderem anhand der Gesamtladung der Schauer 
oder der Länge der Hauptachsen der Ellipse 
und deren Ausrichtung lässt sich die Energie 
des Primärteilchens bestimmen und dessen Richtung rekonstruieren.
Dazu werden auf den gereinigten Kamerabildern die Image Parameter bestimmt.
Um die Rekonstruktion der Ereignisse zu optimieren, 
werden zu den historischen Hillas Parametern zusätzliche Parameter berechnet, 
die die Schauerbilder beschreiben.
% Die Image Parameter entsprechen teilweise den Hillas Parametern,
% und sind zweckmäßig erweitert.


\begin{wrapfigure}[19]{O}{0.6\textwidth}
  \centering
  \includegraphics[width=0.9\linewidth]{tikz/build/imgParam.pdf}
  \caption{Image Parameter eines Schauers.}%
  \label{fig:hillas}
\end{wrapfigure}

Die Image Parameter \textit{length} und \textit{width} geben die Hauptachsen
der Ellipse an.
Da hadronische Schauer höhere transversale Entwicklung zeigen, ist
\textit{width} dazu geeignet, sie von Gammaschauern zu trennen.
\textit{Size} beschreibt die gesamte Anzahl an Photoelektronen im Schauerbild.
Die Anzahl an Photoelektronen verhält sich proportional zur Energie des
Primärteilchens.
\textit{CoG} (Center of Gravity) beschreibt den Schauerschwerpunkt im Kamerabild.
\textit{Conc-n} ist der Anteil der Photoelektronen in den $n$ hellsten Pixeln,
und beschreibt somit die Kompaktheit des Schauers.
Da gammainduzierte Schauer sehr kompakt sind, ist dies auch eine gute
Trennvariable für Gamma/Hadron-Separation.
\textit{Leakage1/2} beschreibt den Anteil des Signals in den \textit{1/2} äußeren
Pixelringen.
Dieser Parameter gibt eine Abschätzung,
zu welchem Anteil der Schauer ausserhalb der Kamera liegt.
\textit{Alpha} beschreibt den Winkel zwischen Hauptachse der Ellipse und
der Linie vom CoG zur erwarteten Quellposition.
% Da gammainduzierte Schauer Richtung Quelle zeigen,
% ist es ebenfalls ein trennstarker Parameter.
\textit{Dist} ist die Entfernung vom CoG zur erwarteten Quellposition.
