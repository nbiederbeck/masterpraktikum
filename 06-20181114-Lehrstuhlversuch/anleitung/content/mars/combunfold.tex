\subsection{CombUnfold}%
\label{sub:combunfold}
Die Bestimmung der Energie von Gamma Schauern beruht auf den Look-Up Table
welche aus den Image Parametern erstellt wurden. 
Die Energie welche auf den Parametern geschätzt wird, beruht auf einer 
endlichen Auflösung.
Um einen Energieschätzer für die Quelle zu erhalten, 
wird das Entfaltungsverfahren verwendet.

\paragraph{Theorie}%
\label{par:theorie}
Da kein Chrenkov-Teleskop in der Laage ist die Energie 
des primärteilche zu messen,
sondern lediglich den Deponierten Photonstrom wird ein Entfaltungsalgorithmus
benötigt.
Dabei treten beim Messen hauptsächlich dreierlei Probleme auf.

Die \textit{limitierte Akzeptanz} beschreibt das Problem, 
dass nicht alle Events welche in der Erdatmosphäre über den Teleskop geschehen,
detektiert werden können. 
Die detektionswahrscheinlichkeit ist sowohl beschränkt 
und abhängig von der Energie.
Desweiteren können die Schauer nur indirekt, 
aufgrund der Erdatmosphäre gemessen werden.
Da die Strecke welche Teilchen durch die Erdatmosphäre durchlaufen die
geschätzte Energie wesentlich beeinflussen,
ist der Fehler durch die \textit{indirekten Messung} direkt mit der
Richtungsrekonstruktion korreliert.
Zusätzlich weisen die geschätzten Energien einen statistischen Fehler auf. 
Da Entfaltungsalgorithmen auf \textit{begrenzten Auflösungen}/ gebinnten Daten
beruhen, kommt es vor das wenn die geschätzte Energie von der Wahren Energie
Abweicht bins über und die Entsprechend anderen unterrepräsentiert sind.

\begin{equation}
	g(y) = \int_\text{c}^\text{d} M(x,y) f(x) \text{dx} + b(y)
\end{equation}

\begin{equation}
	g_i = \sum_j M_{ij} f_j + b_i
\end{equation}

\begin{equation}
	g_i = \int_{y_{i-1}}^{y_i} g(y) \text{dy} \quad \text{und} \quad
	b_i = \int_{y_{i-1}}^{y_i} b(y) \text{dy}
\end{equation}

