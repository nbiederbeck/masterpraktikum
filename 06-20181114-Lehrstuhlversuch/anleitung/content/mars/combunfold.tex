\subsection{CombUnfold}%
\label{sub:combunfold}
Die Bestimmung der Energie von Gammaschauern beruht auf dem Look-Up-Table
Verfahren,
welche aus den Image Parametern erstellt wurden.
Die Energie, welche auf den Parametern geschätzt wird, beruht auf einer
endlichen Auflösung.
% Um einen Energieschätzer für die Quelle zu erhalten,
% wird ein Entfaltungsverfahren verwendet.
Ziel ist es, die Spektrale Energieverteilung einer Quelle zu bestimmen.
Dadurch ist eine genauere Klassifizierung des Objekts und der
Wechselwirkungsprozesse möglich.

\paragraph{Theorie}%

% Tscherenkowteleskope sind nicht in der Lage, die Energie
% des Primärteilchen direkt zu messen,
% sondern lediglich den deponierten Photostrom in den einzelnen PMTs,
% weswegen ein Entfaltungsalgorithmus benötigt wird.
% % Bei der Berechnung der Primärenergie treten beim Messen hauptsächlich dreierlei Probleme auf.
% Aufgrund der statistischen Unsicherheit der Energieschätzer und der begrenzten
% Auflösung in der Energie muss der deponierte Photostrom entfaltet werden.

Bei realen Daten ist die wahre Energie der Primärteilchen nicht bekannt,
daher wird der Fluss in einer SED in Abhängigkeit der geschätzen Energie aufgetragen. 
Aufgrund der statistischen Unsicherheit der Energieschätzer 
und der begrenzten Detektorauflösung, 
weicht die geschätze Energie von der wahren Energie ab. 
Mit Hilfe sogenannter Entfaltungsalgorithmen kann die SED in Abhängigkeit der wahren Energie abgeschätzt werden.


% \begin{description}
%     \item[\quad limitierte Akzeptanz]
%         Nicht alle Schauer können detektiert werden.
%         Die Detektionswahrscheinlichkeit wird beispielsweise limitiert
%         durch Events, welche die Triggerschwelle der PMTs nicht auslösen,
%         die Totzeit der PMTs,
%         und dem Cut auf Events, die zu klein sind, um ihre Richtung zu rekonstruieren.

%     \item[\quad indirekte Messung]
%         Die Primärenergie kann nicht direkt gemessen werden.
%         Stattdessen kann nur der Photostrom gemessen werden,
%         welcher vom Tscherenkowschauer in der Kamera deponiert wird.
%         Da die Strecke, welche die Teilchen durch die Erdatmosphäre durchlaufen,
%         die geschätzte Energie wesentlich beeinflusst,
%         ist der Fehler direkt mit der Richtungsrekonstruktion korreliert.

%     \item[\quad begrenzte Auflösung]
%         Die geschätzten Energien weisen einen statistischen Fehler auf.
%         Durch das Binning der Daten kommt es vor,
%         dass der Bin der geschätzte Energien von dem der wahren abweicht,
%         sodass Bins über- oder unterrepräsentiert sind.
% \end{description}

% Die Probleme werden durch den
Dazu wird ein
Entfaltungsalgorithmus verwendet,
welcher die Observablen $g$
in den Raum der physikalischen Größen $f$ transformiert.
% gelöst.
Das Problem wird durch das Fredholm Integral beschrieben:
\begin{equation}
	g(y) = \int_\text{c}^\text{d} M(x,y) f(x) \, \text{d}x + b(y)
\end{equation}
Hierbei ist $M(x, y)$ eine Funktion, die die Transformation der physikalischen
Daten in das Messsystem vornimmt (bspw.\ eine Detektorfunktion), und $b(y)$ der Bias.

Für die Entfaltung wird der Parameterraum diskretisiert.
Dies geschieht durch Binning der Daten,
\begin{equation}
	g_i = \int_{y_{i-1}}^{y_i} g(y) \, \text{d}y \quad \text{und} \quad
	b_i = \int_{y_{i-1}}^{y_i} b(y) \, \text{d}y,
\end{equation}
Anschließend wird versucht, die Matrixgleichung
\begin{equation}
	g_i = \sum_j M_{ij} f_j + b_i
    \label{eq:matrix_bin}
\end{equation}
zu lösen.
Die Migrationsmatrix lässt sich aus den Monte Carlo Daten
bestimmen.

Eine Methode zur Bestimmung der wahren Verteilung $f$ ist
beispielsweise das Forward Folding.
Hierbei wird eine Funktion mit freien Parametern zum Entfalten angenommen.
Die physikalischen Parameter werden mit Gleichung~\eqref{eq:matrix_bin}
berechnet und mit den gemessen verglichen.
Die Fehler werden mit dem Algorithmus der
% Eine Alternative zur Bestimmung der Migrationsmatrix
% Sie
% besteht aus der
\textit{Minimierung der kleinsten Quadrate}
angepasst.
Dabei werden die Parameter der angenommenen Funktion so bestimmt, dass
\begin{equation}
    \chi^2_0 = {\left( \vec{g} - \mathbf{M} \vec{f} \right)}^\intercal
        \cdot \mathbf{V} \left[ \vec{g} \right] \cdot
        \left( \vec{g} - \mathbf{M} \vec{f} \right)
\end{equation}
% welche die größte Übereinstimmung mit den Monte Carlo-Daten erzeugt.
minimiert wird.

Eine weitere Möglichkeit, die physikalischen Parameter zu bestimmen,
beruht auf der \textit{Inversion} der Migrationsmatrix:
\begin{equation}
    \vec{f} = \mathbf{M}^{-1} \vec{g}.
\end{equation}
Dabei entsteht das Problem, dass die unkorrelierten Messgrößen $g$
korrelierte $f$ erzeugen.
Diese
Lösung
neigt
zu starken Oszillation in $\vec{f}$.
Die Oszillationen werden durch das Entfalten von Bins mit wenig Statistik
erzeugt und führen
zu einem unverhältnismäßig hohen Rauschen.
Um dies zu minimieren und reproduzierbare Ergebnisse zu erzeugen,
muss regularisiert werden.
% Eine Alternative zur Bestimmung der Migrationsmatrix
% besteht aus der \textit{Minimierung der kleinsten Quadrate}.
% Dabei wird die Matrix bestimmt,
% welche die größte Übereinstimmung mit den Monte Carlo-Daten erzeugt.


% Des Weiteren neigt die
% die verstärkt werden und
Durch das Einführen eines Regularisierungsterms können weitere Anforderungen an die
Entfaltung, wie zum Beispiel ein glattes Spektrum
\begin{align}
    \chi^2 &= \chi^2_0 - \frac{\tau}{2} Reg(\vec{f}) \\
    \text{mit z.B. } Reg(\vec{f}) &= \sum_j {\left( \frac{d^2 f_j}{dx^2}
    \right)}^2,
    \label{eq:thikonov}
    % \text{(Tikhonov)}
\end{align}
gestellt werden.
Die Regularisierungsstärke $\tau$ ist entscheidend, wie stark die Korrelationen
der benachtbarten Bins unterdrückt wird. 

% Durch Hinzufügen eines Zusatzterms der zu minimierenden Funktion
% % während des Entfaltungsprozess 
% ist Regularisierung möglich.
Dabei gibt es verschiedene Möglichkeiten,
wovon beispielsweise die Methoden nach
Tikhonov, Bertero, und Schmelling
in MARS implementiert sind.
Bei der Methode nach Tikhonov~\eqref{eq:thikonov} wird die Glattheit der
Funktion durch Minimierung der zweiten Ableitung erreicht.

\subparagraph{Durchführung}%

Die Konfigurationsdatei \texttt{combunfold.rc} wird verwendet.
Hier müssen einige Sachen angepasst werden:
\begin{lstlisting}
  # <PATH-TO> MUSS ein absoluter Pfad sein, also mit einem / starten!!
  MCombineDataForUnfolding.InputFiles[0] = <PATH-TO>/Output_flute.root
\end{lstlisting}
Es kann
\texttt{MCombineDataForUnfolding.NSpectrumIterations}
auf eine andere Zahl gesetzt werden.
Die Inverse der Migrationsmatrix wird iterativ bestimmt.
Daher ist es notwendig, mehrere
Iterationen durchzuführen, damit sie gegen einen sinnvollen Wert strebt.
Sinnvoll sind mindestens mehr als 10 Iterationen.

Außerdem werden mit
\texttt{MCallUnfold.FlagUnfold}
ein Entfaltungsalgorithmus ausgewählt und mit
\texttt{MCallUnfold.FlagCriterion}
ein Kriterium für das Auswählen der Entfaltungsgewichte gesetzt.
Es sollen verschiedene Entfaltungsalgorithmen ausprobiert
und die Gewichte variiert werden.

Um CombUnfold zu starten, muss in das
\texttt{\$MARSSYS}
gewechselt werden.
Das Makro wird direkt in Root ausgeführt:
\begin{lstlisting}
  root
  # Warten, dass root startet
  .x CombUnfold.C("<PATH-TO>/combunfold.rc")
\end{lstlisting}

In der Ausgabedatei
\texttt{<PATH-TO>/Unfolding\_Output\_combunfold.root}
finden sich die entfalteten Spektren.
