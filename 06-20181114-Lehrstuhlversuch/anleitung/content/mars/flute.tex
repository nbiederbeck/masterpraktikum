\subsection{Flute}%
\label{sub:flute}

Ziel von Flute ist es,
den Untergrund zu schätzen,
und das Energiespektrum und
eine Lichtkurve einer Quelle zu bestimmen.

\paragraph{Theorie}%

\subparagraph{Background Estimation}
Zuerst wird der Untergrund abgeschätzt.
Dies geschieht, indem die Daten im Wobble-Modus aufgenommen werden.
Die erwartete Quellposition ist dabei nicht im
Kameramittelpunkt,
sondern um
\SI{0.4}{\degree} um die Kameraachse
verschoben.
Die Quellposition wird um die Kameraachse rotiert.
In äquidistanten Abständen um die Kameraachse werden Off-Positionen bestimmt
(siehe Abbildung~\ref{fig:hillas}).

Anschließend wird der Abstand $\theta$ der rekonstruierten
zu der On- und den Off-Positionen bestimmt.
Die Bins von $\theta^2$ werden in den Off-Positionen
anhand der On-Position (Anzahl Off-Positionen / Messzeiten) normiert.
Im Anschluss wird der aus den Off-Positionen bestimmte Untergrund
aus der Messung der On-Position subtrahiert, sodass Signal übrig bleibt
(vgl. Abbildung~\ref{fig:thetacut}).


\begin{align*}
    On &= Signal + Untergrund \\
    Off_{1} &= Untergrund \\
    Off_{2} &= Untergrund \\
    \Rightarrow Signal &= On - Off_{\text{mean}}
\end{align*}

\subparagraph{Effective Collection Area}
\subparagraph{Estimated Energy Spectrum}
\subparagraph{Flux}
Die effektive Flaeche entspricht der Flaeche bei dem ein idealer Detector 100 \%
der einfallenden Gamma Rays detektieren wuerde.
Dies entspricht dem oberflaechen integrall der Detektorflaeche 
ueber der Effizienz des Detektors eines orthogonal eintreffenden
primaerteilchens.
Von besonderem Interesse ist die "Analysis effective area" welche sich aus den
allen einzel Effizienzen zusammensetzt. 
Die Effizienz ist abhaengig von allerlei Parameter,
wovon die Pointing Position, die Richtung und die Energie des Primaerteilchends 
die wichtigsten sind.
Beispielsweise ist die Collection Area von der Energie abhaengig, da hoeher
Energetische Teilchen mehr Tscherenkovlicht emitttieren und so einfacher zu
detektieren sind.

% Zur Berechnung werden die Anzahl an detektierten Gammas,
% die effektive Observationszeit,
% und die Collecting Area des Teleskop benötigt.

\begin{equation}
\frac{\text{d} \Phi}{\text{d}E} \simeq K {\left( \frac{E}{E_0} \right)}^{- \Gamma}
\end{equation}

\subparagraph{Light Curve}

\paragraph{Durchführung}%

Die Konfigurationsdatei \texttt{flute.rc} wird verwendet.
Es müssen die Felder \texttt{flute.data} und \texttt{flute.mcdata} angepasst werden.
Es werden die von Melibea prozessierten Daten und Monte Carlos genutzt.


\begin{lstlisting}
  flute -q -b --config=flute.rc
\end{lstlisting}
