\subsection{Flute}%
\label{sub:flute}

Ziel von Flute ist es,
das Energiespektrum und
eine Lichtkurve einer Quelle zu bestimmen.

\paragraph{Theorie}%

Die
Spektrale Energieverteilung (\textit{Spectral Energy Distribution: SED})
gibt den Fluss gegen die Energie an.
% Jede Quellart wird durch eine charakteristische SED identifiziert.
Die SED ist charakteristisch für verschiedene Quellarten 
und dient somit im allgemeinen der Quellidentifikation.
Desweiteren können anhand der SED beispielsweise Rückschlüsse auf die
Beschleunigungsmechanismen der Quelle gezogen werden.
Die zeitliche Änderung eines über einen Energiebereich integrierten Flusses wird durch die Light Curve (LC) beschrieben.
% \begin{align*}
%   \text{On} &= \text{Signal} + \text{Untergrund}_{\text{on}} \\
%   \text{Off}_{1} &= \text{Untergrund}_{1} ;
%   \quad \text{Off}_{2} = \text{Untergrund}_{2} \\
%   \Rightarrow \text{Signal} &= \text{On} - \text{Off}_{\text{mean}}
% \end{align*}

\subparagraph{Effective Collection Area}
Ein Maß für die Effizienz des Detektors ist die effektive Fläche.
Sie gibt die Detektorfläche eines idealen Detektors an,
der die gleiche Anzahl an Signalereignisse messen würde,
wie ein realer Detektor.
% Die effektive Fläche entspricht der Fläche,
% bei der ein Detektor 100\%
% der einfallenden Gammas detektieren würde.
Diese entspricht dem Oberflächenintegral der Detektorfläche
über der Effizienz des Detektors
bei einem orthogonal eintreffenden Primärteilchen.
Die Effizienz ist abhängig von verschiedenen Parametern,
wobei der Zenitwinkel
und die Energie des Primärteilchens
die wichtigsten sind.
Beispielsweise ist die Collection Area von der Energie abhängig,
da höherenergetische Teilchen mehr Tscherenkowlicht emittieren
und so einfacher zu detektieren sind.

Von besonderem Interesse ist die \enquote{Analysis Effective Area}
(Effektive Fläche der gesamten Analyse),
welche den relativen Verlust von Gammateilchen durch verschiedene
Einflüsse, wie Zenithwinkel, Hardware, Schnitte in der Analyse,
beschreibt.
% welche sich aus den allen Einzeleffizienzen zusammensetzt.

% Die effektive Fläche wird durch die Mittelung von Monte Carlo-Ereignissen berechnet.
Um die effektive Fläche $A_{\text{eff}}$ zu berechnen,
werden auf einer Fläche $A_{\text{MC}}$ $N_{\gamma,\text{mc}}$ Gammaereignisse simuliert.
% auf der Gamma Ereignisse geschehen,
% mit den Teleskopen in ihrer Mitte.
Die effektive Fläche ist dann
\begin{equation}%
  \label{eq:effective_area}
  A_{\text{eff}} =
    \frac{N_{\gamma,\text{final}}}{N_{\gamma,\text{mc}}}
    \cdot A_{\text{MC}} ,
\end{equation}
wobei $N_{\gamma,\text{final}}$ der Anzahl von Gammaereignissen nach der Detektion
und allen Analyseschritten entspricht.
% Die Simulation muss in Bins aller abhängigen Parameter durchgeführt werden.



\subparagraph{Energiespektrum}

Im Allgemeinen wird ein exponentielles Energiespektrum bei hohen Energien
angenommen. 
Es hat die Form 
\begin{equation}%
  \label{eq:photon_index}
  \frac{d\Phi}{dE} \simeq K {\left(\frac{E}{E_0}\right)}^{-\Gamma},
\end{equation}
wobei $\Gamma$ der spektrale Index ist.

Es wird der \textit{Gammafluss} definiert,
als Anzahl von gemessenen Signalereignissen $N_{\gamma}$
über
einer Fläche $A$ über pro Zeit $t$.
\begin{equation}%
  \label{eq:gamma_flux}
  \Phi = \frac{d^2 N_{\gamma}}{dA_{\text{eff}}\, dt_{\text{eff}}}
  \quad \left(\si{\per\centi\meter\tothe2\per\second}\right),
\end{equation}
und der differentielle Fluss pro Energie ist
\begin{equation}%
  \label{eq:differential_energy_spectrum}
  \frac{d\Phi}{dE} = \frac{d^3N_{\gamma}}{dA\, dt\, dE}
  \quad \left(\si{\per\centi\meter\tothe2\per\second\per\tera\electronvolt}\right).
\end{equation}

Der integrierte Fluss ist für Energien $> \SI{200}{\giga\electronvolt}$
beispielsweise
\begin{equation}%
  \label{eq:integral_flux}
  \Phi_{E > \SI{200}{\giga\electronvolt}} =
    \int\limits_{\SI{200}{\giga\electronvolt}}^{\infty} \frac{d \Phi}{dE}
    \text{d} E
  \quad \left(\si{\per\centi\meter\tothe2\per\second}\right).
\end{equation}

Aus dem Binning des Flusses in der Energie
kann die SED berechnet werden:
\begin{equation}%
  \label{eq:spectral_energy_distribution}
  E^2 \cdot \frac{d \Phi}{dE}
  \quad \left(\si{\tera\electronvolt\per\centi\meter\tothe2\per\second}\right)
  % = E \cdot \frac{d \Phi}{d \left(\log E\right)}.
\end{equation}

\subparagraph{Light Curve}
In der Lichtkurve wird
der integrale Fluss $\Phi_{E > E_\text{th}}$ gegen die Zeit $t$
aufgetragen.
Sie ist dafür vorgesehen,
zeitliche Veränderungen des Flusses zu bestimmen.
Dies ist notwendig bei zeitlich veränderlichen Quellen
(z.B.\ AGNs)
oder zur Überprüfung einer Konstanz von Quellen oder Instrumenten
über die Zeit.



\paragraph{Durchführung}%

Die Konfigurationsdatei \texttt{flute.rc} wird verwendet.
Es müssen die Felder \texttt{flute.data} und \texttt{flute.mcdata} angepasst werden.
Es werden die von Melibea prozessierten Daten und Monte Carlos genutzt.


\begin{lstlisting}
  flute -q -b --config=flute.rc
\end{lstlisting}

Flute erzeugt die Datei \texttt{Output\_flute.root}.
Die Analyse kann mit den in die Plots eingezeichneten Theoriewerten überprüft werden.
Es empfielt sich ein wenig, Parameter, wie die Hadronness Efficiency, zu variieren,
um die Auswirkungen in der Analyse zu verstehen.
