\subsection{Flute}%
\label{sub:flute}

Ziel von Flute ist es,
den Untergrund zu schätzen,
und das Energiespektrum und
eine Lichtkurve einer Quelle zu bestimmen.

\paragraph{Theorie}%

\subparagraph{Background Estimation}
Der Untergrund wird abgeschätzt,
indem die Daten im Wobble-Modus aufgenommen werden.
Die erwartete Quellposition ist dabei nicht im
Kameramittelpunkt,
sondern um
\SI{0.4}{\degree} um die Kameraachse
verschoben.
Die Quellposition wird um die Kameraachse rotiert.
In äquidistanten Abständen um die Kameraachse werden Off-Positionen bestimmt
(siehe Abbildung~\ref{fig:hillas}).

Anschließend wird der Abstand $\theta$ der rekonstruierten
zu der On- und den Off-Positionen bestimmt.
Die Bins von $\theta^2$ werden in den Off-Positionen
anhand der On-Position (Anzahl Off-Positionen / Messzeiten) normiert.
Im Anschluss wird der aus den Off-Positionen bestimmte Untergrund
aus der Messung der On-Position subtrahiert, sodass Signal übrig bleibt
(vgl. Abbildung~\ref{fig:thetacut}).


\begin{align*}
  \text{On} &= \text{Signal} + \text{Untergrund}_{\text{on}} \\
  \text{Off}_{1} &= \text{Untergrund}_{1} ;
  \quad \text{Off}_{2} = \text{Untergrund}_{2} \\
  \Rightarrow \text{Signal} &= \text{On} - \text{Off}_{\text{mean}}
\end{align*}

\subparagraph{Effective Collection Area}
Die effektive Fläche entspricht der Fläche,
bei der ein idealer Detektor 100\%
der einfallenden Gammas detektieren würde.
Diese entspricht dem Oberflächenintegral der Detektorfläche
über der Effizienz des Detektors
bei einem orthogonal eintreffenden Primärteilchen.
Von besonderem Interesse ist die \enquote{Analysis Effective Area}
(Effektive Fläche der gesamten Analyse),
welche sich aus den allen Einzeleffizienzen zusammensetzt.
Die Effizienz ist abhängig von verschiedenen Parametern,
wovon die Pointing Position,
und die Richtung und Energie des Primärteilchens
die wichtigsten sind.
Beispielsweise ist die Collection Area von der Energie abhängig,
da höherenergetische Teilchen mehr Tscherenkowlicht emittieren
und so einfacher zu detektieren sind.

Die effektive Fläche wird durch die Mittelung von Monte Carlo-Ereignissen
berechnet.
Dazu wird eine homogene Fläche $A_{\text{MC}}$ simuliert,
auf der Gamma Ereignisse geschehen,
mit den Teleskopen in ihrer Mitte.
Die effektive Fläche ist dann
\begin{equation}%
  \label{eq:effective_area}
  A_{\text{eff}} =
    \frac{N_{\gamma,\text{final}}}{N_{\gamma,\text{mc}}}
    \cdot A_{\text{MC}}
\end{equation}
Die Simulation muss in Bins aller abhängigen Parameter durchgeführt werden.



\subparagraph{Energiespektrum}

Zur Berechnung des Energiespektrums muss ein Spektrum angenommen werden,
das entfaltet werden kann:
\begin{equation}%
  \label{eq:photon_index}
  \frac{d\Phi}{dE} \simeq K {\left(\frac{E}{E_0}\right)}^{-\Gamma},
\end{equation}
dabei ist $\Gamma \approx 2$ der spektrale Index.

Es wird der \textit{Gammafluss} definiert als Anzahl von emittierten Photonen
über der Detektorfläche pro Zeit:
\begin{equation}%
  \label{eq:gamma_flux}
  \Phi = \frac{d^2 N}{dS dt}
  \quad \left(\si{\per\centi\meter\tothe2\per\second}\right),
\end{equation}
und der differentielle Fluss pro Energie ist
\begin{equation}%
  \label{eq:differential_energy_spectrum}
  \frac{d\Phi}{dE} = \frac{d^3N}{dS dt dE}
  \quad \left(\si{\per\centi\meter\tothe2\per\second\per\tera\electronvolt}\right).
\end{equation}

Der Integrale Fluss ist für Energien $> \SI{200}{\giga\electronvolt}$
beispielsweise
\begin{equation}%
  \label{eq:integral_flux}
  \Phi_{E > \SI{200}{\giga\electronvolt}} =
    \int\limits_{\SI{200}{\giga\electronvolt}}^{\infty} \frac{d \Phi}{dE}
    \text{d} E
  \quad \left(\si{\per\centi\meter\tothe2\per\second}\right).
\end{equation}

{\color{red} Warum wird der Fluss bestimmt? Wahrscheinlich hat das was mit dem
Spektrum zu tun. Vielleicht so:}
Aus dem Binning des Flusses in der Energie
kann die Spektrale Energieverteilung
(\textit{Spectral Energy Distribution: SED})
entfaltet werden:
\begin{equation}%
  \label{eq:spectral_energy_distribution}
  E^2 \cdot \frac{d \Phi}{dE}
  \quad \left(\si{\tera\electronvolt\per\centi\meter\tothe2\per\second}\right)
  = E \cdot \frac{d \Phi}{d \left(\log E\right)}.
\end{equation}

\subparagraph{Light Curve}
In der Lichtkurve wird
der Fluss $\Phi$ gegen die Zeit $t$
aufgetragen.
Sie ist dafür vorgesehen,
zeitliche Veränderungen des Flusses zu bestimmen.
Dies ist notwendig bei zeitlich veränderlichen Quellen
(z.B.\ Pulsare, zur Bestimmung der Entfernung)
oder zur Überprüfung einer Konstanz von Quellen oder Instrumenten
über die Zeit.



\paragraph{Durchführung}%

Die Konfigurationsdatei \texttt{flute.rc} wird verwendet.
Es müssen die Felder \texttt{flute.data} und \texttt{flute.mcdata} angepasst werden.
Es werden die von Melibea prozessierten Daten und Monte Carlos genutzt.


\begin{lstlisting}
  flute -q -b --config=flute.rc
\end{lstlisting}
