\subsection{Flute}%
\label{sub:flute}

Ziel von Flute ist es,
den Untergrund zu schätzen,
und das Energiespektrum und
eine Lichtkurve einer Quelle zu bestimmen.

\paragraph{Theorie}%

\subparagraph{Background Estimation}
Zuerst wird der Untergrund abgeschätzt.
Dies geschieht, indem die Daten im Wobble-Modus aufgenommen werden.
Die erwartete Quellposition ist dabei nicht im
Kameramittelpunkt,
sondern um
\SI{0.4}{\degree} um die Kameraachse
verschoben.
Die Quellposition wird um die Kameraachse rotiert.
In äquidistanten Abständen um die Kameraachse werden Off-Positionen bestimmt
(siehe Abbildung~\ref{fig:hillas}).

Anschließend wird der Abstand $\theta$ der rekonstruierten
zu der On- und den Off-Positionen bestimmt.
Die Bins von $\theta^2$ werden in den Off-Positionen
anhand der On-Position (Anzahl Off-Positionen / Messzeiten) normiert.
Im Anschluss wird der aus den Off-Positionen bestimmte Untergrund
aus der Messung der On-Position subtrahiert, sodass Signal übrig bleibt
(vgl. Abbildung~\ref{fig:thetacut}).


\begin{align*}
    \text{On} &= \text{Signal} + \text{Untergrund}_{\text{on}} \\
    \text{Off}_{1} &= \text{Untergrund}_{1} ;
    \quad \text{Off}_{2} = \text{Untergrund}_{2} \\
    \Rightarrow \text{Signal} &= \text{On} - \text{Off}_{\text{mean}}
\end{align*}

\subparagraph{Effective Collection Area}
Die effektive Fläche entspricht der Fläche,
bei der ein idealer Detektor 100\%
der einfallenden Gammas detektieren würde.
Diese entspricht dem Oberflächenintegral der Detektorfläche
über der Effizienz des Detektors
bei einem orthogonal eintreffenden Primärteilchen.
Von besonderem Interesse ist die \enquote{Analysis of Effective Area},
welche sich aus den allen Einzeleffizienzen zusammensetzt.
Die Effizienz ist abhängig von verschiedenen Parametern,
wovon die Pointing Position,
und die Richtung und Energie des Primärteilchens
die wichtigsten sind.
Beispielsweise ist die Collection Area von der Energie abhängig,
da höherenergetische Teilchen mehr Tscherenkowlicht emittieren
und so einfacher zu detektieren sind.

Die effektive Flaeche wird durch die Mittlung von Monte-Carlo-Ereignissen
berechnet.
Dazu wird eine homogene Fläche simuliert, auf der Gamma-Ray ereignisse
geschehen. 

\subparagraph{Estimated Energy Spectrum}
\subparagraph{Flux}

\begin{equation}
\frac{\text{d} \Phi}{\text{d}E} \simeq K {\left( \frac{E}{E_0} \right)}^{- \Gamma}
\end{equation}

\subparagraph{Light Curve}

\paragraph{Durchführung}%

Die Konfigurationsdatei \texttt{flute.rc} wird verwendet.
Es müssen die Felder \texttt{flute.data} und \texttt{flute.mcdata} angepasst werden.
Es werden die von Melibea prozessierten Daten und Monte Carlos genutzt.


\begin{lstlisting}
  flute -q -b --config=flute.rc
\end{lstlisting}
