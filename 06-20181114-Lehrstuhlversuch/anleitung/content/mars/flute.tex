\subsection{Flute}%
\label{sub:flute}
Ziel von Flute ist,
es das Energiespektrum und die Lichtkurve einer Quelle zu bestimmen.

\paragraph{Theorie}%
\label{par:theorie}

Zur Berechnung werden die Anzahl an detektiereten Gamma-Rays,
die effektive Observationszeit,
und die Collecting Area des Teleskop benötigt.

\begin{equation}
\frac{\text{d} \Phi}{\text{d}E} \simeq K {\left( \frac{E}{E_0} \right)}^{- \Gamma}
\end{equation}


\paragraph{Durchführung}%

Die Konfigurationsdatei \texttt{flute.rc} wird verwendet.
Es müssen die Felder \texttt{flute.data} und \texttt{flute.mcdata} angepasst werden.
Es werden die von Melibea prozessierten Daten und Monte Carlos genutzt.


\begin{lstlisting}
  flute -q -b --config=flute.rc
\end{lstlisting}
