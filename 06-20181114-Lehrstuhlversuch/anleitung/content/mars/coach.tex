\subsection{Coach}%
\label{sub:coach}
Coach trainiert Modelle zur Separation von Signal und Untergrund, 
sowie der Rekonstuktion von Energie und Richtung.
\paragraph{Theorie}%
% \label{par:theorie}
{\color{red}Hier sollten am besten fuenf halbe sätze stehen damit hier nicht zwei title aufeinander
folgen. Bin aber nicht so kreativ hier so viel hin zu schreiben.}

\subparagraph{Gamma/Hadron-Separation}
Die Teleskope triggern drei Arten von Events. 
Neben den eigentlichen Luftschauern welche durch Protonen 
oder Gamma verursacht werden,
erzeugen Myonen oder versentliche trigger weitere Events.
Myonen erzeugen ringförmige Ereignisse in der Kamera.
Versentliche Triggers entstehen z.B. durch NSB und elekronisches Rauschen.

Für die Analyse einer Quelle ist es wichtig das Signal (Gammaschauer) vom 
Untergrund (alles andere) zu separieren.
Da Untergrundevents um mehrere größenordnungen häufiger auftreten (1:\num{e5})
geschieht dies durch ein multidimensionales maschinelles Klassifizierung 
System (Random Forest).

Ziel des Modell ist es Gammas dem Wert 0 und Untergrund den Wert 1 zuzuweisen.
Dazu werden für unabhängige Entscheidungsbäume, 
einer begrenzten Tiefe, 
der Informationsgehalt maximiert.
Durch Mittelung eines Events über alle trainierten Bäume entseht ein Wert
zwischen eindeutig Gamma und eindeutig Hadron, welcher als Hadroness bezeichnet
wird.
Durch schnitte auf diesem Wert kann eine Separation durchgeführt werden.

\subparagraph{Energierekonstruktion}%
\label{par:energie}

Die Energie des Gammas ist beinahe proportional
zur Anzahl der Tscherenkowphotonen
und somit der Photoelektronen in der Kamera,
jedoch hängt die Menge an detektiertem Licht von weiteren Parametern,
wie unter anderem Winkel der Kamera und Lage des Schauers im Kamerabild, ab.

In MARS wird die Energie des Primärteilchens auf zwei Arten rekonstruiert:
\begin{description}
	\item[\quad Look-up Tables] Für eine Look-up Table werden Monte Carlo 
		Gammas in ausgewählten Parametern gebinnt.
		Diesen Bins wird eine mittlere Energie aller
		zugehörigen Gammas zugewiesen.
		Bei realen Events werden die Parameter mit der Tabelle verglichen
		und so eine Energie geschätzt.
		Es muss darauf geachtet werden,
		dass die Bins möglichst homogen und mit ausreichend Statistik gefüllt sind.
	\item[\quad Random Forest] Außerdem wird eine Random Forest Regression (s.o.)
		zur Bestimmung der Energie durchgeführt.
\end{description}

\subparagraph{Richtungsrekonstruktion}%
\label{par:position}
Die Rekonstuktion der ursprünglichen Richtung ist einer der Hauptziele.
Dafür werden nach der ersten Abschätzung \ref{par:superstar} weitere 
Analyseschritte durchgeführt.
Dazu wird zunächst der \textit{DISP (\textbf{D}istance between the 
\textbf{I}mage controid and the \textbf{S}ource \textbf{P}osition)} bestimmt.
Dieser gibt die Entfernung Schauerschwerpunkt zur rekonstruierten Quellposition
entlang der Hauptachse der Ellipse an (siehe Abbildung \ref{fig:theta2}).
Das Vorzeichen des DISP Parameter ist nicht bestimmt, sodass es zwei mögliche
Quellpositionen gibt.
Da Magic ein Stereoteleskop ist kann bei der Schätzung des Vorzeichen des DISP
Parameter der Schnittpunkt der beiden Hauptachsen als zusätzliche Information
verwendet werden. 
Die rekonstruierte Position wird möglichst nah an dem Schnittpunkt der Achsen
gewählt, was den Vorteil birgt das ca 10-20 \% schlecht rekonstruierte Events
verworfen werden. 
Die Schätzung des DISP-Parameters wird momentan über einen Random-Forrest
getätigt, wo Informationen wie typischerweise, die Momenten entlang der 
Hauptachse oder der Zeitgradient eingehen.
Hadronische Schauer, welche fälschlicherweise die Gamma/Hadron-Seperation
überstanden habe, weisen meist eine schlechte Richtungsrekonstruktion auf. 

\paragraph{Durchführung}%
In der \texttt{coach.rc} sind die Pfade
\texttt{RF.mcdata},
\texttt{RF.data},
und \texttt{RF.outpath}
anzupassen.
Es sollen hier die Rootfiles mittels Wildcards angegeben werden
(Bsp.: \texttt{*.root}).
Die Monte Carlo Daten sind in einen Trainingsdatensatz
und einem Testdatensatz aufgeteilt.
Die Trainingsdaten sind durch die Endung
\texttt{*\_wr\_1.root} gekennzeichnet.
Desweiteren ist der Zenit sinnvoll einzustellen.
Zur schnelleren Erzeugung von Ergebnissen kann die Anzahl an
Entscheidungsbäumen auf \num{50}
reduziert werden.
Es sind die Argumente
\texttt{-RFgh}, \texttt{-LUTs}, und \texttt{-RFdisp}
zum Erstellen eines Random Forest für die Gamma-/Hadron-Separation,
einer Look-Up-Table für die Energieschätzung,
und eines weiteren Random Forest für zusätzliche Energieschätzung.

\begin{lstlisting}
  coach -q -b	\
    --config=coach.rc \
    -RFgh \
    -LUTs \
    -RFdisp
  # oder parallel zur schnelleren Rechnung:
  coach -q -b --config=coach.rc -RFgh
  coach -q -b --config=coach.rc -LUTs
  coach -q -b --config=coach.rc -RFdisp
\end{lstlisting}

Es werden fünf Dateien erzeugt,
die als weitere Eingangsdaten für die folgenden Analyseschritte verwendet
werden.
