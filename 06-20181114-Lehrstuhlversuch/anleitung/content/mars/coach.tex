\subsection{Coach}%
\label{sub:coach}

\paragraph{Theorie}%
\label{par:theorie}
Decision Trees erklaeren
Forrest als besonders robustes modell ...


\paragraph{Durchführung}%
In der \texttt{coach.rc} sind die Pfade anzupassen. 
Die Monte Carlo Daten sind in ein Trainingsdatensatz 
und einem Testdatensatz aufgeteilt.
Die Trainingsdaten sind durch das Kürzel 
\texttt{*\_wr\_1*} gekennzeichnet. 
Desweiteren ist der Zenith sinnvoll einzustellen.
Zur erzeugung schneller Ergebnisse kann die Anzahl an Trees auf \num{50}
reduziert werden.
Es sind die Argumente \texttt{-RFgh}, \texttt{-LUTs} und
\texttt{-RFdisp} zum erstellen eines RandomForrest für 
die Gamma-/Hadron-Seperation, eines Look-Up-Tables und 
eines Energie-Schätzers.
\begin{lstlisting}
	coach -q -b	\
		--config=coach.rc \
		-RFgh \
		-LUTs \
		-RFdisp 
\end{lstlisting}
Als Output werden drei Modelle erzeugt welche in den Folgenden Analyseschritte
auf echten einem Testdatensatz evaluiert und echte Daten verarbeitet.
