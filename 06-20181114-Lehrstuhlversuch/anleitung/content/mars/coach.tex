\subsection{Coach}%
\label{sub:coach}

\paragraph{Theorie}%
\label{par:theorie}
Decision Trees erklären
Forest als besonders robustes Modell \ldots


\paragraph{Durchführung}%
In der \texttt{coach.rc} sind die Pfade
\texttt{RF.mcdata},
\texttt{RF.data},
und \texttt{RF.outpath}
anzupassen.
Es sollen hier die Rootfiles mittels Wildcards angegeben werden
(Bsp.: \texttt{*.root}).
Die Monte Carlo Daten sind in einen Trainingsdatensatz
und einem Testdatensatz aufgeteilt.
Die Trainingsdaten sind durch die Endung
\texttt{*\_wr\_1.root} gekennzeichnet.
Desweiteren ist der Zenit sinnvoll einzustellen.
Zur schnelleren Erzeugung von Ergebnissen kann die Anzahl an
Entscheidungsbäumen auf \num{50}
reduziert werden.
Es sind die Argumente
\texttt{-RFgh}, \texttt{-LUTs}, und \texttt{-RFdisp}
zum Erstellen eines Random Forest für die Gamma-/Hadron-Seperation,
einer Look-Up-Table für die Energieschätzung,
und eines weiteren Random Forest für zusätzliche Energieschätzung.

\begin{lstlisting}
  coach -q -b	\
    --config=coach.rc \
    -RFgh \
    -LUTs \
    -RFdisp
  # oder parallel zur schnelleren Rechnung:
  coach -q -b --config=coach.rc -RFgh
  coach -q -b --config=coach.rc -LUTs
  coach -q -b --config=coach.rc -RFdisp
\end{lstlisting}

Es werden vier Dateien erzeugt,
die als weitere Eingangsdaten für die folgenden Analyseschritte verwendet
werden.
