\subsection{Coach}%
\label{sub:coach}

Coach trainiert Modelle zur Separation von Signal und Untergrund,
sowie der Rekonstruktion von Energie und Richtung.
Weitere Informationen zu Maschinellem Lernen sind in der Vorlesung
\enquote{Statistische Methoden der Datenanalyse} zu finden.

\paragraph{Theorie}%

\subparagraph{Gamma/Hadron-Separation}
Für die Analyse einer Quelle ist es wichtig,
das Signal (Gammaschauer) vom
Untergrund (hauptsächlich Protonschauer) zu separieren.
Da Untergrundevents um mehrere Größenordnungen häufiger auftreten (1:\num{e5}),
geschieht dies durch ein multidimensionales maschinelles
Klassifizierungsystem (Random Forest).

\begin{wrapfigure}[17]{L}{0.55\textwidth}
  \centering
  \includegraphics[width=\linewidth]{build/hadroness.pdf}
  \caption{Schnitte auf der Hadroness und ihre Konsequenzen für die Analyse.}%
  \label{fig:uebersicht}
\end{wrapfigure}

Ziel des Modells ist es,
den Ereignissen einen Klassifizierungswert zwischen 0 und 1 zuzuweisen,
die sogenannte Hadronness.
% Gammas den Wert 0 und Untergrund den Wert 1 zuzuweisen.
Dazu wird für unabhängige Entscheidungsbäume einer begrenzten Tiefe
auf Daten mit bekannter Zuordnung (Gamma oder Hadron) so trainiert, dass die
Trennkraft maximiert wird.
% der Informationsgehalt maximiert.
Ein Ereignis wird mit größerer Wahrscheinlichkeit einem
Gammateilchen zugeordnet, wenn die Hadronness kleiner ist.
% Durch Mittelung eines Events über alle trainierten Bäume entsteht ein Wert
% zwischen eindeutig Gamma und eindeutig Hadron, welcher als Hadroness bezeichnet
% wird.
Diese wird durch die Mittelung der Entscheidungen (0 oder 1) aller trainierter
Bäume berechnet.
Durch Schnitte auf diesem Wert kann somit die Gamma-Hadron-Separation durchgeführt werden.

\subparagraph{Energierekonstruktion}%
\label{par:energie}

Die Energie des Gammas ist annähernd proportional
zur Anzahl der Tscherenkowphotonen
und somit der Photoelektronen
in den PMTs, welche dem Schauer zugeordnet werden,
% in der Kamera,
jedoch hängt die Menge an detektiertem Licht von weiteren Parametern,
% wie unter anderem Winkel der Kamera und Lage des Schauers im Kamerabild, ab.
wie beispielsweise von dem Zenithwinkel der Quellposition ab.

In MARS kann die Energie des Primärteilchens auf zwei Arten rekonstruiert werden:
\begin{description}
	\item[\quad Look-up Tables] Für eine Look-up Table werden Monte Carlo
		Gammas in ausgewählten Parametern gebinnt.
		Diesen Bins wird eine mittlere Energie aller
		zugehörigen Gammas zugewiesen.
		Bei realen Events werden die Parameter mit der Tabelle verglichen
		und so eine Energie geschätzt.
		Es muss darauf geachtet werden,
		dass die Bins möglichst homogen und mit ausreichend Statistik gefüllt sind.
	\item[\quad Random Forest] Außerdem wird eine Random Forest Regression (s.o.)
		zur Bestimmung der Energie durchgeführt.
\end{description}

\subparagraph{Richtungsrekonstruktion}%
\label{par:position}

Die Rekonstruktion der ursprünglichen Richtung der Gamma\-/Ereignisse
ist notwendig, um zu überprüfen, ob das Ereignis der angenommenen
Quellposition zuzuordnen ist.
Dafür werden nach der ersten Abschätzung in Kapitel~\ref{sub:superstar} weitere
Analyseschritte durchgeführt.
Dazu wird zunächst die \textit{DISP (\textbf{D}istance between the
\textbf{I}mage controid and the \textbf{S}ource \textbf{P}osition)} bestimmt.
Diese gibt die Entfernung vom Schauerschwerpunkt zur rekonstruierten Quellposition
entlang der Hauptachse der Ellipse an (siehe Abbildung~\ref{fig:reco}).
Das Vorzeichen des DISP-Parameters ist nicht bestimmt,
sodass es zwei mögliche Quellpositionen gibt.
Da MAGIC ein Stereoteleskop ist,
kann bei der Schätzung des Vorzeichens des DISP-Parameters
der Schnittpunkt der beiden Hauptachsen als zusätzliche Information
verwendet werden.
Die angenommene rekonstruierte Position ist diejenge, welche näher am
Schnittpunkt der Achsen liegt.
Dadurch wird die Auswirkung von schlecht richtungsrekonstruierten Ereignissen
auf die angenomme Quellposition verringert.
Die Schätzung des DISP-Parameters wird über einen Random-Forest getätigt,
wo Informationen wie die Momente entlang der Hauptachse eingehen.
Hadronische Schauerweisen eine kleinere Richtungsgenauigkeit auf, da auf diesen
nicht trainiert wird.
Die Richtungsrekonstruktion vom Untergrund ist jedoch ohne Belangen, solange sie
homogen wie der Untergrund selber verteilt ist.


\paragraph{Durchführung}%
In der Konfigurationsdatei \texttt{coach.rc} sind die Pfade
\texttt{RF.mcdata},
\texttt{RF.data},
und \texttt{RF.outpath}
anzupassen.
Es sollen hier die Rootfiles mittels Wildcards angegeben werden
(Bsp.: \texttt{*.root}).
Die Monte Carlo Daten sind in einen Trainingsdatensatz
und einem Testdatensatz aufgeteilt.
Die Trainingsdaten sind durch die Endung
\texttt{*\_wr\_1.root} gekennzeichnet.
Desweiteren ist der Zenit sinnvoll einzustellen.
Es sind die Argumente
\texttt{-RFgh}, \texttt{-LUTs}, und \texttt{-RFdisp}
zum Erstellen eines Random Forest für die Gamma-/Hadron-Separation,
einer Look-Up-Table für die Energieschätzung,
und zwei weiterer Random Forests für die Richtungsrekonstruktion
der Ereignisse beider Teleskope.

\begin{lstlisting}
  coach -q -b	\
    --config=coach.rc \
    -RFgh \
    -LUTs \
    -RFdisp
  # oder parallel zur schnelleren Rechnung:
  coach -q -b --config=coach.rc -RFgh
  coach -q -b --config=coach.rc -LUTs
  coach -q -b --config=coach.rc -RFdisp
\end{lstlisting}

Es werden vier Dateien erzeugt,
die als weitere Eingangsdaten für die folgenden Analyseschritte verwendet
werden.
Dies sind die beiden Random Forests
\texttt{DispRF.root},
für die Richtungsrekonstruktionen
% für die beiden Teleskope,
in den Ordnern \texttt{disp1}/\texttt{disp2},
der Random Forest für die Separation
\texttt{RF.root} und die Look-Up-Table \texttt{Energy\_Table.root} für die
Energieschätzung.
