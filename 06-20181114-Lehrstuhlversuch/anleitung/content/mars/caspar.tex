\subsection{Caspar}%
\label{sub:caspar}
Skymaps sind Grundlage einer jeden Datenanalyse 
und sind von besonderem Interess bei ausgedehnten Quellen. 
Durch das vermessen von ausgedehnten Quellen, 
lässt sich beispielsweise das Alter einer Supernova-Explosion schätzen.

\paragraph{Theorie}%
\label{par:theorie}
Eine Skymap ist eine zwei-dimensionale Representation der aufgenommen Photonen.
Ziel ist es den gemessenen Events einer Position zuzuordnen. 
Dazu werden gewöhnlich drei verschieden Koordinatensysteme genutzt:
\begin{description}
	\item[\quad Kamera Koordinaten] werden genutzt um die Sensivität des Teleskop
		als eine Funktion in der Kamera zu beschreiben. 
		Eine Representation ist in Abbildung~\ref{fig:??} zu sehen. 
		Anhand derer lassen sich beispielsweise inhomogenitäten,
		wie sie ein helle Quelle verursacht,
		bereinigen.

	\item[\quad Azimuthal Koordinaten] sind abhängig von der Position des
		Beobachters auf der Erde. 
		Sie sind nicht für die observation von Astronomischen Objekten von nutzen,
		aber um die Performance zu testen.
		Dazu werden Daten bei verschiedenen Zenitwinkel verglichen was eine
		variation der Atmospheren Dicke entspricht. 

	\item[\quad Equatorial Koordinaten] werden genutzt um eine Skymap einer
		astrophysikalischen Quelle anzufertigen. 
		Mglw noch beschreiben wie es aussieht, kann ich aber nicht so schnell.
\end{description}

Die Anfertigung einer Skymap ist nicht so simple wie bei einem gewöhnlichen
Bild. 
Für eine Skymap reicht nicht allein die Konstruktion eines 2D array,
bei dem die Information allein im Signalverlauf der Pixel enthalten ist,
wie es bei gewöhnlichen Bildern der Fall ist.

Vielmehr wird für jedes gamma Event ein Einzelnes Bild aufgenommen, 
auf dem die Richtung rekonstruiert wird.
Eine Skymap bildet eine Menge an rekonstruierten Richtungen im Himmel ab. 
Da die Richtungskonstruktion der Events einen gewissen Fehler aufweist,
schmieren die Rekonstruierten um die wahre Quellposition aus. 
Das Ausschmieren der Quelle wird durch die \textit{Point Spread Function}
beschrieben 
und kann auf einer Skymap bestimmt werden. 
Durch die PSF kann das Auflösungsvermögen eines Teleskop beschrieben werden und
bildet daher eine wichtige größe.

\paragraph{Durchführung}%

Die Konfigurationsdatei \texttt{caspar.rc} wird verwendet.
Es muss das Feld \texttt{Caspar.dataName} angepasst werden.
Wie bei Odie werden die von Melibea prozessierten Daten genutzt.


\begin{lstlisting}
  caspar -q -b --config=caspar.rc
\end{lstlisting}
