\section{MARS}%
\label{sec:mars}
\begin{wrapfigure}[35]{r}{0.5\textwidth}
	\centering
	\includegraphics[height=0.8\textheight]{tikz/build/overview.pdf}
	\caption{Uebersicht der einzelnen Analyseschritte}%
	\label{fig:uebersicht}
\end{wrapfigure}

MARS ist ein Software-Paket, das in \texttt{C++} geschrieben ist.
Es beinhaltet zum einen Container, um die aufgenommenen Daten zu speichern,
zum anderen viele Ausführbare,
mit denen die gesamte Analysekette für MAGIC ausgeführt wird.

Die Hauptprogramme werden im Folgenden näher erklärt.
Dabei sind sie so angeordnet, dass die Ergebnisse der einzelnen
Programme die Eingangsdaten der nächsten sind.

\subsection{Sorcerer}%
\label{sub:sorcerer}

Das Programm Sorcerer (Simple, Outright Raw Calibration; Easy, Reliable
Extraction Routines) wird verwendet, um Rohdaten zu kalibireren.
Es soll nicht weiter ausgeführt werden, da mit bereits kalibrierten Daten
gearbeitet wird.

\subsection{Star}%
\label{sub:star}

Star wird verwendet, um auf kalibrierten Daten
das Image Cleaning %(s.\ \ref{sub:imagecleaning})
durchzuführen.
Im Anschluss werden die Hillas Parameter ausgerechnet.


\subsection{Superstar}%
\label{sub:superstar}

Da MAGIC ein Stereoteleskop ist,
müssen die Daten beider Teleskope gemeinsam analysiert werden.
Superstar wird verwendet,
um die Image Parameter der gereinigten Daten beider
Teleskope zu einer Stereo-Parameter Datei zusammenzuführen.
Dabei werden weitere Parameter bestimmt,
die nur aus den Informationen beider Teleskope gewonnen werden können.

\paragraph{Theorie}%

\begin{wrapfigure}[15]{o}{0.4\textwidth}
  \centering
  \includegraphics[width=0.9\linewidth]{tikz/build/theta2.pdf}
  \caption{Stereoparameter eines Schauers.}%
  \label{fig:reco}
\end{wrapfigure}

Da die Quellposition in verschiedene Bereichen der beiden Kamerasensoren
liegt,
weise die Ellipsen unterschiedliche Orientierung auf.
Aus dem Schnittpunkt zweier Geraden,
die durch die Hauptachsen der Ellipsen gehen,
wird die rekonstruierte Quellposition bestimmt.

Auf die ähnliche Art wird über eine Koordinatentrafo vom Kamerasystem ins
Erdsystem der Aufprallpunkt des Schauers auf dem Erdboden
bestimmt.

% \textit{Height of shower max},
% \textit{Cherenkov radius},
% \textit{Cherenkov density},

Außerdem werden die Einzelergebnisse der beiden Teleskope verglichen,
um weitere Parameter zu bestimmen.
Für die \textit{Energy discrepancy} wird
ein $\chi^2$-Test durchgeführt,
wobei angenommen wird,
dass die wahre Energie in beiden Teleskopen identisch ist.
Die Richtungsrekonstruktionen beider Teleskope werden verglichen,
und die Differenz in \textit{DISP difference} gespeichert.

\subsection{Quate}%
\label{sub:quate}

Quate ist ein Programm,
das
% Mittelwerte von einem Parametersatz ausrechnet, und
Daten selektiert und in die Klassen \textit{gut} und \textit{schlecht} einteilt.
Durch das Aussortieren kann unter anderem gewährleistet werden,
dass sich die Daten der Monte Carlo-Simulation mehr mit den Messdaten decken.



% \paragraph{Theorie}%
% Hier ist einerseits das Problem, dass der Leser gar nicht weiss, was Monte Carlos sind und warum die mit den Daten uebereinstimmen muessen.

% Im Wetter koennen Daten und MCs nicht uebereinstimmen, weil das Wetter nicht mitsimuliert wird.

% Die gemessenen Daten werden nach Wetter selektiert, einfach um zu sehen, wie gut die Daten sind und ob ggf Korrekturen stattfinden muessen.

% Ausserdem braucht man in einer normalen Analyse Crab Daten, die den Messdaten besonders aehnlich in Zenit, Wetter, Mondbedingungen, etc sind. Hier natuerlich irrrelevant weil eh Crab :)

% Hadronische Daten werden nur fuer die gh Separation gebraucht und die sollen den Daten moeglichst aehnlich im Zenit sein.

% Soviel zum Verstaendnis. Aber da der Versuch ohne Quate durchgefuehrt wird: Kapitel weglassen! Eine allgemeine Erklaerung zu "welche Daten brauchen wir warum" wo das von oben drinsteht (ohne Wetter, nur was man wofuer braucht) Vielleicht direkt nach dem Schauerkapitel, wo eh noch der Satz zur gh Separation hinsoll...
% % \label{par:theorie}
Ziel der Analyse ist es, die Unsicherheiten z.B.\ des Flusses so gering wie
möglich zu halten.
Primärteilchen schauern in einer Höhe von \SIrange{10}{20}{\kilo\meter}
auf.
Da die Entwicklung des Schauers vom Dichteprofil der Atmosphäre abhängig ist,
ist es wichtig, dieses zu kennen.
Das Dichteprofil variiert abhängig vom Zenithwinkel und Wetter.
Daten und Monte Carlos müssen deshalb im Zenithbereich übereinstimmen.
% da die Weglänge durch die Atmosphäre entscheidend für die Schauerentwicklung ist.

% Weitere Instrumente, z.B. Lidar, messen Eigenschaften der Atmosphäre.
% Dazu misst Lidar die Lichttransmission.
% Ziel ist es, einen Datensatz zu erstellen,
% mit wenig variierenden und gut simulierbaren Wetterbedingungen.

% Die Wirkungsquerschnitte der Schauerprozesse sind dichteabhängig.
% Zur guten Rekonstruktion eines Schauers muss deswegen die Dichteverteilung
% bekannt sein.
% Dies Reduziert die Unsicherheiten auf die abgeleiteten Größen.

% % \paragraph{Durchführung}%

% % Das Programm \texttt{quate} wird über die
% % Datei \texttt{quate.rc} konfiguriert.
% % In dieser werden Einstellungen getroffen,
% % sodass die Daten sinnvoll prozessiert werden.
% % Dazu ist der Azimut und Zenit entsprechend der zu
% % observierenden Quelle einzustellen.
% % Dies ist über die Parameter \texttt{AzMin},
% % \texttt{AzMax}, \texttt{ZdMin} und \texttt{ZdMin}
% % möglich.
% % Zusätzlich ist der Parameter
% % \texttt{MinAerosolTrans9km = 0.8} anzupassen,
% % damit {\color{red}\ldots Aerosol \ldots} berücksichtigt wird.
% % Überlegen Sie sich,
% % bei welchen Daten das Selektieren sinnvoll ist
% % (On/Off/MC)
% % und führen Sie dieses durch.
% % Geben Sie dazu als \texttt{<INPUT-DIRECTORY>} die entsprechenden Pfade zu den Daten an.

% % \begin{lstlisting}
% %   quate -b -s -q \
% %     --stereo \
% %     --config=quate.rc \
% %     --out=<OUTPUT-DIRECTORY> \
% %     --ind=<INPUT-DIRECTORY>
% % \end{lstlisting}
% % Quate erstellt im \texttt{<INPUT-DIRECTORY>}
% % jeweils einen Ordner \texttt{bad} und \texttt{good},
% % in denen symbolische Links zu den jeweils ausgewählten Dateien liegen.
% % Dies ist für die folgenden Analyseschritte wichtig.

\subsection{Coach}%
\label{sub:coach}



\subsection{Melibea}%
\label{sub:melibea}

\paragraph{Theorie}

Melibea wendet auf die von Superstar berechneten
Stereo-Parameter Dateien
die von Coach trainierten Modelle an.
Das Programm weist damit jedem Event
eine geschätzte Energie und Hadronness zu
und schätzt eine Quellposition (DISP).

% Auf diesen Werten können im weiteren Verlauf der Analyse
% Schnitte angewendet werden.
% Ein Schnitt auf der Hadronness $H_{\text{cut}}$
% weist Events mit einer Hadronnes $H_{\text{Evt}}~<~H_{\text{cut}}$
% die Klasse Signal zu.
% Events mit $H_{\text{Evt}} > H_{\text{cut}}$ werden der Klasse Untergrund
% zugeordnet.
% Hierbei ist das Ziel, einen möglichst reinen
% Datensatz (frei von Untergrund) zu erzeugen,
% unter der Einschränkung nicht zu viele Gammas zu verwerfen.
% Für eine Veranschaulichung des Schnittes siehe Abbildung~\ref{fig:uebersicht}.
% Für die weiteren Analyseschritte ist ein möglichst
% großer Datensatz von Vorteil.
% Die Hadronness muss so eingestellt werden,
% dass ein Kompromiss zwischen Reinheit und Größe des Datensatzes erreicht wird.


\paragraph{Durchführung}%

Für Stereo-Analysen muss die Konfigurationsdatei \texttt{melibea\_stereo.rc}
verwendet werden.
Es gibt für diese Analyse keine notwendigen Anpassungen in dieser Datei.
Melibea muss auf Daten und dem Testdatensatz der Monte Carlos einzeln angewendet werden.

\begin{lstlisting}
  # <PATH-TO> entspricht dem RF.outpath aus coach!
  melibea -q -b -f \
    --stereo \
    --config=melibea_stereo.rc \
    --rf \
    -erec \
    --calcstereodisp \
    --calc-disp-rf \
    --calc-disp2-rf \
    --disp-rf-sstrained \
    --rftree=<PATH-TO>/RF.root \
    --etab=<PATH-TO>/Energy_Table.root \
    --rfdisptree=<PATH-TO>/disp1/DispRF.root \
    --rfdisp2tree=<PATH-TO>/disp2/DispRF.root \
    --ind=<INPUT-DIRECTORY>/*.root \
    --out=<OUTPUT-DIRECTORY> \
    --log=<OUTPUT-DIRECTORY>/melibea.log
  # Ausserdem bei Monte Carlo
    -mc
    --ind=<INPUT-DIRECTORY>/*_wr_2.root \
\end{lstlisting}

Melibea erstellt für jede Eingangsdatei eine Outputdatei der
Form \texttt{*\_Q\_*} und die Datei \texttt{melibea.root}.

\subsection{Odie}%
\label{sub:odie}
Es wird ein Maß benötigt, um entscheiden zu können,
ob die Ursache der gemessen Events die Quelle selber ist.
Das Wahrscheinlichkeitsmaß wird gewöhnlicherweise in Form von Gaußschen
Standardabweichungen $\sigma$ angegeben.
Odie berechnet die Siginifikanz der Daten an einer entsprechenden Quellposition.
Ab einer Signifikanz \SI{3}{\sigma} wird von einem \textit{Hinweis}
und ab \SI{5}{\sigma} von einer \textit{Detektion} gesprochen.

\paragraph{Theorie}%

Die Wahrscheinlichkeit wird angegeben, indem geprüft wird,
ob die über\-schüssigen Ereignisse auf eine zufällige
Fluktaktion des Untergrunds zurück\-zu\-führen sind.
Dies geschieht mit einem Likelihood\-/Quotienten\-/Test.

Die Herleitung ist im Einzelnen für die Durchführung des
Versuches nicht relevant,
dient aber zum vollen Verständnis des Versuchs.

Zur Bestimmung des Photonenflusses der Quelle wird für die Zeit $t_\text{on}$ in die Region der
erwarteten Quelle geschaut und die Anzahl an Ereignissen $N_\text{on}$ gezählt.
Zur Abschätzung des Untergrundes werden für die Zeit $t_\text{off}$
(äquivalent zu $t_\text{on} / \alpha$)
Events in einer Region $N_\text{off}$,
in welcher keine Gammaquelle erwartet wird,
% wo keine Quelle erwartet wird,
gemessen.
Da das Verhältnis von Signal $N_\text{S}$ zu Untergrund
$N_\text{B}$ sehr klein ist
und der Untergrund statistischen Schwankungen unterliegt,
ist mit
\begin{equation}
	N_\text{S} = N_\text{on} - \hat{N}_\text{B} = N_\text{on} - \alpha N_\text{off}
\end{equation}
die Signifikanz
\begin{equation}
	S = \frac{N_\text{S}}{\sqrt{N_\text{S}}},
\end{equation}
nur grob abgeschätzt, da es sich um statistisches Rauschen des
Hintergrunds handeln kann.
Da Gamma Ray Ereignisse alle unabhängig voneinander sind,
sind diese poissonverteilt.

Eine statistisch stabile Methode, die Signifikanz zu bestimmen, stellt die
Likelihood Ratio Methode nach Li und Ma da.
Es wird angenommen, dass die erwartete Anzahl an Signal
Photonen $\langle N_\text{on} \rangle$ und die Anzahl an Untergrundereignissen
$\langle N_\text{off} \rangle$ nicht bekannt sind.
Die zu prüfende Nullhypothese ist:
\begin{quote}
	Keine Quelle existiert und
    alle gemessenen Events $N_\text{on}$ gehören zum Untergrund.
\end{quote}
Dies ist äquivalent zu $\langle N_\text{S} \rangle=0$.

Zur Bestimmung der Liklihoodfunktion der Nullhypothese,
verschwinden die
Ereignisse aus der Quelle und die Untergrundereignisse entsprechen der
gewichteten Summe aus der On und Off Region:
\begin{equation}
	L(X|E_0, \hat{T}_\text{c})= P_\text{r} \left[
		N_\text{on}, N_\text{off} |
		\langle N_\text{S} \rangle = 0,
		\langle N_\text{B} \rangle = \frac{\alpha}{1 + \alpha} (N_\text{on} +
			N_\text{off})
	\right]
\end{equation}
Die maximale Likelihood ergibt sich, indem die On-Region
um die Events der Hintergrundstrahlung bereinigt wird:
\begin{equation}
	L(X|\hat{E}, \hat{T})= P_\text{r} \left[
		N_\text{on}, N_\text{off} |
		\langle N_\text{S} \rangle = N_\text{on} - \alpha N_\text{off},
		\langle N_\text{B} \rangle = \alpha N_\text{off}
	\right]
\end{equation}
Durch Einsetzen in den Likelihood Quotienten Test
\begin{equation}
	\lambda = \frac{L(X|E_0, \hat{T}_\text{c})}{L(X|\hat{E}, \hat{T})}
\end{equation}
und die Annahme, dass genügend Ereignisse detektiert wurden, sodass diese einer
$\chi^2$-Verteilung mit einem Freiheitsgrad folgen,
\begin{equation}
	\sqrt{- 2 \ln \lambda} = \chi(1),
\end{equation}
ergibt sich die Li und Ma Signifikanz
\begin{equation}
  S = \sqrt{2} {\left(
      N_\text{on} \log \left[
        \left( \frac{1 + \alpha}{\alpha} \right) \left(
          \frac{N_\text{on}}{N_\text{on} + N_\text{off}}
        \right)
      \right]
      + N_\text{off} \log \left[
        (\alpha + 1) \left(
          \frac{N_\text{off}}{N_\text{on} + N_\text{off}}
        \right)
      \right]
  \right)} ^ {1/2}.
\end{equation}

Die Signifikanz kann sowohl auf Rohdaten als auch Gamma/Hadron-separierten Daten
bestimmt werden.
Sie hängt neben dem Verhältnis von $N_\text{on}$ zu $N_\text{off}$ auch
maßgeblich von der Größe von $N_\text{on}$ ab.
Eine gute Gamma/Hadron Separation hält den Untergrund gering,
das Signal jedoch hoch
und maximiert somit die Signifikanz.

Der Parameter $\theta$ gibt den Abstand zwischen
angenommener und rekonstruierter Quellposition
für jedes Event an.
Durch einen Schnitt in $\theta$ werden Events verworfen,
welche einen zu großen Abstand von der angenommenen Quelle aufweisen.
Hingegen können diffuse Untergrundereignisse jeden möglichen Abstand zur
angenommenen Signalquelle haben,
da sie den Himmelsausschnitt homogen ausfüllen.
Ziel ist es, einen Schnitt in $\theta^2$ (\texttt{theta2}) so zu setzen,
dass die Signifikanz maximal wird.
Ein beispielhafter Schnitt in Theta ist in
Abbildung~\ref{fig:thetacut} zu sehen.
% Er kann zum Beispiel durch Intervallschachtelung bestimmt werden.

\begin{wrapfigure}[10]{L}{0.45\textwidth}
		\centering
		\includegraphics[width=\linewidth]{build/theta2.pdf}
		\caption{Theta2 Schnitt auf Daten zur Maximierung der Signifikanz.}%
		\label{fig:thetacut}
\end{wrapfigure}


\paragraph{Durchführung}%

% Die Konfigurationsdatei \texttt{odie.rc} wird verwendet.
Es wird die Konfigurationsdatei \texttt{odie.rc} verwendet.
Es muss das Feld \texttt{Odie.dataName} angepasst werden.
Es gilt wieder, mit Wildcards die von Melibea prozessierten Daten zu nutzen
(Bsp.: \texttt{*\_Q\_*.root}).

\begin{lstlisting}
	odie -q -b --config=odie.rc
\end{lstlisting}

\subsection{Caspar}%
\label{sub:caspar}



\subsection{Flute}%
\label{sub:flute}

Ziel von Flute ist es, 
das Energiespektrum und
eine Lichtkurve einer Quelle zu bestimmen.

\paragraph{Theorie}%

Die
Spektrale Energieverteilung (\textit{Spectral Energy Distribution: SED})
gibt den Fluss gegen die Energie an. 
Jede Quellart wird durch eine charakteristische SED identifiziert. 
Anhand der SED können beispielsweise Rückschlüsse auf die
Beschleunigungsmechanismen der Quelle gezogen werden.
Die zeitliche Änderung des Flusses wird durch die Light Curve (LC) beschrieben.
% \begin{align*}
%   \text{On} &= \text{Signal} + \text{Untergrund}_{\text{on}} \\
%   \text{Off}_{1} &= \text{Untergrund}_{1} ;
%   \quad \text{Off}_{2} = \text{Untergrund}_{2} \\
%   \Rightarrow \text{Signal} &= \text{On} - \text{Off}_{\text{mean}}
% \end{align*}

\subparagraph{Effective Collection Area}
Die effektive Fläche entspricht der Fläche,
bei der ein idealer Detektor 100\%
der einfallenden Gammas detektieren würde.
Diese entspricht dem Oberflächenintegral der Detektorfläche
über der Effizienz des Detektors
bei einem orthogonal eintreffenden Primärteilchen.
Von besonderem Interesse ist die \enquote{Analysis Effective Area}
(Effektive Fläche der gesamten Analyse),
welche sich aus den allen Einzeleffizienzen zusammensetzt.
Die Effizienz ist abhängig von verschiedenen Parametern,
wovon die Pointing Position,
und die Energie des Primärteilchens
die wichtigsten sind.
Beispielsweise ist die Collection Area von der Energie abhängig,
da höherenergetische Teilchen mehr Tscherenkowlicht emittieren
und so einfacher zu detektieren sind.

% Die effektive Fläche wird durch die Mittelung von Monte Carlo-Ereignissen berechnet.
Um die effektive Fläche $A_{\text{eff}}$ zu berechnen,
werden auf einer homogenen Fläche $A_{\text{MC}}$ $N_{\gamma,\text{mc}}$ Gammaereignisse simuliert.
% auf der Gamma Ereignisse geschehen,
% mit den Teleskopen in ihrer Mitte.
Die effektive Fläche ist dann
\begin{equation}%
  \label{eq:effective_area}
  A_{\text{eff}} =
    \frac{N_{\gamma,\text{final}}}{N_{\gamma,\text{mc}}}
    \cdot A_{\text{MC}} ,
\end{equation}
wobei $N_{\gamma,\text{final}}$ der Anzahl von Gammaereignissen nach der Detektion
und allen Analyseschritten entspricht.
Die Simulation muss in Bins aller abhängigen Parameter durchgeführt werden.



\subparagraph{Energiespektrum}

Zur Berechnung des Energiespektrums muss ein Spektrum angenommen werden,
das entfaltet werden kann:
\begin{equation}%
  \label{eq:photon_index}
  \frac{d\Phi}{dE} \simeq K {\left(\frac{E}{E_0}\right)}^{-\Gamma},
\end{equation}
dabei ist $\Gamma \approx 2$ der spektrale Index.

Es wird der \textit{Gammafluss} definiert,
als Anzahl von gemessenen Signalereignissen
über
einer Fläche $A$ über pro Zeit $t$.
\begin{equation}%
  \label{eq:gamma_flux}
  \Phi = \frac{d^2 N}{dA_{\text{eff}}\, dt_{\text{eff}}}
  \quad \left(\si{\per\centi\meter\tothe2\per\second}\right),
\end{equation}
und der differentielle Fluss pro Energie ist
\begin{equation}%
  \label{eq:differential_energy_spectrum}
  \frac{d\Phi}{dE} = \frac{d^3N}{dA\, dt\, dE}
  \quad \left(\si{\per\centi\meter\tothe2\per\second\per\tera\electronvolt}\right).
\end{equation}

Der Integrale Fluss ist für Energien $> \SI{200}{\giga\electronvolt}$
beispielsweise
\begin{equation}%
  \label{eq:integral_flux}
  \Phi_{E > \SI{200}{\giga\electronvolt}} =
    \int\limits_{\SI{200}{\giga\electronvolt}}^{\infty} \frac{d \Phi}{dE}
    \text{d} E
  \quad \left(\si{\per\centi\meter\tothe2\per\second}\right).
\end{equation}

Aus dem Binning des Flusses in der Energie
kann die SED berechnet werden:
\begin{equation}%
  \label{eq:spectral_energy_distribution}
  E^2 \cdot \frac{d \Phi}{dE}
  \quad \left(\si{\tera\electronvolt\per\centi\meter\tothe2\per\second}\right)
  = E \cdot \frac{d \Phi}{d \left(\log E\right)}.
\end{equation}

\subparagraph{Light Curve}
In der Lichtkurve wird
der integrale Fluss $\Phi_{E > E_\text{th}}$ gegen die Zeit $t$
aufgetragen.
Sie ist dafür vorgesehen,
zeitliche Veränderungen des Flusses zu bestimmen.
Dies ist notwendig bei zeitlich veränderlichen Quellen
(z.B.\ AGNs)
oder zur Überprüfung einer Konstanz von Quellen oder Instrumenten
über die Zeit.



\paragraph{Durchführung}%

Die Konfigurationsdatei \texttt{flute.rc} wird verwendet.
Es müssen die Felder \texttt{flute.data} und \texttt{flute.mcdata} angepasst werden.
Es werden die von Melibea prozessierten Daten und Monte Carlos genutzt.


\begin{lstlisting}
  flute -q -b --config=flute.rc
\end{lstlisting}

Flute erzeugt die Datei \texttt{Output\_flute.root}.

\subsection{CombUnfold}%
\label{sub:combunfold}
\paragraph{Theorie}%
\label{par:theorie}

\begin{equation}
	g(y) = \int_\text{c}^\text{d} M(x,y) f(x) \text{dx} + b(y)
\end{equation}

\begin{equation}
	g_i = \sum_j M_{ij} f_j + b_i
\end{equation}

\begin{equation}
	g_i = \int_{y_{i-1}}^{y_i} g(y) \text{dy} \quad \text{und} \quad
	b_i = \int_{y_{i-1}}^{y_i} b(y) \text{dy}
\end{equation}


