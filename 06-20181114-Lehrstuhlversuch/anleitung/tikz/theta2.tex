\documentclass{standalone}
\usepackage{magic}
\usetikzlibrary{patterns,decorations.pathreplacing}

\begin{document}
\begin{tikzpicture} 

	\coordinate (t1) at (0.5,1);
	\coordinate (t2) at (2.5,0.5);

	% Ellipsen 
	\fill[fill=myellow!20] (t1) ellipse [fill=yellow, 
		x radius= 1.0cm, y radius= .5cm, rotate=45];
	\fill[fill=myellow!40] (t1)++(0.1,0.1) ellipse [fill=yellow, 
		x radius= 0.7cm, y radius= .3cm, rotate=45];
	\fill[fill=myellow!60] (t1)++(0.2,0.2) ellipse [fill=yellow, 
		x radius= 0.4cm, y radius= .2cm, rotate=45];
	\fill[fill=myellow!99] (t1)++(0.3,0.3) ellipse [fill=yellow, 
		x radius= 0.1cm, y radius= .1cm, rotate=45] coordinate (CoGI);
	
	\fill[fill=myellow!20] (t2) ellipse [fill=yellow, x radius= .9cm, y radius=
		.4cm, rotate=-60];
	\fill[fill=myellow!40] (t2)++(-0.05,0.0866) ellipse [fill=yellow, 
		x radius= .6cm, y radius= .25cm, rotate=-60];
	\fill[fill=myellow!60] (t2)++(-0.10,0.1732) ellipse [fill=yellow, 
		x radius= .35cm, y radius= .15cm, rotate=-60];
	\fill[fill=myellow!99] (t2)++(-0.15,0.2598) ellipse [fill=yellow, 
		x radius= .1cm, y radius= .1cm, rotate=-60] coordinate (CoGII);

	% Schwerpunkte der Ellipsen
	\draw [dashed, draw=black!70] (t1)++(45:-2.0cm) -- ++(45:5.9cm);
	\draw [dashed, draw=black!70] (t2)++(120:-2.0cm) -- ++(120:5.8cm);
	
	% rekonstruierte Positionen
	\fill [mblue!50] (CoGI)++(45:-2.1cm) circle (2pt);
	\fill [mblue!50] (CoGI)++(45:2.1cm) circle (2pt);
	\fill [mblue!50] (CoGII)++(120:-1.7cm) circle (2pt) coordinate (RPI);
	\fill [mblue!99] (CoGII)++(120:1.7cm) circle (2pt) coordinate (RPII);
	\node [above left] (label) at (RPII) {rek. Pos.};
	
	% Abstandsmass
	\draw[draw=black!70, decoration={brace,mirror,raise=6pt}, decorate] 
		(CoGII) -- node[sloped, below=8pt] {DISP} (RPI);
	\draw[draw=black!70, decoration={brace,mirror,raise=6pt}, decorate] 
		(CoGII) -- node[sloped, above=8pt] {DISP} (RPII);
	
	% Hexagonal Magic shape
	\node[ regular polygon, regular polygon sides=6, draw=mblue,
		inner sep=2.0cm, line width=0.1cm, ] (hexagon) at (1.5,1) {};

\end{tikzpicture}
\end{document}
