\section{Diskussion}%
\label{sec:diskussion}
Die Messung der longitudinalen Relaxationszeit
ergibt einen Wert von $\input{build/T_1.tex}$.
% Der Wert liegt im Rahmen der Ungenauigkeit in der Nähe des korrekten Wertes
% $T_{1, \text{Betreuer}} = \SI{2.4}{\second}$\cite{t1betreuer},
% hier liegt eine Abweichung von $\Delta T_1 = \SI{16\pm6}{\percent}$ vor.

Bei der Messung der lateralen Relaxationszeit werden die Peaks in der Meiboom-Gill-Methode
durch einen Algorithmus gesucht.
Dieser benötigt Parameter, die auf das Problem spezifisch getunt werden müssen.
In Abbildung~\ref{fig:burst_sequences} ist zu sehen, dass der erste Peak nicht identifiziert ist.
Hingegen ist in Abbildung~\ref{fig:peaks} zu sehen, dass die gefundenen Peaks
optimal durch einen Exponentiellen Zerfall beschrieben werden können.
Dies spiegelt sich in den geringen Fehlern der Parameter (vgl. Gleichung~\eqref{eq:t2}) wider.

Die Bestimmung des Diffusionskoeffizienten hängt maßgeblich von der Bestimmung der Halbwertsbreite des
Spin-Echo-Peaks ab.
Diese ist aufgrund von \num{50000} Datenpunkten, die von dem Oszilloskop gespeichert wurden, sehr genau zu bestimmen.
% Dennoch muss hier ein Fehler liegen, denn die folgende Berechnung des Molekülradius
% zeigt eine Abweichung von einer Größenordnung gegenüber den vergleichenden Berechnungen.
% Dieser Fehler kann auch durch die Bestimmung der Viskosität kommen:
% Da weder die Temperatur des Wassers in der Probe noch die der Umgebungsluft gemessen wurde,
% ist die Temperatur von $T = \SI{22}{\celsius}$ nur eine Schätzung.
Der Literaturwert des Diffusionskoeffizienten von Sauerstoff in Wasser bei \SI{25}{\celsius} liegt bei
\begin{align}
  D_{\text{Lit.}} &= \SI{2.1e-9}{\meter\squared\per\second} \text{,~\cite{diffkoeff}}
  \intertext{verglichen mit dem berechneten}
  \input{build/D.tex}
\end{align}
bedeutet dies eine relative Abweichung von $\Delta D = \SI{9.5}{\percent}$.
Hierbei jedoch muss berücksichtigt werden, dass die Temperatur während der Messung lediglich
\SI{22}{\celsius} betrug.

Die Bestimmung des Molekülradius mit den verschiedenen Varianten zum Vergleich liegen alle im Bereich von \SIrange[range-phrase={\text{~bis~}}]{1.2}{1.8}{\angstrom},
was mit den Literaturwerten von Wasserstoff und Sauerstoff gut übereinstimmt:
\begin{align*}
  r_{\text{H}} &= \SI{1.2}{\angstrom} \\
  r_{\text{O}} &= \SI{1.52}{\angstrom}\text{,~\cite{vdw}}
\end{align*}

% Weitere Gründe für die Ungenauigkeiten bei den Messwerten sind die
% möglicherweise nicht korrekt eingestellten Parameter an der Apparatur
% (Shim-Parameter, Pulszeiten).
