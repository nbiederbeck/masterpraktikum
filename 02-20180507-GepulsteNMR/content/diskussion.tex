\section{Diskussion}%
\label{sec:diskussion}
Die Messung der longitudinalen Relaxationszeit
ergibt einen Wert von $T_1 = \SI{2015.14 \pm 153.68}{\milli\second}$.
Hierbei werden im exponentiellen Fit die beiden letzten Messwerte doppelt und vierfach gewichtet,
um die grosse Messzeit und damit den geringeren relativen Messfehler zu berücksichtigen.
Der Wert liegt im Rahmen der Ungenauigkeit in der Nähe des korrekten Wertes
$T_{1, \text{Betreuer}} = \SI{2.4}{\second}$\cite{t1betreuer},
hier liegt eine Abweichung von $\Delta T_1 = \SI{16\pm6}{\percent}$ vor.

Bei der Messung der lateralen Relaxationszeit werden die Peaks in der Meiboom-Gill-Methode
durch einen Algorithmus gesucht.
Dieser benötigt Parameter, die auf das Problem spezifisch getunt werden müssen.
In Abbildung~\ref{fig:burst_sequences} ist zu sehen, dass der erste Peak nicht identifiziert ist.
Hingegen ist in Abbildung~\ref{fig:peaks} zu sehen, dass die gefundenen Peaks
optimal durch einen Exponentiellen Zerfall beschrieben werden können.
Dies spiegelt sich in den geringen Fehlern der Parameter (vgl. Gleichung~\eqref{eq:t2}) wider.

Die Bestimmung des Diffusionskoeffizienten hangt massgeblich von der Bestimmung der Halbwertsbreite des
Spin-Echo-Peaks ab.
Diese ist aufgrund von \num{50000} Datenpunkten, die von dem Oszilloskop gespeichert wurden, sehr genau zu bestimmen.
Dennoch muss hier ein Fehler liegen, denn die folgende Berechnung des Molekülradius
zeigt eine Abweichung von einer Größenordnung gegenüber den vergleichenden Berechnungen.
Dieser Fehler kann auch durch die Bestimmung der Viskosität kommen:
Da weder die Temperatur des Wassers in der Probe noch die der Umgebungsluft gemessen wurde,
ist die Temperatur von $T = \SI{22}{\celsius}$ nur eine Schätzung.

Weitere Grunde für die Ungenauigkeiten bei den Messwerten sind die
möglicherweise nicht korrekt eingestellten Parameter an der Apparatur
(Shim-Parameter, Pulszeiten).
