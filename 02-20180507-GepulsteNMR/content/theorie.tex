\section{Theorie}%
\label{sec:theorie}

\subsection{Magnetieserung im thermischen Gleichgewicht}%
\label{ssub:magnetieserung_im_thermischen_gleichgewicht}
Durch das Anlegen eines magnetischen Feldes spalten der Bahndrehimpuls die Energieniveaus in $2 I
+ 1$ in aequidistante Energieniveaus auf. 
Der Abstand der aufgespaltenen Niveaus betraegt 
\begin{equation}
		\label{eq:delta_e}
		\Delta E = \gamma B_0 \hbar. 
\end{equation}
Dies geschieht dadurch das die Entartung der Spins durch das Anlegen eines
Magnetfeldes aufgehoben wird.
\begin{figure}[ht]
		\centering
		\includegraphics[width=0.8\linewidth]{./build/aufspaltungE.pdf}
		\caption{Aufspaltung der E-Niveaus durch auesseres B-Feld}
		\label{fig:aufsp_E}
\end{figure}
Die Besetzungswahrscheinlichkeit im thermischen Gleichgewicht ist entsprechend
der Boltzmanverteilung verteilt.
\begin{equation}
		\label{eq:boltzmann}
		\frac{N(m)}{N(m-1)} = \exp \left( - \frac{\gamma B_0 \hbar}{kT} \right)
\end{equation}
Daraus resultiert das die E-Niveaus und Orientierungsrichtungen unterschiedlich
besetzt sind, woraus eine Netto Kernspinpolarisation entsteht. 
Fuer kleine Zeemanaufspaltungen $(m \gamma B_0 \hbar << kT)$ laesst sich die Kernspinpolarisation bis zur
zweiten Ordnung Tayloren 
\begin{equation}
		\label{eq:kernpo}
		\langle I_Z \rangle_P = \frac{\sum_\text{m} \hbar m \exp\left(-\frac{m \gamma B_0
		\hbar}{kT}\right)}{\sum_\text{m}\exp\left(-\frac{m \gamma B_0
		\hbar}{kT}\right)} \approx - \frac{\hbar^2 \gamma B_0}{4 kT}
\end{equation}
welche eine netto Magnetisierung $M_0$ erzeugen. 
Der Erwartungswert der Magnetisierung entlang der Feldrichtung betraegt
\begin{equation}
		\label{eq:magn}
		M_0 = N \frac{\mu_0 \gamma^2 \hbar^2 B_0}{4 kT} .
\end{equation}

\subsection{Larmor-Praezession}%
\label{sub:larmor_praezession}
Durch die Aufspaltung der Energieniveaus audgrund der Zeemaneffekts fueren
die magnetischen Momente eine Praessesionsbewegung um die Achse des von aussen
angelegten magentischen Feldes durch. 
Dieser Effekt heist Larmor Praezession. 
Auf die Magnetisierung wirkt ein Drehmoment welches versucht die Magnetisierung
parallel zur Magnetfeldrichtung zu drehen.
\begin{equation}
		\label{eq:dreh}
		\vec{D} = \vec{M} \times B_0 \vec{z}_\text{e} = \frac{\text{d} \vec{I}}{\text{d}t} 
\end{equation}
Dies erzeugt eine aenderung des Gesamtdrehimpulses $I$ welcher ueber den
gyromagnetischen Faktor $\gamma$ mit dem Magnetischen Moment verbunden ist.
\begin{equation}
		\label{eq:dreh}
		\frac{\text{d} \vec{M}}{\text{d}t} = \gamma \vec{M} \times B_0
		\vec{z}_\text{e}
\end{equation}
Durch die Wahl der B-Feld richtung in $\vec{z}_\text{e}$,
\begin{equation}
		\label{eq:ablM}
		\frac{\text{d} \vec{M_\text{z}}}{\text{d}t} = 0 ,  \hspace{1em}
		\frac{\text{d} \vec{M_\text{x}}}{\text{d}t} = \gamma \text{B}_0
		M_\text{y}, \hspace{1em}
		\frac{\text{d} \vec{M_\text{y}}}{\text{d}t} = \gamma \text{B}_0
		M_\text{x}, 
\end{equation}
 ist die Magnetisierungin $z$ eine Konstante und die fuehrt in der x-y-Ebene
 eine Praessesionsbewegung 
\begin{equation}
		\label{eq:schwM}
		M_\text{z} = \text{const}, \hspace{1em} M_\text{x} = A \cos(\gamma B_0
		t ), \hspace{1em} M_\text{y} = -A \sin(B_0 t)
\end{equation}
mit der Lamorfrequenz
\begin{equation}
		\label{eq:larmorf}
		\omega_\text{L} = \gamma B_0
\end{equation}
durch. 

\subsection{Relaxationserscheinungen}%
\label{sub:relaxationserscheinungen}
Durch hochfrequente Anregungen kann die Magnetisierung aus der
Gleichgewichtslage gebracht werden.
Nach beendigung der Stoerung strebt die Magnetisierung in der Relaxationszeit
gegen ihren Gleichgewichtswert. 
Die Bewegungsgleichungen werden die Blochschen Gleichungen genannt
\begin{align}
		\frac{\text{d} M_\text{z}}{\text{d} t} &= \frac{M_0 - M_\text{z}}{T_1} \\
		\frac{\text{d} M_\text{x}}{\text{d} t} &= \gamma B_0 M_\text{y} -
		\frac{M_\text{x}}{T_2} \\               
		\frac{\text{d} M_\text{y}}{\text{d} t} &= \gamma B_0 M_\text{x} - \frac{M_\text{y}}{T_2} 
\end{align}
wobei zwei verschiedenen Relaxationszeiten auftreten:
\begin{itemize}
		\item \textbf{Spin-Gitter-Relaxationszeit/logitudinale:}
				ist die Relaxationszeit parallel der Feldrichtung. Gibt die zeit
				an die es dauert bis Spinsysteme in Gitterschwingungen
				uebergehen vice versa. 
		\item \textbf{Spin-Spin-Relaxationszeit/transversale:} Relaxationszeit
				aufgrund der Spin Wechselwirkung von naechsten Nachbarn. Abnahme
				der zum B-Feld senkrechten Komponente
\end{itemize}

\subsection{HF-Einstrahlungsvorgaenge}%
\label{sub:hf_einstrahlungsvorgaenge}
Durch das einstrahlen von Hochfrequenz welche senkrecht zur Gleichgewichtslaage
der Magnetiserung steht 
\begin{equation}
		\label{eq:bhf}
		\vec{B}_\text{HF} = 2 \vec{B}_1 \cos(\omega t)
\end{equation}
laesst dieser sich, aus der Gleichgewichtslaage bringen. 
Fuer Frequenzen nahe der Lamorfrequenz kann man von einem in der x-y-Ebene
aufgespannten Ebene rotieredes Feld ausgehen. 
Durch die Transformation der Einheitsvektoren in das mit $B_1$ sich mitbewegende
System laesst sich die Zeitabhaengigkeit eliminieren. 
Unter beruecksichtigung das die Einheitsvektoren nun zeitabhaengige Grossen sind
laesst sich die zeitliche Ableitung der Magnetisierung darstellen als 
\begin{equation}
		\label{eq:moment}
		\frac{\text{d}\vec{M}}{\text{d} t} = \gamma \left( \vec{M} \times \left(
		\vec{B}_\text{ges} + \frac{\vec{\omega}}{\gamma} \right) \right) \ .
\end{equation}
Durch das einfuehren des effektiven Magnetfeldes 
\begin{equation}
		\label{eq:sumB}
		\vec{B}_\text{eff} = \vec{B}_0 + \vec{B}_1 + \frac{\vec{\omega}}{\gamma}
\end{equation}
ergibt sich die bekannte Kreiselgleichung, dessen Loesung die
Praessesionsbewegung der Magnetisierung um die Ruhelaage sind. 
Da die Lamorfrequenz $\omega_\text{L}$ antiparallel zum angelgeten Feld steht,
ist der fall fuer $\omega = \omega_\text{L}$ von besonderem Interesse.
Dabei Betraegt das effektive B-Feld $B_1$ und der Oeffnungswinkel der
Praezession 90 grad. 
Bleibt das Hochfrequenzfeld fuer eine Zeit von 
\begin{equation}
		\label{eq:90kick}
		\Delta t_{90} = \frac{\pi}{2 \gamma B_1}
\end{equation}
eingeschaltet, werden die Spins aus der Gelichgewichtslaage in die y-Ebene
heraugedreht. 
Fuer die doppelte dauer laesst sich ein 180 grad zustand herstellen, andenen die
Relaxationszeiten $T_1$ und $T_2$ gemessen werden koennen. 

\subsection{Relaxationsverhalten einer fluessigen Probe}%
\label{sub:relaxationsverhalten_einer_fluessigen_probe}
%#Fluessigkeit
Aufgrund der brownschen Molekularbewegung und der inhomogenitaet des
magnetischen Feldes wird die Larmorfrequenz eine Funktion der Zeit. 
Durch die Diffusion wird die Refokussierung der Spins gestoert, sodass die
Signalamplitude schneller abnimmt.
Zur berechnung wird die Diffusionsstromdichte 
\begin{equation}
		\vec{j} = -D \nabla n	
\end{equation}
berechnet welche von der Diffusionskonstantte und den Moeglichen
Spineinstellungen abhaenigig ist. 
Der Gradient der Stromdichte entspricht grade der Magnetisierung
\begin{equation}
		\frac{\partial M}{\partial t} = - \nabla \vec{J}_\text{M}
\end{equation}
was redondant fuer Spin \sfrac{1}{2} teilchen mit der Aussage 
\begin{equation}
		\label{eq:dm}
		\frac{\partial M}{\partial t} = - D \Delta M
\end{equation}
ist. Durch die Erweiterung der Blochsen Gleichungen um den Term \ref{eq:dm}
ergibt sich unter beruecksichtigung des Feldgradientens
\begin{equation}
		\label{eq:gradB}
		B_\text{z} = B_0 + G \cdot z
\end{equation}
die Bewegungsgleichung
\begin{equation}
		\frac{\partial M_\text{tr}}{\partial t} = \left(- i \omega_\text{L} - i \gamma
Gz - \frac{1}{T_2} + D \Delta \right) M_\text{tr}
\end{equation}
Loesung dieser Gleichung ist 
\begin{equation}
		M_\text{tr} = f(x,y,z,t) \cdot \exp(-i\omega_\text{L}t) \cdot
		\exp\left(-\frac{t}{T_\text{2}}\right)
\end{equation}
wobei f wiederum eine Funktion ist
\begin{equation}
		\frac{\partial f}{\partial t} = -i \gamma Gzf + D \Delta f
\end{equation}
Der Diffusionsterm wird fuer die Loesung dieser Gleichung erstmal vernachlaessig
t und durch die Einfuehrung der Amplitudenfunktion 
\begin{equation}
		f(t) = A \exp \left( -i \gamma Gz ( t -2n\tau) \right)
\end{equation}
beruecksichtigt. Durch periodische Randbedingungen ergibt sich ein eponentieller
Zusammenhand mit der halbwertsdauer
\begin{equation}%
  \label{eq:diffkoef}
  T_\text{D} = \frac{3}{D \gamma^2 G^2 \tau^2}
\end{equation}
Im folgenden wird beruecksichtigt das die Magnetisierung einerseits von der
Relaxationszeit $T_2$ als auch von der Diffusionskonstanate abhaengig ist.
\begin{equation}
		\label{eq:my}
		M_\text{y}(t) = M_0 \exp \left( - \frac{t}{T_2} \right) \exp \left( -
		\frac{t}{T_\text{D}} \right)
\end{equation}
Fuer grosse $T_D$ ist die Bestimmung der Relaxationszeit moeglich unter
vernachlaessigung der Diffusion. 
Durch hohe Feldgradienten besteht die moeglichkeit $T_D$ klein gegenueber $T_2$
zu waehlen und somit die Relaxationszeit zu bestimmen. 
