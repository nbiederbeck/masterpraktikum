\section{Theorie}%
\label{sec:theorie}
\subsection{Magnetieserung im thermischen Gleichgewicht}%
\label{ssub:magnetieserung_im_thermischen_gleichgewicht}
Durch das Anlegen eines äußeren magnetischen Feldes $B_0$ spalten der Bahndrehimpuls die Energieniveaus in $2 I
+ 1$ in äquidistante Energieniveaus auf. 
Der Abstand der aufgespaltenen Niveaus beträgt 
\begin{equation}
		\label{eq:delta_e}
		\Delta E = \gamma B_0 \hbar. 
\end{equation}
Dies geschieht dadurch, dass die Entartung der Spins durch das Anlegen eines
Magnetfeldes aufgehoben wird.
\begin{figure}[ht]
		\centering
		\includegraphics[width=0.8\linewidth]{./build/aufspaltungE.pdf}
		\caption{Aufspaltung der E-Niveaus durch äußeres magnetisches Feld.}
		\label{fig:aufsp_E}
\end{figure}
Die Besetzungswahrscheinlichkeit im thermischen Gleichgewicht ist entsprechend
der Boltzmanverteilung verteilt.
\begin{equation}
		\label{eq:boltzmann}
		\frac{N(m)}{N(m-1)} = \exp \left( - \frac{\gamma B_0 \hbar}{kT} \right)
\end{equation}
Daraus resultiert das die Energieniveaus vice versa Orientierungsrichtungen unterschiedlich
besetzt sind, woraus eine Netto Kernspinpolarisation entsteht. 
Für kleine Zeemanaufspaltungen $(m \gamma B_0 \hbar \ll kT)$ lässt sich die
Kernspinpolarisation fuer Spin $\sfrac{1}{2}$ bis zur
zweiten Ordnung durch
\begin{equation}
		\label{eq:kernpo}
		\langle I_Z \rangle_P = \frac{\sum_\text{m} \hbar m \exp\left(-\frac{m \gamma B_0
		\hbar}{kT}\right)}{\sum_\text{m}\exp\left(-\frac{m \gamma B_0
		\hbar}{kT}\right)} \approx - \frac{\hbar^2 \gamma B_0}{4 kT}
\end{equation}
nähern, welche eine Netto-Magnetisierung $M_0$ erzeugen. 
Der Erwartungswert der Magnetisierung entspricht dem Mittelwert der
Kernspinpolarisation dessen Betrag 
\begin{equation}
		\label{eq:magn}
		M_0 = N \frac{\mu_0 \gamma^2 \hbar^2 B_0}{4 kT} .
\end{equation}
ist.

\subsection{Larmor-Präzession}%
\label{sub:larmor_praezession}
Durch die Aufspaltung der Energieniveaus aufgrund der Zeemaneffekts füren
die magnetischen Momente eine Präzessionsbewegung um die Achse des von aussen
angelegten magentischen Feldes durch. 
Dieser Effekt heist Larmor Präzession. 
Auf die Magnetisierung wirkt ein Drehmoment welches versucht die Magnetisierung
parallel zur Magnetfeldrichtung zu drehen.
\begin{equation}
		\label{eq:dreh}
		\vec{D} = \vec{M} \times B_0 \vec{z}_\text{e} = \frac{\text{d} \vec{I}}{\text{d}t} 
\end{equation}
Dies erzeugt eine Änderung des Gesamtdrehimpulses $I$ welcher über den
gyromagnetischen Faktor $\gamma$ mit dem Magnetischen Moment verbunden ist.
\begin{equation}
		\label{eq:dreh}
		\frac{\text{d} \vec{M}}{\text{d}t} = \gamma \vec{M} \times B_0
		\vec{z}_\text{e}
\end{equation}
Durch die Wahl der B-Feldrichtung in $\vec{z}_\text{e}$,
\begin{equation}
		\label{eq:ablM}
		\frac{\text{d} \vec{M_\text{z}}}{\text{d}t} = 0 ,  \hspace{1em}
		\frac{\text{d} \vec{M_\text{x}}}{\text{d}t} = \gamma \text{B}_0
		M_\text{y}, \hspace{1em}
		\frac{\text{d} \vec{M_\text{y}}}{\text{d}t} = \gamma \text{B}_0
		M_\text{x}, 
\end{equation}
 ist die Magnetisierung in $z$ eine Konstante welche in der $x-y$-Ebene
 eine Präzessionsbewegung 
\begin{equation}
		\label{eq:schwM}
		M_\text{z} = \text{const}, \hspace{1em} M_\text{x} = A \cos(\gamma B_0
		t ), \hspace{1em} M_\text{y} = -A \sin(B_0 t)
\end{equation}
mit der Lamorfrequenz
\begin{equation}
		\label{eq:larmorf}
		\omega_\text{L} = \gamma B_0
\end{equation}
ausführt. 

\subsection{Relaxationserscheinungen}%
\label{sub:relaxationserscheinungen}
Durch hochfrequente Anregungen kann die Magnetisierung aus der
Gleichgewichtslage gebracht werden.
Nach Beendigung der Störung strebt die Magnetisierung in der Relaxationszeit
$T_\text{i}$ gegen ihren Gleichgewichtswert. 
Die Bewegungsgleichungen werden die Blochschen Gleichungen genannt
\begin{align}
		\frac{\text{d} M_\text{z}}{\text{d} t} &= \frac{M_0 - M_\text{z}}{T_1} \\
		\frac{\text{d} M_\text{x}}{\text{d} t} &= \gamma B_0 M_\text{y} -
		\frac{M_\text{x}}{T_2} \\               
		\frac{\text{d} M_\text{y}}{\text{d} t} &= \gamma B_0 M_\text{x} - \frac{M_\text{y}}{T_2} 
\end{align}
wobei zwei verschiedenen Relaxationszeiten auftreten:
\begin{itemize}
		\item \textbf{Spin-Gitter-Relaxationszeit/logitudinale:}
				ist die Relaxationszeit parallel der Feldrichtung. Gibt die zeit
				an die es dauert bis Spinsysteme in Gitterschwingungen
				übergehen vice versa. 
		\item \textbf{Spin-Spin-Relaxationszeit/transversale:} Relaxationszeit
				aufgrund der Spin Wechselwirkung von nächsten Nachbarn. Abnahme
				der zum B-Feld senkrechten Komponente
\end{itemize}

\subsection{HF-Einstrahlungsvorgaenge}%
\label{sub:hf_einstrahlungsvorgaenge}
Durch das einstrahlen von Hochfrequenz welche senkrecht zur Gleichgewichtslaage
der Magnetiserung steht 
\begin{equation}
		\label{eq:bhf}
		\vec{B}_\text{HF} = 2 \vec{B}_1 \cos(\omega t)
\end{equation}
lässt dieser sich, aus der Gleichgewichtslaage bringen. 
Für Frequenzen nahe der Lamorfrequenz kann man von einem in der x-y-Ebene
aufgespannten Ebene rotieredes Feld ausgehen. 
Durch die Transformation der Einheitsvektoren in das mit $B_1$ sich mitbewegende
System lässt sich die Zeitabhaengigkeit eliminieren. 
Unter Berücksichtigung, dass die Einheitsvektoren nun zeitabhängige Größen sind
lässt sich die zeitliche Ableitung der Magnetisierung darstellen als 
\begin{equation}
		\label{eq:moment}
		\frac{\text{d}\vec{M}}{\text{d} t} = \gamma \left( \vec{M} \times \left(
		\vec{B}_\text{ges} + \frac{\vec{\omega}}{\gamma} \right) \right) \ .
\end{equation}
Durch das einführen des effektiven Magnetfeldes 
\begin{equation}
		\label{eq:sumB}
		\vec{B}_\text{eff} = \vec{B}_0 + \vec{B}_1 + \frac{\vec{\omega}}{\gamma}
\end{equation}
ergibt sich die bekannte Kreiselgleichung, dessen Lösung die
Präzessionsbewegung der Magnetisierung um die Ruhelaage sind. 
Da die Lamorfrequenz $\omega_\text{L}$ antiparallel zum angelgeten Feld steht,
ist der fall für $\omega = \omega_\text{L}$ von besonderem Interesse.
Dabei beträgt das effektive B-Feld $B_1$ und der Öffnungswinkel der
Präzession \SI{90}{\degree}. 
Bleibt das Hochfrequenzfeld für eine Zeit von 
\begin{equation}
		\label{eq:90kick}
		\Delta t_{90} = \frac{\pi}{2 \gamma B_1}
\end{equation}
eingeschaltet, werden die Spins aus der Gleichgewichtslaage in die y-Ebene
heraugedreht. 
Für die doppelte Dauer lässt sich ein \SI{180}{\degree} Zustand herstellen, andenen die
Relaxationszeiten $T_1$ und $T_2$ gemessen werden können. 

\subsection{Relaxationsverhalten einer flüssigen Probe}%
\label{sub:relaxationsverhalten_einer_fluessigen_probe}
%#Fluessigkeit
Aufgrund der brownschen Molekularbewegung und der inhomogenität des
magnetischen Feldes wird die Larmorfrequenz eine Funktion der Zeit. 
Durch die Diffusion wird die Refokussierung der Spins gestört, sodass die
Signalamplitude schneller, als durch die Relaxationszeit beschrieben, abnimmt.
Zur Berechnung wird die Diffusionsstromdichte 
\begin{equation}
		\vec{j} = -D \nabla n	
\end{equation}
berechnet welche von der Diffusionskonstantem und den möglichen
Spineinstellungen abhänigig ist. 
Der Gradient der Stromdichte entspricht grade der Magnetisierung
\begin{equation}
		\frac{\partial M}{\partial t} = - \nabla \vec{J}_\text{M}
\end{equation}
was redondant für Spin \sfrac{1}{2} Teilchen mit der Aussage 
\begin{equation}
		\label{eq:dm}
		\frac{\partial M}{\partial t} = D \Delta M
\end{equation}
ist. Durch die Erweiterung der Blochsen Gleichungen um den Term \ref{eq:dm}
ergibt sich unter Berücksichtigung des Feldgradientens
\begin{equation}
		\label{eq:gradB}
		B_\text{z} = B_0 + G \cdot z
\end{equation}
die Bewegungsgleichung
\begin{equation}
		\frac{\partial M_\text{tr}}{\partial t} = \left(- i \omega_\text{L} - i \gamma
Gz - \frac{1}{T_2} + D \Delta \right) M_\text{tr}
\end{equation}
Lösung dieser Gleichung ist 
\begin{equation}
		M_\text{tr} = f(x,y,z,t) \cdot \exp(-i\omega_\text{L}t) \cdot
		\exp\left(-\frac{t}{T_\text{2}}\right)
\end{equation}
wobei $f$ wiederum eine Funktion  
\begin{equation}
		\frac{\partial f}{\partial t} = \left(-i \gamma Gz + D \Delta \right) f
\end{equation}
mit implzierter Abhängigkeit der Zeit $t$ ist.
Der Diffusionsterm wird für die Loesung dieser Gleichung erstmal vernachlässigt 
und durch die Einführung der Amplitudenfunktion 
\begin{equation}
		f(t) = A \exp \left( -i \gamma Gz ( t -2n\tau) \right)
\end{equation}
berücksichtigt. Durch periodische Randbedingungen ergibt sich ein exponentieller
Zusammenhang mit der Halbwertsdauer
\begin{equation}%
  \label{eq:diffkoef}
  T_\text{D} = \frac{3}{D \gamma^2 G^2 \tau^2}
\end{equation}
Im folgenden wird berücksichtigt, dass die Magnetisierung einerseits, von der
Relaxationszeit $T_2$, als auch andererseits von der Diffusionskonstanate abhängig ist.
\begin{equation}
		\label{eq:my}
		M_\text{y}(t) = M_0 \exp \left( - \frac{t}{T_2} \right) \exp \left( -
		\frac{t}{T_\text{D}} \right)
\end{equation}
Für große $T_D$ ist die Bestimmung der Relaxationszeit möglich unter
Vernachlässigung der Diffusion. 
Durch hohe Feldgradienten besteht die Möglichkeit $T_D$ klein gegenüber $T_2$
zu wählen und somit die Relaxationszeit zu bestimmen, indem über den Zeitraum 
$\tau = 2$ die Magnetisierung
\begin{equation}
		\label{eq:magy}
		M_\text{y}(t) = M_0 \exp \left( - \frac{t}{T_2} \right) \exp \left( -
		\frac{-D \gamma^2 G^2 t^3}{12} \right)
\end{equation}
gemessen wird.
