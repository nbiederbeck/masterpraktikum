\section{Theorie}%
\label{sec:theorie}
Ziel des Versuches ist es die beiden Abklingzeiten einer Probe zu bestimmen. 
Dabei werden verschiedene Messmethoden genutzt und sich mit deren
Charakteristiken vertraut gemacht. 
Desweiteren wird die Viskosität, der Molekülradius und die Van der Waals
Konstante von Wasser bestimmt.
\subsection{Magnetisierung im thermischen Gleichgewicht}%
\label{ssub:magnetieserung_im_thermischen_gleichgewicht}
Durch das Anlegen eines äußeren magnetischen Feldes $B_0$ spaltet der Spin die Energieniveaus in $2 I
+ 1$ in äquidistante Energieniveaus auf. 
Der Abstand der aufgespaltenen Niveaus beträgt 
\begin{equation}
		\label{eq:delta_e}
		\Delta E = \gamma B_0 \hbar. 
\end{equation}
Dies geschieht dadurch, dass die Entartung der Spins durch das Anlegen eines
Magnetfeldes aufgehoben wird.
\begin{figure}[ht]
		\centering
		\includegraphics[width=0.8\linewidth]{./build/aufspaltungE.pdf}
		\caption{Aufspaltung der E-Niveaus durch äußeres magnetisches
		Feld. \cite{anleitung}}
		\label{fig:aufsp_E}
\end{figure}
Die Besetzungswahrscheinlichkeit im thermischen Gleichgewicht ist entsprechend
der Boltzmanverteilung verteilt:
\begin{equation}
		\label{eq:boltzmann}
		\frac{N(m)}{N(m-1)} = \exp \left( - \frac{\gamma B_0 \hbar}{kT} \right)
\end{equation}
Daraus resultiert, dass die Energieniveaus beziehungsweise Orientierungsrichtungen unterschiedlich
besetzt sind, woraus eine Netto-Kernspinpolarisation entsteht. 
Für kleine Zeemanaufspaltungen $(m \gamma B_0 \hbar \ll kT)$ lässt sich die
Kernspinpolarisation für Spin $\sfrac{1}{2}$ bis zur
linearen Ordnung durch
\begin{equation}
		\label{eq:kernpo}
		\langle I_Z \rangle_P = \frac{\sum_\text{m} \hbar m \exp\left(-\frac{m \gamma B_0
		\hbar}{kT}\right)}{\sum_\text{m}\exp\left(-\frac{m \gamma B_0
		\hbar}{kT}\right)} \approx - \frac{\hbar^2 \gamma B_0}{4 kT}
\end{equation}
nähern, welche eine Netto-Magnetisierung $M_0$ erzeugen. 
Der Erwartungswert der Magnetisierung entspricht dem Mittelwert der
Kernspinpolarisation, dessen Betrag 
\begin{equation}
		\label{eq:magn}
		M_0 = N \frac{\mu_0 \gamma^2 \hbar^2 B_0}{4 kT} 
\end{equation}
ist.

\subsection{Larmor-Präzession}%
\label{sub:larmor_praezession}
Durch die Aufspaltung der Energieniveaus aufgrund der Zeemaneffekts führen
die magnetischen Momente eine Präzessionsbewegung um die Achse des von außen
angelegten magnetischen Feldes durch. 
Dieser Effekt heißt Larmor-Präzession. 
Durch das magnetische Feld werden die Dipole durch das Drehmoment $\vec{M}$ in
Feldrichtung gedreht,
% Auf die Magnetisierung wirkt ein Drehmoment, welches versucht, die Magnetisierung
% parallel zur Magnetfeldrichtung zu drehen.
\begin{equation}
		\label{eq:dreh}
		\vec{D} = \vec{M} \times B_0 \vec{z}_\text{e} = \frac{\text{d} \vec{I}}{\text{d}t} 
\end{equation}
was zu einer Änderung des Gesamtdrehimpulses $I$ führt.
Die Stärke und Richtung des Dipols ist proportional zum gyromagnetischen Faktor,
welcher das magnetisches Moment $\mu$ und den Drehimpuls $J$ verknüpft.
\begin{equation}
		\label{eq:dreh}
		\frac{\text{d} \vec{M}}{\text{d}t} = \gamma \vec{M} \times B_0
		\vec{z}_\text{e}
\end{equation}
Durch die Wahl der B-Feldrichtung in $\vec{z}_\text{e}$,
\begin{equation}
		\label{eq:ablM}
		\frac{\text{d} \vec{M_\text{z}}}{\text{d}t} = 0 ,  \hspace{1em}
		\frac{\text{d} \vec{M_\text{x}}}{\text{d}t} = \gamma \text{B}_0
		M_\text{y}, \hspace{1em}
		\frac{\text{d} \vec{M_\text{y}}}{\text{d}t} = \gamma \text{B}_0
		M_\text{x}, 
\end{equation}
 ist die Magnetisierung in $z$ eine Konstante welche in der $x-y$-Ebene
 eine Präzessionsbewegung 
\begin{equation}
		\label{eq:schwM}
		M_\text{z} = \text{const}, \hspace{1em} M_\text{x} = A \cos(\gamma B_0
		t ), \hspace{1em} M_\text{y} = -A \sin(\gamma B_0 t)
\end{equation}
mit der Lamorfrequenz
\begin{equation}
		\label{eq:larmorf}
		\omega_\text{L} = \gamma B_0
\end{equation}
ausführt. 

\subsection{Relaxationserscheinungen}%
\label{sub:relaxationserscheinungen}
Durch hochfrequente Anregungen kann die Magnetisierung aus der
Gleichgewichtslage gebracht werden.
Nach Beendigung der Störung strebt die Magnetisierung in der Relaxationszeit
$T_\text{i}$ gegen ihren Gleichgewichtswert. 
Die Bewegungsgleichungen werden die Blochschen Gleichungen genannt
\begin{align}
		\frac{\text{d} M_\text{z}}{\text{d} t} &= \frac{M_0 - M_\text{z}}{T_1} \\
		\frac{\text{d} M_\text{x}}{\text{d} t} &= \gamma B_0 M_\text{y} -
		\frac{M_\text{x}}{T_2} \\               
		\frac{\text{d} M_\text{y}}{\text{d} t} &= \gamma B_0 M_\text{x} - \frac{M_\text{y}}{T_2} 
\end{align}
wobei zwei verschiedenen Relaxationszeiten auftreten:
\begin{itemize}
		\item \textbf{Spin-Gitter-Relaxationszeit/longitudinal:}
				Relaxationszeit parallel zur Feldrichtung. Gibt die Zeit
				an, die es dauert bis Spinsysteme in Gitterschwingungen
				übergehen vice versa. 
		\item \textbf{Spin-Spin-Relaxationszeit/transversal:} Relaxationszeit
				unter anderem aufgrund der Spin Wechselwirkung von nächsten
				Nachbarn, als auch Feldinhomogenitäten. Abnahme
				der zum $B$-Feld senkrechten Komponente.
\end{itemize}

\subsection{HF-Einstrahlungsvorgänge}%
\label{sub:hf_einstrahlungsvorgaenge}
Durch das Einstrahlen von Hochfrequenz, welche senkrecht zur Gleichgewichtslage
der Magnetisierung steht 
\begin{equation}
		\label{eq:bhf}
		\vec{B}_\text{HF} = 2 \vec{B}_1 \cos(\omega t)
\end{equation}
lässt diese sich aus der Gleichgewichtslaage bringen. 
Durch die Transformation der Einheitsvektoren in das mit $B_1$ sich mitbewegende
System lässt sich die Zeitabhängigkeit des $B_1$ Feldes für einen mitbewegten
Beobachter eliminieren. 
Unter Berücksichtigung, dass die Einheitsvektoren nun zeitabhängige Größen sind,
lässt sich die zeitliche Ableitung der Magnetisierung darstellen als 
\begin{equation}
		\label{eq:moment}
		\frac{\text{d}\vec{M}}{\text{d} t} = \gamma \left( \vec{M} \times \left(
		\vec{B}_\text{ges} + \frac{\vec{\omega}}{\gamma} \right) \right) \ .
\end{equation}
Durch das Einführen des effektiven Magnetfeldes 
\begin{equation}
		\label{eq:sumB}
		\vec{B}_\text{eff} = \vec{B}_0 + \vec{B}_1 + \frac{\vec{\omega}}{\gamma}
\end{equation}
ergibt sich die bekannte Kreiselgleichung, dessen Lösung die
Präzessionsbewegung der Magnetisierung um die Ruhelage ist. 
Da die Lamorfrequenz $\omega_\text{L}$ antiparallel zum angelegeten Feld steht,
ist der Fall für $\omega = \omega_\text{L}$ von besonderem Interesse da
$B_\text{eff}$ = $B_1$ ist.
Dabei beträgt das effektive B-Feld $B_1$ und der Öffnungswinkel der
Präzession \SI{90}{\degree}. 
Bleibt das Hochfrequenzfeld für eine Zeit von 
\begin{equation}
		\label{eq:90kick}
		\Delta t_{90} = \frac{\pi}{2 \gamma B_1}
\end{equation}
eingeschaltet, werden die Spins aus der Gleichgewichtslage in die y-Ebene
herausgedreht. 
Für die doppelte Pulslänge des $\Delta t_{90}$ lässt sich ein \SI{180}{\degree}
Puls herstellen.
Dieser entspricht einem Spinflip. 

\subsection{Diffusionsverhalten einer flüssigen Probe}%
\label{sub:relaxationsverhalten_einer_fluessigen_probe}
%#Fluessigkeit
Aufgrund der Brownschen Molekularbewegung und der Inhomogenität des
magnetischen Feldes wird die Larmorfrequenz eine Funktion der Zeit. 
Die dephasierten Spins laufen immmer weiter auseinander und koennen durch einen
Spinflip refokussiert werden.
Dabei behalten sie ihre Richtung bei und laufen nach einem \SI{180}{\degree}
Puls wieder zusammen.
Durch die Diffusion wird die Refokussierung der Spins gestört, sodass die
Signalamplitude schneller, als durch die Relaxationszeit beschrieben, abnimmt.
Zur Berechnung wird die Diffusionsstromdichte 
\begin{equation}
		\vec{j} = -D \nabla n	
\end{equation}
berechnet, welche von der Diffusionskonstante und den möglichen
Spineinstellungen abhänigig ist. 
Der Gradient der Stromdichte entspricht grade der Magnetisierung
\begin{equation}
		\frac{\partial M}{\partial t} = - \nabla \vec{J}_\text{M}
\end{equation}
was äquivalent für Spin \sfrac{1}{2} Teilchen mit der Aussage 
\begin{equation}
		\label{eq:dm}
		\frac{\partial M}{\partial t} = D \Delta \vec{M}
\end{equation}
ist. Durch die Erweiterung der Blochschen Gleichungen um den Term~\ref{eq:dm}
ergibt sich unter Berücksichtigung des Feldgradienten
\begin{equation}
		\label{eq:gradB}
		B_\text{z} = B_0 + G \cdot z
\end{equation}
die Bewegungsgleichung
\begin{equation}
		\frac{\partial M_\text{tr}}{\partial t} = \left(- i \omega_\text{L} - i \gamma
Gz - \frac{1}{T_2} + D \Delta \right) M_\text{tr}
\end{equation}
Lösung dieser Gleichung ist 
\begin{equation}
		M_\text{tr} = f(x,y,z,t) \cdot \exp(-i\omega_\text{L}t) \cdot
		\exp\left(-\frac{t}{T_\text{2}}\right)
\end{equation}
wobei $f$ wiederum eine Funktion  
\begin{equation}
		\frac{\partial f}{\partial t} = \left(-i \gamma Gz + D \Delta \right) f
\end{equation}
mit implizierter Abhängigkeit der Zeit $t$ ist.
Der Diffusionsterm wird für die Lösung dieser Gleichung erstmal vernachlässigt 
und durch die Einführung der Amplitudenfunktion 
\begin{equation}
		f(t) = A \exp \left( -i \gamma Gz ( t -2n\tau) \right)
\end{equation}
berücksichtigt. Durch sukzessive Einsetzen ergibt sich ein exponentieller
Zusammenhang mit der Halbwertsdauer
\begin{equation}%
  \label{eq:diffkoef}
  T_\text{D} = \frac{3}{D \gamma^2 G^2 \tau^2}
\end{equation}
Im folgenden wird berücksichtigt, dass die Magnetisierung einerseits von der
Relaxationszeit $T_2$, als auch andererseits von der Diffusionskonstante abhängig ist.
\begin{equation}
		\label{eq:my}
		M_\text{y}(t) = M_0 \exp \left( - \frac{t}{T_2} \right) \exp \left( -
		\frac{t}{T_\text{D}} \right)
\end{equation}
Für große $T_D$ ist die Bestimmung der Relaxationszeit möglich unter
Vernachlässigung der Diffusion. 
Durch hohe Feldgradienten besteht die Möglichkeit $T_D$ klein gegenüber $T_2$
zu wählen und somit die Relaxationszeit zu bestimmen, indem über den Zeitraum 
$t = 2\tau$ die Magnetisierung
\begin{equation}
		\label{eq:magy}
		M_\text{y}(t) = M_0 \exp \left( - \frac{t}{T_2} \right) \exp \left( -
		\frac{D \gamma^2 G^2 t^3}{12} \right)
\end{equation}
gemessen wird.
