\section{Durchfuehrung}%
\label{sec:durchfuehrung}

\subsection{Einstellung der Apparatur}%
\label{sub:einstellung_der_apparatur}

Zuerst wird zu justage zwecken eine mit paramagnetischen Zentren versetzte
Wasserprobe in die Spule gebracht.
Dies verkuerzt die Relaxationszeit um die Richtige Larmorfrequenz zu finden. 
Im Anschluss werden die FID Parameter iterativ optimiert und nach dem austausch
mit bidestilierten Wasser noch ein weiteres mal ueberprueft. 
Dafuer muss die Wiederholungszeit auf ueber \SI{15}{\second} gestellt werden.

\subsection{Bestimmung der longitudinale Relaxationszeit}%
\label{sub:bestimmung_der_longitudinale_relaxationszeit}
Zunaechst wird die Pulslaengen A und B so eingestellt das zuerst der 
$\Delta t_{180}$ und im Anschluss erst der $\Delta t_{90}$ Kick auftritt. 
Es werden ca 10 Werte im logarithmischen Abstand genommen und die
Spannungsmaxima des 2 Kicks auf einem Teleskope abgelesen.

\subsection{Meiboom-Gill-Methode}%
\label{sub:meiboom_gill_methode}
Zur bestimmung des Spin-echos mittels der Meiboom-Gill-Methode ist zunaechst der
Schalter MG umzulegen. Anschliessend kann bei ausreichender Anzahl der 180 grad
kicks eine Echoamplitude auf einen Stick aufgezeichnet werden dessen Peaks einen
exponentiellen abfall folgen. 
Die Abstaende $\tau_1$ sollten hinreichend klein seien, damit der Therm der
Diffusion hinreichend konstant wird.
Wenn dies beachtet wird ist durch den exponentiellen Zusammenhang die
Relaxationszeit zu bestimmen. 
Desweiteren wird eine Aufnahme auf den Stick nach der Carr-Purcell-Methode auf
dem Stick gespeichert.

\subsection{Diffusionskonstante}%
\label{sub:diffusionskonstante}
Zur Messung der Diffusionskonstante muss $\tau_1$ gros gewaehlt werden. 
Desweiteren wird der Feldgradient maximal eingestellt.
Der wert des Feldgradientens laesst sich aus der Halbwertsbreite des Spinechos
berechnen.
Aufgrund fehlender Apparatur kann die Temperatur nicht weiter bestimmt werden
und wird als Zimmertemperatur angenommen. 

