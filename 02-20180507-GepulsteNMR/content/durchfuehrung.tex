\section{Durchführung}%
\label{sec:durchfuehrung}

\subsection{Einstellung der Apparatur}%
\label{sub:einstellung_der_apparatur}

Zuerst wird zu Justage zwecken eine mit paramagnetischen Zentren versetzte
Wasser-Probe in die Spule gebracht.
Dies verkürzt die Relaxationszeit um die Richtige Larmorfrequenz zu finden. 
Im Anschluss werden die FID Parameter iterativ optimiert und nach dem Austausch
mit bidestilliertem Wasser noch ein weiteres mal überprüft. 
Dafür muss die Wiederholungszeit auf über \SI{15}{\second} eingestellt werden.

\subsection{Bestimmung der longitudinale Relaxationszeit}%
\label{sub:bestimmung_der_longitudinale_relaxationszeit}
Zunächst wird die Pulslängen A und B so eingestellt das zuerst der 
$\Delta t_{180}$ und im Anschluss erst der $\Delta t_{90}$ Kick auftritt. 
Es werden ca 10 Werte im logarithmischen Abstand genommen und die
Spannungsmaxima des 2 Kicks auf einem Oszilloskop abgelesen.

\subsection{Meiboom-Gill-Methode}%
\label{sub:meiboom_gill_methode}
Zur Bestimmung des Spin-Echos mittels der Meiboom-Gill-Methode ist zunächst der
Schalter MG umzulegen. Anschließend kann bei ausreichender Anzahl der 180 grad
Kicks eine Echoamplitude auf einen Stick aufgezeichnet werden dessen Peaks einen
exponentiellen Abfall folgen. 
Die Abstände $\tau_1$ sollten hinreichend klein seien, damit der Term der
Diffusion hinreichend konstant wird.
Wenn dies beachtet wird ist durch den exponentiellen Zusammenhang die
Relaxationszeit zu bestimmen. 
Desweiteren wird eine Aufnahme auf den Stick nach der Carr-Purcell-Methode auf
dem Stick gespeichert.

\subsection{Diffusionkonstante}%
\label{sub:diffusionskonstante}
Zur Messung der Diffusionkonstante muss $\tau_1$ gros gewählt werden. 
Desweiteren wird der Feldgradient maximal eingestellt.
Der Wert des Feldgradientens lässt sich aus der Halbwertsbreite des Spinechos
berechnen.
Aufgrund fehlender Apparatur kann die Temperatur nicht weiter bestimmt werden
und wird als Zimmertemperatur angenommen. 

\subsection{Viskosität}
Die Viskosität von Wasser wird mit einem Viskosimeter bestimmt. Dazu wird
dieses befüllt und zunächst zur Justage evakuiert.
Die Messung der Durchlaufzeit des Wassers startet durch das Entlüften des
Systems. 
