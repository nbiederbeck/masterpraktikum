\section{Durchführung}%
\label{sec:durchfuehrung}

\subsection{Einstellung der Apparatur}%
\label{sub:einstellung_der_apparatur}

Zuerst wird zu Justagezwecken eine mit paramagnetischen Zentren (Chloridsulfat
ClSO$_4$) versetzte Wasser-Probe in die Spule gebracht.
Dies verkürzt die Relaxationszeit, um schneller die richtigen Paramter zu finden. 
Im Anschluss werden die FID Parameter (Larmorfrequenz, Shim, Pulslänge) iterativ optimiert und nach dem Austausch
mit bidestilliertem Wasser noch ein weiteres mal überprüft. 
Die Parameter sind optimal eingestellt wenn die Amplitude möglichst groß, die
Länge des Freien Induktionszerfall möglichst klein und es zu keinen
Überschwingern der Induktionsspannung kommt.
Dafür muss die Wiederholungszeit auf über \SI{15}{\second} eingestellt werden.

\subsection{Bestimmung der longitudinale Relaxationszeit}%
\label{sub:bestimmung_der_longitudinale_relaxationszeit}
Zunächst werden die Pulslängen A und B so eingestellt, dass zuerst der 
$\Delta t_{180}$ und im Anschluss erst der $\Delta t_{90}$ Puls auftritt. 
Es werden 13 Werte im logarithmischen Abstand genommen und die
Spannungsmaxima des 2 Pulses auf einem Oszilloskop abgelesen.

\subsection{Meiboom-Gill-Methode}%
\label{sub:meiboom_gill_methode}
Zur Bestimmung des Spin-Echos mittels der Meiboom-Gill-Methode ist zunächst der
Schalter MG umzulegen. Anschließend kann bei ausreichender Anzahl der \SI{180}{\degree}
Pulse eine Echoamplitude auf einen USB-Stick aufgezeichnet werden, dessen Peaks einen
exponentiellen Abfall folgen. 
Die Abstände $\tau_1$ sollten hinreichend klein sein, damit der Term der
Diffusion hinreichend konstant wird.
Wenn dies beachtet wird, ist durch den exponentiellen Zusammenhang die
Relaxationszeit zu bestimmen. 
Des Weiteren wird eine Aufnahme nach der Carr-Purcell-Methode auf
dem USB-Stick gespeichert.

\subsection{Diffusionkonstante}%
\label{sub:diffusionskonstante}
Zur Messung der Diffusionkonstante wird die Zeit, indem der Spinflip einsetzt
variiert. 
Des Weiteren wird der Feldgradient maximal eingestellt.
Der Wert des Feldgradienten lässt sich aus der Halbwertsbreite des Spinechos
berechnen.
Aufgrund fehlender Apparatur kann die Temperatur nicht weiter bestimmt werden
und wird als Zimmertemperatur angenommen. 

\subsection{Viskosität}
Die Viskosität von Wasser wird mit einem Viskosimeter bestimmt. Dazu wird
dieses befüllt und zunächst zum ansaugen des Wassers evakuiert.
Die Messung der Durchlaufzeit des Wassers startet sobald das Wasser in einem
Behälter eine Markierung unterschreitet. 
Ist so viel Wasser über eine Kapillare abgelaufen das eine zweite Markierung
erreicht wird, wird die Zeit gestoppt.
