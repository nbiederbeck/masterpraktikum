\section{Auswertung}%
\label{sec:auswertung}

\subsection{Messung $T_1$}%
\label{sub:messung_t_1_}

Für die Messung der longitudinalen Relaxationszeit $T_1$ wird der Zeitabstand
$\tau$ zwischen dem $\Delta t_{180}$- und $\Delta t_{90}$-Impuls
gegen die Impulshöhe $U$ aufgetragen.
Die Messdaten stehen in Tabelle~\ref{tab:t1} und sind in Abbildung~\ref{fig:messung_T1}
logarithmisch aufgetragen.

\begin{table}[ht]
  \centering
  \caption{%
    Messwerte für die Bestimmung der longitudinalen Relaxationszeit $T_1$.%
  }%
  \label{tab:t1}
  \input{build/table_messung_T1.tex}
\end{table}

\begin{figure}[ht]
  \centering
  \includegraphics[width=0.9\linewidth]{build/messung_T1.png}
  \caption{%
    Messwerte und exponentieller Fit für die Bestimmung der longitudinalen Relaxationszeit $T_1$.%
  }%
  \label{fig:messung_T1}
\end{figure}

Ein exponentieller Fit der Form
\begin{align}
  \nonumber U &= -a \cdot e^{-\sfrac{\tau}{T_1}} + m \\
  \intertext{ergibt die Parameter}
  \label{eq:t1} T_1 &= \SI{2015.14 \pm 153.68}{\milli\second} \\
  \nonumber a &= \SI{1402.42 \pm 23.87}{\milli\volt} \\
  \nonumber m &= \SI{651.88 \pm 15.46}{\milli\volt}.
\end{align}
Die letzten beiden Messwerte werden hierbei doppelt und vierfach gewichtet,
da diese aufgrund der langen Messzeit die geringsten Messfehler besitzen.

\subsection{Messung $T_2$}%
\label{sub:messung_t_2_}

Die laterale Relaxationszeit $T_2$ wird nach der Meiboom-Gill-Methode bestimmt.
Die vom Oszilloskop aufgenommene Burstsequenz ist in Abbildung~\ref{fig:burst_sequences}
dargestellt, ebenso die Carr-Purcell-Methode.

Die Peaks werden mittels exponentiellem Fit nach
\begin{align}
  \nonumber U &= a \cdot e^{-\sfrac{t}{T_2}} + m \\
  \intertext{gefittet. Die Parameter lauten}
  \label{eq:t2} T_2 &= \SI{1.47 \pm 0.06}{\milli\second} \\
  \nonumber a &= \SI{0.67 \pm 0.01}{\milli\volt} \\
  \nonumber m &= \SI{0.00 \pm 0.02}{\milli\volt}.
\end{align}

Der Fit und die Peaks sind in Abbildung~\ref{fig:peaks} dargestellt.
\begin{figure}[ht]
  \centering
  \includegraphics[width=0.9\linewidth]{build/burst_peaks_mg.png}
  \caption{Peaks des Meiboom-Gill Bursts inklusive exponentieller Fit.}%
  \label{fig:peaks}
\end{figure}

\begin{figure}[ht]
  \centering
  \includegraphics[width=0.9\linewidth]{build/burst_sequences.png}
  \caption{%
    Burstsequenzen der Meiboom-Gill-Methode (oben) und Carr-Purcell-Methode (unten).
    Markiert sind die Peaks, die mittels \texttt{scipy.signal.find\_peaks\_cwt}\cite{scipy} gefunden wurden.
  }%
  \label{fig:burst_sequences}
\end{figure}

\subsection{Messung Diffusionskoeffizient $D$}%
\label{sub:messung_diffusionskoeffizient_d_}

Fuer die Bestimmung des Diffusionskoeffizienten muss der Feldgradient $G$ bekannt sein.
Dieser wird ueber die Halbwertsbreite $t_{\sfrac{1}{2}}$ des Spin-Echos bestimmt.

\subsubsection{Halbwertsbreite}%
\label{sub:halbwertsbreite}

Aus Gleichung~\eqref{eq:halbwertszeit} laesst sich bei bekannter Halbwertsbreite des Spin-Echos der Feldgradient $G$ bestimmen.

\begin{align}
  \nonumber
  \frac{1}{4} d \gamma G t_{\sfrac{1}{2}} &= \num{2.2} \hspace{1em} \text{\cite{anleitung}} \\
  \intertext{mit $d = \SI{4.4}{\milli\meter}$ Probendurchmesser, $\gamma = \SI{2.68e8}{\radian\per\second\per\tesla}$ gyromagnetisches Verhaeltnis von Protonen\cite{gammap}}
  \label{eq:halbwertszeit}
  \Rightarrow G &= \frac{\num{8.8}}{d \gamma t_{\sfrac{1}{2}}}
\end{align}

Die Halbwertsbreite wird bestimmt, indem die volle Breite bei halber Hoehe des Spin-Echo-Peaks bestimmt wird (vgl. Abbildung~\ref{fig:halftime_sequence} unten).

Es ist
\begin{align}%
  \label{eq:t_1/2}
  t_{\sfrac{1}{2}} = \SI{92.1}{\micro\second}.
\end{align}

Somit wird
\begin{align}%
  \label{eq:g}
  G = \SI[]{8.1e-05}{\tesla\per\milli\meter\per\radian}.
\end{align}

Der Diffusionskoeffizient ergibt sich nach Gleichung~\eqref{eq:diffkoef}:
\begin{align}%
  \label{eq:d}
  D = \SI[]{1.71e-17}{\meter\squared\per\second}
\end{align}


\begin{figure}[ht]
  \centering
  \includegraphics[width=0.9\linewidth]{build/halftime_sequence.png}
  \caption{Sequenz für die Bestimmung der Halbwertsbreite $t_{\sfrac{1}{2}}$.}%
  \label{fig:halftime_sequence}
\end{figure}

\begin{table}[ht]
  \centering
  \caption{Messwerte für die Bestimmung des Diffusionskoeffizienten.}%
  \input{build/table_messung_D.tex}
\end{table}

\begin{figure}[ht]
  \centering
  \includegraphics[width=0.9\linewidth]{build/messung_D.png}
  \caption{Messwerte für die Bestimmung des Diffusionskoeffizienten und exponentieller Fit.}%
  \label{fig:messung_D}
\end{figure}


\subsection{Eigenschaften der Probe}%
\label{sec:eigenschaften_der_probe}

\subsubsection{Viskosität}%
\label{sub:viskositaet}

Die Viskositaet wird folgendermassen bestimmt:
\begin{align}
  \nonumber
  \eta \left(T\right) &= \alpha \rho \left(t - \delta\right) \\
  \intertext{mit $\alpha = \SI[]{1.024e-09}{\meter\squared\per\second\squared}$ Eichkonstante der Apparatur, $\delta = \SI[]{0.5}{\second}$ ein Korrekturglied, $\rho = \SI[]{1.0}{\gram\per\centi\meter\cubed}$ die Dichte von Wasser, ist}
  \label{eq:viskositaet}
  \eta \left(\SI{22}{\celsius}\right)&= \SI[]{0.94}{\milli\pascal\second}.
\end{align}

\subsubsection{Molekülradius}%
\label{sub:molekuelradius}
Der Molekülradius berechnet sich ueber die nachstehende Formel:
\begin{align}
  \label{eq:molekuelradius}
  D = \frac{k T}{6 \pi r \eta}
\end{align}

Diese Bestimmung wird verglichen mit einer Berechnung aus Molekuelgewicht und Dichte.
Die Ergebnisse lauten
\begin{align}
  r_{\text{Viskositaet}}&= \SI[]{0.0135}{\meter} \\
  r_{\text{Molekuelgewicht}}&= \SI[]{0.026}{\meter}
\end{align}

