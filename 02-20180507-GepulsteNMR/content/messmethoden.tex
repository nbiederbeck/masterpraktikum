\section{Messmethoden}%
\label{sec:messmethoden}
In Abbildung \ref{fig:aufbau} ist der aufbau der gepulsten Kernspinresonanz zu
sehen. 
Dazu wird eine Probe zwischen zwei Polschuhe gebracht und ein magnetisches Feld
angelegt.
Senkrecht zum magnetischen Feld wird ein weiteres Hochfrequentes feld angebracht
und auf die Larmorfrequenz gestellt. 
Mittels der dauer laesst sich der des Pulses laesst sich die Auslenkung der
Magnetisierung aus der Gleichgewichtslage bestimmen. 
\begin{figure}[ht]
		\centering
		\includegraphics[width=0.3\linewidth]{./build/aufbau.pdf}
		\caption{Aufbau zur Bestimmung der gepulsten Kernspinresonanz}
		\label{fig:aufbau}
\end{figure}
\subsection{Freie Induktionszerfall}%
\label{sub:freie_induktionszerfall}
Durch das einschalten der Hochfrequenz fuer $\Delta t_{90}$ wird die
Magnetisierung in sie x-y-Ebene gebracht, fuehrt dort praezessionbewegungen aus und relaxiert anschliessend wieder in
die Gleichgewichtslage. 
Das relaxieren der Spins nach einem \SI{90}{\degree} Puls in die
Gleichgewichtslaage wird als freier Induktionszerfall bezeichnet und ist in
abbildung \ref{fig:sign} im Zeitraum 0 bis $\tau$ eingezeichnet.
Ursachen fuer den freien Induktionszerfall sind einerseits Naechste nachbarn
wechselwirkungen, als auch Feldinhomogenitaeten, so dass es zur dephasierung
kommt.
Die Messbare Relaxationszeit $T_2^*$ betraegt dann
\begin{equation}
		\frac{1}{T_2^*} = \frac{1}{T_2} + \frac{1}{T_{\Delta B}}
\end{equation}
wobei $T_{\Delta B}$ eine apparative Groesse ist. 
Fuer hinreichend homogene Felder kann dennoch die Relaxationszeit bestimmt
werden. 
\subsection{Spin-Echo-Verfahren}%
\label{sub:spin_echo_verfahren}
Mittels der Spin-Echo-Methode koennen durch einen zweiten Puls zeitlich
konstante Stoereffekte zur Relaxationszeitmessung eliminiert werden. 
Zuerst werden die Spins um \SI{90}{\degree} in die y-Ebene gedreht und laufen
darauf zurueck in die Gleichgewichtslage.
Dabei kommt es zur Dephasierung der Lamorfrequenz. 
Nachdem die Spins soweit dephasiert sind das kein Induktionsstrom mehr gemessen
werden kann wird mittels eines 180 grad kick diese umgedreht, sodass diese
wieder zusammen laufen.
\begin{figure}[h]
		\centering
		\includegraphics[width=0.8\linewidth]{./build/signalverlauf.pdf}
		\caption{}
		\label{fig:sign}
\end{figure}
Dabei treffen sich die dephasierten Spins zum Zeitpunkt $2 \tau$ wieder und
erzeugen ein Induktionsmaxima der hoehe 
\begin{equation}
		\label{eq:}
		M_\text{y}(t) = M_0 \exp\left( \frac{t}{T_2} \right)
\end{equation}
Die Exponentielle Abnahme beruht auf irreversible Wechselwirkung der Spins mit
ihrer Umgebung. 

\subsection{Carr-Purcell-Methode}%
\label{sub:car_purcell}
Mittels der Carr-Purcell-Methode laesst sich durch Refokussierung der Echos die
Amplitude meherer Echos messen. 
Dazu muessen die 180 grad pulse in aequidistanten abstaenden von 2n$\tau$
kommen. 
Ist der Zeitpunkt des 180 Grad pulses nicht exakt justiert dreht dieser die
MAgnetieseierug nicht mehr in die x-y-ebene. 
Es kommt zur verstaerkung dieses fehlers weil immer in dieselbe Richtung weiter
gedreht wird.

\begin{figure}[ht]
		\centering
		\includegraphics[width=0.9\linewidth]{./build/pulssequenz.pdf}
		\caption{}
		\label{fig:}
\end{figure}

\subsection{Meiboom-Gill-Methode}%
\label{sub:meiboom_gill_methode}
Die addition der Fehler kann durch eine spezielle Vorrichtung bei der
Meiboom-Gill-Methode vermieden werden.
Dazu wird die Phase des 180 Grad pulses um 90 grad gegenueber dem 90 grad puls phasenverschoben.
Dies hatt zur Folge das die Fehljustierung des Spins um 180 grad + delta gedreht
wird und dadurch leicht herausklappt. 
Zum Zeitpunkt $2 \tau$ kommt es zur refokussierung in y-richtung.
Fuer Gradzahlige Echos ergibt sich die richtige Amplitude.

\subsection{Spin-Gitter-Relaxationszeit}%
\label{sub:spin_gitter_relaxationszeit}
Zur bestimmung der Longitudinalen Relaxationszeit wird zunaechst ein 180 grad
puls eingesetzt. 
Dieser dreht die Magnetisierung der Probe.
In folge der Relaxation wird nach einer zeit $\tau$ ein zweiter Puls auf die
Probe gegeben um die restliche z-Komponente zu messen. 
Aus den Anfangsbedingungen $M_x(0) = M_y(0) =0, M_z(0)=-M_0$ ergibt sich aus der
Blochgleichung 
\begin{equation}
		\label{eq:magn}
		M_\text(z) = M_0 \left(1 - 2 \exp \left(- \frac{\tau}{T_1} \right)
		\right)
\end{equation}
Durch Variation der Zeit zwischen dem 90 und 180 grad puls kann durch Messung
des Signals die Relaxationszeit $T_1$ bestimmt werden.

