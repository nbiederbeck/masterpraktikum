\section{Messmethoden}%
\label{sec:messmethoden}
In Abbildung~\ref{fig:aufbau} ist der Aufbau der gepulsten Kernspinresonanz zu
sehen. 
Dazu wird eine Probe zwischen zwei Polschuhe gebracht und ein magnetisches Feld
angelegt.
Senkrecht zum magnetischen Feld wird ein weiteres hochfrequentes Feld angebracht
und auf die Larmorfrequenz gestellt. 
Mittels der Dauer des Pulses lässt sich die Auslenkung der
Magnetisierung aus der Gleichgewichtslage einstellen. 
\begin{figure}[h]
		\centering
		\includegraphics[width=0.3\linewidth]{./build/aufbau.pdf}
		\caption{Aufbau zur Bestimmung der gepulsten Kernspinresonanz.
		\cite{anleitung}}%
		\label{fig:aufbau}
\end{figure}
Des Weiteren wird die Induktionsspannung in Abhängigkeit der Zeit entlang der $x-y$-Ebene gemessen,
um aus ihr die Relaxationszeiten $T_1$ und $T_2$ zu berechnen.

\subsection{Freier Induktionszerfall}%
\label{sub:freie_induktionszerfall}
Durch das Einschalten der Hochfrequenz für $\Delta t_{90}$ wird die
Magnetisierung in die $x$-$y$-Ebene gebracht, führt dort Präzessionsbewegungen aus und 
relaxiert wieder in die Gleichgewichtslage. 
Das Relaxieren der Spins nach einem \SI{90}{\degree} Puls in die
Gleichgewichtslage wird als freier Induktionszerfall bezeichnet und ist in
Abbildung~\ref{fig:sign} im Zeitraum 0 bis $\tau$ eingezeichnet.
Ursachen für den freien Induktionszerfall sind einerseits nächste Nachbar
Wechselwirkungen, als auch Feldinhomogenitäten, so dass es zur Dephasierung
entlang der $z$-Achse kommt.
Die Messbare Relaxationszeit $T_2^*$ beträgt dann
\begin{equation}
		\frac{1}{T_2^*} = \frac{1}{T_2} + \frac{1}{T_{\Delta B}}
\end{equation}
wobei $T_{\Delta B}$ eine apparative Größe ist. 
Für hinreichend homogene Felder kann dennoch die Relaxationszeit $T_2$ 
bestimmt werden. 
\subsection{Spin-Echo-Verfahren}%
\label{sub:spin_echo_verfahren}
Mittels der Spin-Echo-Methode können durch einen zweiten Puls zeitlich
konstante Störeffekte zur Relaxationszeitmessung eliminiert werden. 
Mittels eines \SI{90}{\degree} Puls werden die Spins in die y-Ebene gedreht
und relaxieren zurück in die Gleichgewichtslage.
Dabei kommt es zur Dephasierung der Larmorfrequenz. 
Nachdem die Spins soweit dephasiert sind das kein Induktionsstrom mehr gemessen
werden kann, wird mittels eines \SI{180}{\degree} Pulses die Spinorientierung umgeklappt, 
sodass diese wieder zusammen laufen.
\begin{figure}[h]
		\centering
		\includegraphics[width=0.8\linewidth]{./build/signalverlauf.pdf}
		\caption{Induktionsspannungsverlauf in Abhängigkeit der Zeit bei dem
		Spin-Echo-Verfahren. \cite{anleitung}}%
		\label{fig:sign}
\end{figure}
Dabei treffen sich die dephasierten Spins zum Zeitpunkt $2 \tau$ wieder und
erzeugen ein Induktionsmaximum der Höhe 
\begin{equation}
		\label{eq:}
		M_\text{y}(t) = M_0 \exp\left(-\frac{t}{T_2} \right)
\end{equation}
Die Exponentielle Abnahme beruht auf irreversiblen Wechselwirkungen der Spins mit
ihrer Umgebung. 

\subsection{Carr-Purcell-Methode}%
\label{sub:car_purcell}
Mittels der Carr-Purcell-Methode lässt sich durch Refokussierung der Echos die
Amplitude mehrerer Echos messen. 
Dazu müssen die \SI{180}{\degree} Pulse in äquidistanten Abständen von $2\tau$
folgen. 
Ist die Länge des \SI{180}{\degree} Pulses nicht exakt justiert dreht dieser die
Magnetisierung nicht mehr in die x-y-Ebene. 
Es kommt zur Verstärkung dieses Fehlers weil immer in dieselbe Richtung weiter
gedreht wird.

\begin{figure}[ht]
		\centering
		\includegraphics[width=0.9\linewidth]{./build/pulssequenz.pdf}
		\caption{Induktionsspannungsverlauf in Abhängigkeit der Zeit bei der
		Meiboom-Gill-Methode. \cite{anleitung}}%
		\label{fig:}
\end{figure}

\subsection{Meiboom-Gill-Methode}%
\label{sub:meiboom_gill_methode}
Die Addition der Fehler kann durch eine spezielle Vorrichtung bei der
Meiboom-Gill-Methode vermieden werden.
Dazu wird die Phase des \SI{180}{\degree} Pulses um \SI{90}{\degree} gegenüber
dem \SI{90}{\degree} Puls phasenverschoben.
Dies hat zur Folge, dass die Fehljustierung des Spins um $\SI{180}{\degree} 
+ \delta$ gedreht wird und dadurch leicht herausklappt. 
Zum Zeitpunkt $n \cdot 4 \tau$ kommt es zur Refokussierung in $y$-Richtung.
Für gradzahlige Echos ergibt sich die richtige Amplitude.

\subsection{Spin-Gitter-Relaxationszeit}%
\label{sub:spin_gitter_relaxationszeit}
Zur Bestimmung der longitudinalen Relaxationszeit wird zunächst ein
\SI{180}{\degree}
Puls eingesetzt. 
Dieser dreht die Magnetisierung der Probe entlang der $z$-Achse um, sodann die
Spins wieder gegen ihre Gleichgewichtslage relaxieren.
Nach einer Zeit $\tau$ wird ein zweiter Puls auf die
Probe gegeben um die restliche z-Komponente zu messen. 
Aus den Anfangsbedingungen $M_x(0) = M_y(0) =0, M_z(0)=-M_0$ ergibt sich aus der
Blochgleichung 
\begin{equation}
		\label{eq:magn}
		M_\text{z}(\tau) = M_0 \left(1 - 2 \exp \left(- \frac{\tau}{T_1} \right)
		\right)
\end{equation}
Durch Variation der Zeit zwischen dem 90 und 180 grad Puls kann durch Messung
des Signals die Relaxationszeit $T_1$ bestimmt werden.

\subsection{Viskosimeter}%
\label{sub:viskosimeter}
Zur Viskosität-Messung wird ein Kapillar-Viskosimeter genutzt. 
Dazu wird ein Volumen mit der zu bestimmenden Flüssigkeit befüllt, welche
anschließend über die Kapillare abläuft.
Beim Ablaufprozess wird der Druck konstant auf Normaldruck gehalten. 
Da Reibungeffekt den Ablauf durch die Kapillare bestimmen, fließen viskosere
Flüssigkeiten langsamer ab, als weniger viskose Flüssigkeiten.
Durch Messung der Zeit kann durch die Proportionalität der Viskosität zur Zeit
diese bestimmt werden.
